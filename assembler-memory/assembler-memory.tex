\documentclass{article}
\usepackage[utf8]{inputenc}
\usepackage[brazil]{babel}
\usepackage[left=1cm, right=1.5cm, top=5cm, bottom=5cm]{geometry}
\usepackage{cancel}
\usepackage{tikz}
\usepackage{makeidx}
\usepackage{graphicx}
\usepackage{hyperref}
\renewcommand{\thefootnote}{\alph{footnote}}
\setlength{\parskip}{\baselineskip}
\setlength{\parindent}{0pt}

\hypersetup {
  colorlinks,
  citecolor = NavyBlue,
  filecolor = NavyBlue,
  linkcolor = NavyBlue,
  urlcolor = NavyBlue
}

\author{Servente}

\title{Montador e sua relação com a Memória}

\begin{document}

\maketitle

\begin{enumerate}

\item[pg 118] Uma variável C é genericamente uma localização na memória, e sua 
interpretação depende de seu tipo classe de armazenamento. Exemplos incluem 
inteiros, caracteres, etc. C tem duas classes de armazenamento: automático  
\item[pg 121] \textbf{Alocando Espaço para Novos Dados na Heap}: Quando 
adicionamos variáveis automáticas que são locais no escopo de procedimentos, os 
programadores C precisarão de espaço em memória para variáveis estáticas e para 
estruturas de dados dinâmicos. \textbf{A pilha} começa no maior endereço e 
aumenta em direção ao menor endereço. A primeira 

\begin{verbatim}
$sp -> 7fff fffc hex ------>+--------------+
                            |    Stack     |
                            |      \/      |
                            |              |
                            |              |
                            |              |
                            |      /\      |
                            | Dynamic data |
                            +--------------+
                            |              |
$gp -> 1000 8000 hex ------>| Static data  |
                            |              |
       1000 0000 hex ------>|              |
                            +--------------+
                            |    Text      |
 pc -> 0040 0000 hex ------>|              |
                            +--------------+
                            |   Reserved   |
               0 hex        +--------------+
\end{verbatim}

\item[139-142]
\item[B10-B17]
\item[B20-B22]

\end{enumerate}

\paragraph{2.30.1}

\paragraph{2.21.2}

\paragraph{2.21.1}

\end{document}
