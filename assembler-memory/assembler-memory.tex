\documentclass{article}
\usepackage[utf8]{inputenc}
\usepackage[brazil]{babel}
\usepackage[left=1cm, right=1.5cm, top=5cm, bottom=5cm]{geometry}
\usepackage{cancel}
\usepackage{tikz}
\usepackage{makeidx}
\usepackage{graphicx}
\usepackage{hyperref}
\renewcommand{\thefootnote}{\alph{footnote}}
\setlength{\parskip}{\baselineskip}
\setlength{\parindent}{0pt}

\hypersetup {
  colorlinks,
  citecolor = NavyBlue,
  filecolor = NavyBlue,
  linkcolor = NavyBlue,
  urlcolor = NavyBlue
}

\author{Servente}

\title{Montador e sua relação com a Memória}

\begin{document}

\maketitle

\begin{enumerate}

\item[pg 118] Uma variável C é genericamente uma localização na memória, e sua 
interpretação depende de seu tipo classe de armazenamento. Exemplos incluem 
inteiros, caracteres, etc. C tem duas classes de armazenamento: automático  
\item[pg 121] \textbf{Alocando Espaço para Novos Dados na Heap}: Quando 
adicionamos variáveis automáticas que são locais no escopo de procedimentos, os 
programadores C precisarão de espaço em memória para variáveis estáticas e para 
estruturas de dados dinâmicos. \textbf{A pilha} começa no maior endereço e 
aumenta em direção ao menor endereço. A primeira 

\begin{verbatim}
$sp -> 7fff fffc hex ------>+--------------+
                            |    Stack     |
                            |      \/      |
                            |              |
                            |              |
                            |              |
                            |      /\      |
                            | Dynamic data |
                            +--------------+
                            |              |
$gp -> 1000 8000 hex ------>| Static data  |
                            |              |
       1000 0000 hex ------>|              |
                            +--------------+
                            |    Text      |
 pc -> 0040 0000 hex ------>|              |
                            +--------------+
                            |   Reserved   |
               0 hex        +--------------+
\end{verbatim}

Imediatamente acima da área estática é a área dinâmica de dados. Essa área, 
pelo próprio nome já dá para entender, implica dados alocados pelo programa em 
tempo de execução. Um programa.c a rotina \verb|malloc| usa essa área. Por 
causa que segmento de dados começa muito longe do programa, no endereço 
10000000 hex, instruções \verb|load| e \verb|store| não podem referenciar 
diretamente objetos de dados com seus 16-bit de offset.

Como a flecha para cima (\verb|/\|) indica, \verb|malloc| expande área dinâmica 
de dados com a chamada de sistema \verb|sbrk|, a qual diz para o sistema 
operacional adicionar mais páginas para o espaço virtual do programa, 
imediatamente acima do espaço dinâmico de dados.

Por exemplo, para carrega o valor do endereço de dados 10010020 hex em um 
registrador \$v0 requer duas instruções:

\begin{verbatim}
lui $s0, 0x1001      # 0x1001 significa 1001 base 16
lw  $v0, 0x0020($s0) # 0x10010000 + 0x0020 = 0x10010020
\end{verbatim}

\item[139] O compilador transforma código C em linguagem de montagem, uma forma 
simbólica para que a máquina possa processar. A máquina entende linguagem de 
montagem, não entende código em alto nível. Nós utilizamos código de alto nível 
para obter maior rendimento e precisão em desenvolvimento de software.

\item[140] A transformação hierárquica para um programa em C acontece da 
seguinte forma: Um código.c passa pelo \underline{compilador} e vira um 
programa em linguagem de montagem que passa então pelo \underline{montador} e 
vira uma objeto, um módulo em linguagem de máquina juntamente a esse objeto 
temos outros objetos da biblioteca local (já compiladas e montadas) então o 
\underline{ligador} associa ambas e vira um executável em linguagem de máquina 
e então passa pelo \underline{carregador} e aí então vai para a memória.

\item[141] O montador traduz pseudo-instruções para instruções válidas pelo 
sistema como: \verb|move $t0, $t1| vira \verb|add $t0, $zero, $t1|. O assembler 
MIPS também converte \verb|blt| em duas instruções \textbf{slt} e \textbf{bne}. 
Traduz também: branch para uma distância muito longe em um par de branch mais 
jump para poder saltar longas distâncias. Pseudo-instruções dão uma 
flexibilidade a mais para quem precisa realmente programar em linguagem de 
montagem. O arquivo objeto para uma sistema UNIX, contem tipicamente:

\begin{enumerate}
\item Um cabeçalho (tamanhos e posições).
\item Um segmento de texto (programa, lógica).
\item Uma área de dados estáticos (dados para o programa).
\item Informações de realocação (identificadores das palavras de dados que 
dependem de endereço absoluto).
\item Uma tabela de símbolos (rótulos restantes que não foram definidos).
\item Informações de deputação (contém informações de como foi compilado, para 
o depurador associar instruções de máquina com \verb|código.c|).
\end{enumerate}

\item[142] O ligador faz os seguintes passos:

\begin{enumerate}
\item Insere simbolicamente na memória, módulos de código e dados.
\item Determina endereços de rótulos de dados e rótulos de instruções.
\item Empacota endereços externos mais endereços internos.
\end{enumerate}

O ligador usa informação de realocação e a tabela de símbolos em cada módulo
objeto para resolver todas as referências indefinidas. Se todas as referências
externas foram resolvidas, então o ligador poderá determinar locais na memória
que cada módulo irá ocupar. O ligador produz então o arquivo executável que pode
rodar no computador. Linguagem de montagem é linguagem de programação, a
diferença entre ela e as linguagens de alto nível como java e c é que assembler
provê apenas poucos tipos de dados e de controle de fluxo.

O objeto é a seguinte estrutura:

\begin{table}[ht!]
\centering
\begin{tabular}{|c|}
\hline \\
\begin{tabular}{|c|c|c|c|c|c|}
\hline Cabeçalho &
       Texto &
       Dados &
       Realocação &
       Tabela de Símbolos &
       Depuração \\
\hline
\end{tabular} \\ \\ \hline
\end{tabular}
\caption{arquivo.o}
\end{table}

O carregador faz os seguintes passos:

\begin{enumerate}
\item Le as informações no arquivo executável para determinar o tamanho dos 
segmentos de dados e texto.
\item Cria um novo espaço de endereçamento para o programa.
\item Copia as instruções e os dados do arquivo executável para o novo espaço 
de endereçamento.
\item Copia argumentos passados para o programa para dentro da pilha.
\item Inicializa os registradores da máquina (Java não usa registradores, por 
isso é que Java não inicializa as variáveis automaticamente, já C faz isso).
\item Pula para uma rotina de inicialização
\end{enumerate}

\end{enumerate}

\paragraph{\cancel{2.21.1} $\rightarrow$ 2.21.1a} Traduzir o código em C abaixo 
para linguagem de montagem, respeitando as convenções e padrões, etc.

Código em linguagem C---alto nível:

\begin{verbatim}
int my_global = 100;
main() {
  int x = 10;
  int y = 20;
  int z;
  z = my_function(x, y)
}

int my_function(int x, int y) {
  return x - y + my_global;
}
\end{verbatim}

Tradução para MIPS---linguagem de montagem:

\begin{verbatim}
my_global: .word 100

main: addi $t0, $0, 10
      addi $t1, $0, 20
      addi $t3, $0, 0

      addi $a0, $0, $t0
      addi $a1, $0, $t1            # (a0, a1) <-- (x, y)

      jal my_function              # salta para rotina folha
      addi $t3, $0, $v0

my_function: addi $sp, $sp, -12
             sw   $ra, 0($sp)
             sw   $s0, 4($sp)
             sw   $s1, 8($sp)      # salva contexto

             la $s1, my_global
             lw $s1, 0($s1)        # busca my_global

             sub $s0, $a0, $a1     # x - y
             add $v0, $s0, $s1     # x - y + my_global

             lw   $s1, 8($sp)      # restaura contexto
             lw   $s0, 4($sp)
             lw   $ra, 0($sp)
             addi $sp, $sp, 12

             jr $ra
\end{verbatim}

\paragraph{\cancel{2.21.2}} Acho que não preciso fazer essa questão (olhando na 
quarta edição revisada, pode ser que na edição não revisada seja uma questão 
diferente).

\paragraph{\cancel{2.30.1}} Na edição 4 revisada essa questão é uma de traduzir 
pseudo instruções.

\end{document}
