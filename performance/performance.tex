\documentclass{article}
\usepackage[utf8]{inputenc}
\usepackage[brazil]{babel}
\usepackage[left=1cm, right=1.5cm, top=5cm, bottom=5cm]{geometry}
\usepackage{cancel}
\usepackage{makeidx}
\usepackage{graphicx}
\usepackage{hyperref}
\usepackage[usenames,dvipsnames]{xcolor}
\renewcommand{\thefootnote}{\alph{footnote}}
\setlength{\parskip}{\baselineskip}
\setlength{\parindent}{0pt}

\hypersetup {
  colorlinks,
  citecolor = NavyBlue,
  filecolor = NavyBlue,
  linkcolor = NavyBlue,
  urlcolor = NavyBlue
}

\author{Lucas}

\title{Performance}

\begin{document}

\maketitle

\begin{enumerate}

\item[pg 26] O transistor é um interruptor simples de ligar/desligar, controlado
por eletricidade. O circuito integrado (CI) combina dezenas de centenas de
transistores em um único chip. VLSI (milhares de transistores), chamamos de
circuito integrado de larguíssima escala (very large-scale integrated circuit).

\item[pg 28] Tempo de Resposta (Responde Time) também chamado Tempo de Execução.
O tempo total requerido para o computador completar uma tarefa qualquer,
incluindo acessos ao disco, memória, I/O, overhead do sistema operacional, tempo
de execução da CPU, etc.

Vazão (Throughput) também chamado de largura de banda. É outra medida de
desempenho, é o número de tarefas completadas por unidade de processamento.

\item[pg 29] Na página 29 temos uma equação simples, comparativa: 
$Performance_{x} = \frac{1}{Execution\ Time_{x}}$ Se o desempenho de $x$ é 
melhor que o de $y$ então $Performance_{x} > Performance_{y}$ ou ainda 
$\frac{1}{Execution\ Time_{x}} > \frac{1}{Execution\ Time_{y}}$ ou também 
$Execution\ Time_{x} < Execution\ Time_{y}$. Isso significa que o tempo de 
execução em $y$ é mais longo que o tempo de execução em $x$, bastante simples.

$$\frac{Performance_{x}}{Performance_{y}} = \frac{Execution\
Time_{y}}{Execution\ Time_{x}} = n$$

\item[pg 30] O conceito de medir desempenho de computador se baseia em:  
entender qual montante de (trabalho) ele estará processando, em um  dado tempo. 
O processador que realizar o mesmo montante no menor tempo, teoricamente é mais 
rápido. Tempo de execução do programa é medido em segundos por programa. A 
definição mais direta é chamada de 'wall clock time', 'response time', ou 
'elapsed time'. Esses termos significam o total para uma tarefa ser completada.

\item[pg 31] Por consistência, vamos manter a distinção entre desempenho, 
baseado em 'elapsed time' e por sua vez baseada em CPU 'execution time'. Clock 
cycle (Ciclos de Relógio) também chamado de 'clock tick', 'period', 'cycle'. É 
o tempo para um período de relógio (geralmente em picosegundos, microsegundos, 
nanosegundos, ou segundos, depende do problema), usualmente o relógio do 
processador, o qual está rodando à uma certa taxa constante (por exemplo: 4 GHz)

\item[pg 32] Uma fórmula simples que relaciona fatores simples é:

$$CPU\ execution\ time\ for\ a\ program\ = CPU\ clock\ cycles\ for\ a\ program 
\times Clock\ cycle\ time$$

Um exemplo genérico de melhoramento de performance usando essa fórmula acima é:

$$CPU\ time_{A} = \frac{CPU\ clock\ cycles_{A}}{Clock\ rate_{A}}$$

$$\frac{Seconds}{Program} = \frac{CPU\ clock\ cycles_{A}}{f \times 10^{n}\ 
Hz}$$

$$CPU\ clock\ cycles_{A} = (t\ \textbf{[s]}) \times (f \times 
10^{n}\ \textbf{[Hz]})$$

Onde $t$ é por exemplo: $12$ segundos, ou 3 nanosegundos. E $f \times
10^{n}$ é por exemplo: $9\times10^{9}$ hertz, ou 9 Gigahertz. E assim por
diante.

$$\frac{Seconds}{Program} = \frac{(CPU\ clock\ cycles_{B} = CPU\ clock\ 
cycles_{A} \times scalar\ cost)}{Clock\ Rate_{B}}$$

assim, podemos simplificar para:

$$\frac{Seconds}{Program} = \frac{(CPU\ clock\ cycles_{A} \times 
scalar\ cost)}{Clock\ Rate_{B}}$$

Ou seja se eu tenho um computador A que processa um programa qualquer em $13$ 
segundos, sendo que esse tem $3$ GHz. Para que eu gere um computador B que 
processe o mesmo $30\%$ mais rápido que o computador A. Eu precisaria que ele 
tivesse 4,68 GHz se quisemos que o CPU B processasse a mesma informação em 10 
segundos (3 segundos a menos, melhoria de $30\%$), pagando um custo de 1,4x 
ciclos a mais, aí vai do seu critério, \underline{quanto você quer pagar a 
mais}?

\begin{table}[ht!]
  \centering
  \begin{tabular}{|l|l|}
  \hline Changes & $\frac{Seconds}{Program} = \frac{(CPU\ clock\ cycles_{A} 
  \times scalar\ cost)}{Clock\ Rate_{B}}$ \\ 
  \hline $\uparrow {\frac{Seconds}{Program}}_{B}$ (e.g: $3\ s\rightarrow 
  4\ s$) & Decresce $Clock\ rate_{B}$ \\ 
  \hline $\uparrow (CPU\ clock\ cycles_{A} \times 
  improvement)$ & Aumenta $Clock\ rate_{B}$ \\ 
  \hline 
  \end{tabular}
\end{table}

\item[pg 33] O termo clock cycles per intruction CPI, o qual é a média numérica 
de ciclos de clock para cada instruções executada, varia bastante de classes de 
instruções para classes de instruções, exemplos o CPI de instruções do tipo R 
não vai ser o mesmo CPI para instruções do tipo I, de fato instruções do tipo I 
contém em seu corpo branch, loads, stores, essas instruções contém veneno para 
o datapath, da forma que ele leva mais ciclos para processar.

$$CPU\ clock\ cycles = Instructions\ for\ a\ program \times Average\ clock\ 
cycles\ per\ instruction$$

Suponha você agora que temos um processador X com tempo para cada ciclo de 
relógio de $N$ pico segundos e em média faz $s$ ciclos por instrução (CPI), e 
um processador Y com tempo de clock cycle de $K$ pico segundos e em média faz 
$p$ ciclos por instrução (CPI). Quem é mais rápido?

$CPU\ clock\ cycles_{A} = (I\ \textbf{instr.} \times s\ \textbf{CPI})\ 
\textbf{ciclos}$

$CPU\ clock\ cycles_{B} = (I\ \textbf{instr.} \times p\ \textbf{CPI})\ 
\textbf{ciclos}$

$CPU\ time_{A} = CPU\ clock\ cycles_{A} \times Clock\ cycle\ time$\footnote{O 
fator tempo de ciclo de relógio, é o tempo de 1 ciclo de relógio. O montante 
total de ciclos para executar um programa completamente, executar uma tarefa 
qualquer, vai levar (pensando logicamente) $n\ \textbf{cycles} \times clock\ 
cycle\ time$.}

$CPU\ time_{A} = I \times s \times N\ \textbf{pico segundos}$

$CPU\ time_{B} = CPU\ clock\ cycles_{B} \times Clock\ cycle\ time$

$CPU\ time_{B} = I \times p \times K\ \textbf{pico segundos}$

Basta comparar ambos os resultados de tempo de CPU e ver qual se desempenhou 
melhor.

$$\frac{CPU\ time_{B}}{CPU\ time_{A}} = \frac{I \times p \times K}{I \times s 
\times N} = \frac{p\ \textbf{CPI} \times K\ \textbf{pico segundos}}{s\ 
\textbf{CPI} \times N\ \textbf{pico segundos}}$$

Se por acaso, $p \times K > s \times N$ então $CPU_{B}$ se desempenha melhor no 
conjunto de instruções que você passou como parâmetro. Caso contrário $CPU_{A}$ 
se desempenha melhor.

$1,7\ \textbf{CPI} \times 190\ \textbf{pico segundos} < 2\ \textbf{CPI} \times 
180\ \textbf{pico segundos}$ pois $323\ \textbf{pico segundos}  < 360\ 
\textbf{pico segundos}$ então $CPU_{A}$ tem 11\% menos desempenho que $CPU_{B}$.

\item[pg 35] Equação de Desempenho Clássica

$$CPU\ time = Instruction\ count \times CPI \times Clock\ cycle\ time$$
$$CPU\ time = \frac{Instruction\ count\ \times CPI}{Clock\ rate}$$

Essas fórmulas são bastante úteis pois elas separam os três fatores que afetam 
o desempenho. Nós podemos usar essas fórmulas para comparar duas implementações 
diferentes ou também para avaliar uma alternativa se nós sabemos os impactos 
nesses três parâmetros.\\

$CPU\ clock\ cycles = \sum_{i=1}^{n}(CPI_{i} \times C_{i})$\footnote{CPI é 
uma média de ciclos por instruções.}\\

Por exemplo: $CPU\ clock\ cycles = 2 \times 3 + 1,7 \times 9 + 1 \times 5 =
26$ ciclos de clock. Veja que poderíamos considerar as contagens de cada classe
como sendo $\times 10^{6}$ instruções $CPU\ clock\ cycles = 2 \times 3\ \times
10^{6} + 1,7 \times 9\ \times 10^{6} + 1 \times 5\ \times 10^{6} = 26\ \times
10^{6}$ ciclos de clock. Não tem problema nenhum. Para calcular o CPI geral,
temos a seguinte fórmula $CPI_{total} = \frac{CPU\ clock\ cycles}{Instruction\
count}$. Ora, então temos: Contagem 1 $3+9+5=17$ e aplicando a fórmula temos o 
seguinte: $CPI_{total} = \frac{26}{17}=1,5$

$$Time=\frac{Seconds}{Program}=\frac{Instructions}{Program}\times\frac{Clock\ 
cycles}{Instruction}\times\frac{Seconds}{Clock\ cycle}$$

Sempre tenha em mente que a medida confiável de medição do desempenho 
(performance) é \underline{Tempo}. Por exemplo: mudar o conjunto de instruções 
para uma contagem menor, pode liderar para tempo do ciclo de relógio maior (ou 
seja, mais lento); ou também pode liderar para um maior CPI (média de ciclos 
por instruções) que infere mudanças de melhoria na contagem de instrução. 
Similarmente, por causa que CPI depende do tipo de instrução executada, 
{\bfseries \color{Red} \Large o código que executa o menor número de 
instruções pode \underline{não} ser o mais rápido}.

\end{enumerate}

\clearpage%
\section{Questões de performance}

\paragraph{1.3.1} Qual processador tem o melhor desempenho expresso em 
instruções por segundo?

Você pode decidir se quer contar como Milhões de Instruções por Segundo, ou 
Instruções por Segundo. Basta multiplicar por $10^{6}$ todos os resultados 
abaixo.

$MIPS_{a.P_{1}} = \frac{Clock\ Rate}{CPI \times 10^{6}} = \frac{3\ 
\small{GHz}}{1,5 \times 10^{6}} = 2 \times 10^{3}$ milhões de instruções por 
segundo.

$MIPS_{a.P_{2}} = \frac{Clock\ Rate}{CPI \times 10^{6}} = \frac{2,5\ 
\small{GHz}}{1,0 \times 10^{6}} = 2,5 \times 10^{3}$ milhões de instruções por 
segundo.

$MIPS_{a.P_{3}} = \frac{Clock\ Rate}{CPI \times 10^{6}} = \frac{4\ 
\small{GHz}}{2,2 \times 10^{6}} = 1,8 \times 10^{3}$ milhões de instruções por 
segundo.

$MIPS_{b.P_{1}} = \frac{Clock\ Rate}{CPI \times 10^{6}} = \frac{2\ 
\small{GHz}}{1,2 \times 10^{6}} =$ {\color{Red} $1,6$} $\times 10^{3}$ milhões 
de instruções por segundo.

$MIPS_{b.P_{2}} = \frac{Clock\ Rate}{CPI \times 10^{6}} = \frac{3\ 
\small{GHz}}{0,8 \times 10^{6}} =$ {\color{Blue} $3,75$} $\times 10^{3}$ 
milhões de instruções por segundo.

$MIPS_{b.P_{2}} = \frac{Clock\ Rate}{CPI \times 10^{6}} = \frac{4\ 
\small{GHz}}{2,0 \times 10^{6}} = 2 \times 10^{3}$ milhões de instruções por 
segundo.

\underline{CPI} e \underline{Frequência} trabalham juntas na vazão de 
instruções.

\paragraph{1.3.2} Achar o número de ciclos e o número de instruções para 10 
segundos de execução de uma tarefa por esses processadores.

$Clock\ Cycle\ Time_{a.P_{1}} = \frac{1}{Clock\ Rate} = \frac{1}{3\ 
\small{GHz}} = 0,33\ nanosegundos$

$Clock\ Cycle\ Time_{a.P_{2}} = \frac{1}{Clock\ Rate} = \frac{1}{2,5\ 
\small{GHz}} = 0,4\ nanosegundos$

$Clock\ Cycle\ Time_{a.P_{3}} = \frac{1}{Clock\ Rate} = \frac{1}{4\ 
\small{GHz}} = 0,25\ nanosegundos$

$Clock\ Cycle\ Time_{b.P_{1}} = \frac{1}{Clock\ Rate} = \frac{1}{2\ 
\small{GHz}} = 0,5\ nanosegundos$

$Clock\ Cycle\ Time_{b.P_{2}} = \frac{1}{Clock\ Rate} = \frac{1}{3\ 
\small{GHz}} = 0,33\ nanosegundos$

$Clock\ Cycle\ Time_{b.P_{3}} = \frac{1}{Clock\ Rate} = \frac{1}{4\ 
\small{GHz}} = 0,25\ nanosegundos$

$10\ s = I \times 1,5 \times 0,33 \times 10^{-9} \rightarrow I = 20$ bilhões de 
instruções.

$10\ s = I \times 1,0 \times 0,4 \times 10^{-9} \rightarrow I = 25$ bilhões de 
instruções.

$10\ s = I \times 2,2 \times 0,25 \times 10^{-9} \rightarrow I = 18$ bilhões de 
instruções.

$10\ s = I \times 1,2 \times 0,5 \times 10^{-9} \rightarrow I = 16$ bilhões de 
instruções.

$10\ s = I \times 0,8 \times 0,33 \times 10^{-9} \rightarrow I = 37$ bilhões de 
instruções.

$10\ s = I \times 2,0 \times 0,25 \times 10^{-9} \rightarrow I = 20$ bilhões de 
instruções.

\paragraph{1.3.4} O ICP é o inverso do CPI então, lembrar que se CPI é uma 
média de ciclos por instrução, ICP também é uma média de instruções por ciclo.

$Tempo = 7 = {Nro.\ Instr}\times{CPI}\times\frac{1}{Freq.\ Relog.} \rightarrow 
\frac{7 \times 3 \times 10^{9}}{20 \times 10^{9}} = CPI = 1,0 \rightarrow ICP = 
\frac{1}{CPI} = \frac{1}{1} = 1$

$Tempo = 10 = {Nro.\ Instr}\times{CPI}\times\frac{1}{Freq.\ Relog.} \rightarrow 
\frac{10 \times 2,5 \times 10^{9}}{30 \times 10^{9}} = CPI = 0,8 \rightarrow 
ICP = \frac{1}{CPI} = \frac{1}{0,8} = 1,25$

$Tempo = 9 = {Nro.\ Instr}\times{CPI}\times\frac{1}{Freq.\ Relog.} \rightarrow 
\frac{9 \times 4 \times 10^{9}}{90 \times 10^{9}} = CPI = 0,4 \rightarrow ICP = 
\frac{1}{CPI} = \frac{1}{0,4} = 2,5$

$Tempo = 5 = {Nro.\ Instr}\times{CPI}\times\frac{1}{Freq.\ Relog.} \rightarrow 
\frac{5 \times 2 \times 10^{9}}{20 \times 10^{9}} = CPI = 0,5 \rightarrow ICP = 
\frac{1}{CPI} = \frac{1}{0,5} = 2$

$Tempo = 8 = {Nro.\ Instr}\times{CPI}\times\frac{1}{Freq.\ Relog.} \rightarrow 
\frac{8 \times 3 \times 10^{9}}{30 \times 10^{9}} = CPI = 0,8 \rightarrow ICP = 
\frac{1}{CPI} = \frac{1}{0,8} = 1,25$

$Tempo = 7 = {Nro.\ Instr}\times{CPI}\times\frac{1}{Freq.\ Relog.} \rightarrow 
\frac{7 \times 4 \times 10^{9}}{25 \times 10^{9}} = CPI = 1,1 \rightarrow ICP = 
\frac{1}{CPI} = \frac{1}{1,1} = 0,9$

\clearpage
\paragraph{1.4.1a} Dado um programa de 1 milhão de instruções divididas em 
classes de 10\% classe A, 20\% classe B, 50\% classe C, e 20\% classe D, qual 
implementação é mais rápida?\footnote{Tempo é a medida mais honesta que você 
pode ter, em performance.}

\begin{verbatim}
        Clock    CPI Class
        Rate     A  B   C  D
a.
   P1 | 2,5 GHz  1  2   3  3
   ---+---------------------
   P2 | 3,0 GHz  2  2   2  2
b.
   P1 | 2,5 GHz  2  1,5 2  1
   ---+---------------------
   P2 | 3,0 GHz  1  2,0 1  1
\end{verbatim}

$$Tempo = No.\ Instr. \times CPI \times \frac{1}{Clock\ Rate}$$

\textbf{Processador 1.}

\begin{tabular}{|c|l|l|}
\hline Classe A & $10 \times 10^{4} = 1 \times 10^{5}$ &

$Tempo = 1 \times 10^{5} \times 1 \times \frac{1}{2,5\ \textbf{\tiny GHz}} = 
0,4 \times 10^{-4}$ segundos \\

\hline Classe B & $20 \times 10^{4} = 2 \times 10^{5}$ &

$Tempo = 2 \times 10^{5} \times 2 \times \frac{1}{2,5\ \textbf{\tiny GHz}} = 
1,6 \times 10^{-4}$ segundos \\

\hline Classe C & $50 \times 10^{4} = 5 \times 10^{5}$ &

$Tempo = 5 \times 10^{5} \times 3 \times \frac{1}{2,5\ \textbf{\tiny GHz}} = 
6,0 \times 10^{-4}$ segundos \\

\hline Classe D & $20 \times 10^{4} = 2 \times 10^{5}$ &

$Tempo = 2 \times 10^{5} \times 3 \times \frac{1}{2,5\ \textbf{\tiny GHz}} = 
2,4 \times 10^{-4}$ segundos \\

\hline & & Tempo Total $10,4 \times 10^{-4}$ segundos \\

\hline
\end{tabular}

\textbf{Processador 2.}

\begin{tabular}{|c|l|l|}
\hline Classe A & $10 \times 10^{4} = 1 \times 10^{5}$ &

$Tempo = 1 \times 10^{5} \times 2 \times \frac{1}{3\ \textbf{\tiny GHz}} = 
0,6 \times 10^{-4}$ segundos \\

\hline Classe B & $20 \times 10^{4} = 2 \times 10^{5}$ &

$Tempo = 2 \times 10^{5} \times 2 \times \frac{1}{3\ \textbf{\tiny GHz}} = 
1,3 \times 10^{-4}$ segundos \\

\hline Classe C & $50 \times 10^{4} = 5 \times 10^{5}$ &

$Tempo = 5 \times 10^{5} \times 2 \times \frac{1}{3\ \textbf{\tiny GHz}} = 
3,3 \times 10^{-4}$ segundos \\

\hline Classe D & $20 \times 10^{4} = 2 \times 10^{5}$ &

$Tempo = 2 \times 10^{5} \times 2 \times \frac{1}{3\ \textbf{\tiny GHz}} = 
1,3 \times 10^{-4}$ segundos \\

\hline & & Tempo Total $6,5 \times 10^{-4}$ segundos \\

\hline
\end{tabular}

Nós constatamos que o Processador 1 leva mais tempo para completar a execução 
das instruções (separadas em classes).

\clearpage
\paragraph{1.4.1b} Dado um programa de 1 milhão de instruções divididas em 
classes de 10\% classe A, 20\% classe B, 50\% classe C, e 20\% classe D, qual 
implementação é mais rápida?\footnote{Tempo é a medida mais honesta que você 
pode ter, em performance.}

$$Tempo = No.\ Instr. \times CPI \times \frac{1}{Clock\ Rate}$$

\textbf{Processador 1.}

\begin{tabular}{|c|l|l|}
\hline Classe A & $10 \times 10^{4} = 1 \times 10^{5}$ &

$Tempo = 1 \times 10^{5} \times 2 \times \frac{1}{2,5\ \textbf{\tiny GHz}} = 
0,8 \times 10^{-4}$ segundos \\

\hline Classe B & $20 \times 10^{4} = 2 \times 10^{5}$ &

$Tempo = 2 \times 10^{5} \times 1,5 \times \frac{1}{2,5\ \textbf{\tiny GHz}} = 
1,2 \times 10^{-4}$ segundos \\

\hline Classe C & $50 \times 10^{4} = 5 \times 10^{5}$ &

$Tempo = 5 \times 10^{5} \times 2 \times \frac{1}{2,5\ \textbf{\tiny GHz}} = 
4,0 \times 10^{-4}$ segundos \\

\hline Classe D & $20 \times 10^{4} = 2 \times 10^{5}$ &

$Tempo = 2 \times 10^{5} \times 1 \times \frac{1}{2,5\ \textbf{\tiny GHz}} = 
0,8 \times 10^{-4}$ segundos \\

\hline & & Tempo Total $6,8 \times 10^{-4}$ segundos \\

\hline
\end{tabular}

\textbf{Processador 2.}

\begin{tabular}{|c|l|l|}
\hline Classe A & $10 \times 10^{4} = 1 \times 10^{5}$ &

$Tempo = 1 \times 10^{5} \times 1 \times \frac{1}{3\ \textbf{\tiny GHz}} = 
0,3 \times 10^{-4}$ segundos \\

\hline Classe B & $20 \times 10^{4} = 2 \times 10^{5}$ &

$Tempo = 2 \times 10^{5} \times 2 \times \frac{1}{3\ \textbf{\tiny GHz}} = 
1,3 \times 10^{-4}$ segundos \\

\hline Classe C & $50 \times 10^{4} = 5 \times 10^{5}$ &

$Tempo = 5 \times 10^{5} \times 1 \times \frac{1}{3\ \textbf{\tiny GHz}} = 
1,6 \times 10^{-4}$ segundos \\

\hline Classe D & $20 \times 10^{4} = 2 \times 10^{5}$ &

$Tempo = 2 \times 10^{5} \times 1 \times \frac{1}{3\ \textbf{\tiny GHz}} = 
0,6 \times 10^{-4}$ segundos \\

\hline & & Tempo Total $3,8 \times 10^{-4}$ segundos \\

\hline
\end{tabular}

\paragraph{1.4.2a} What is the global CPI for each implementation?

$CPI_{a.P_{1}} = \frac{\sum_{i=0}^{3} (CPI_{i} \times Contagem\
Instr_{i})}{Total\ Instr} = \frac{(1 \times 10^{4} \times 10 + 2 \times 10^{4}
\times 20 + 3 \times 10^{4} \times 50 + 3 \times 10^{4} \times 20)}{10^{6}} =
\frac{260 \times 10^{4}}{10^{6}} = 2,6$ ciclos por instrução (em média e
considerando separação por classes).

$CPI_{a.P_{2}} = \frac{\sum_{i=0}^{3} (CPI_{i} \times Contagem\
Instr_{i})}{Total\ Instr} = \frac{(2 \times 10^{4} \times 10 + 2 \times 10^{4}
\times 20 + 2 \times 10^{4} \times 50 + 2 \times 10^{4} \times 20)}{10^{6}} =
\frac{200 \times 10^{4}}{10^{6}} = 2$ ciclos por instrução (em média e
considerando separação por classes).

O $CPI_{a.P_{1}} > CPI_{a.P_{2}}$ possivelmente terá melhor desempenho, porém. 
Essa análise confirma para essas configurações de conjuntos de instruções 
separados por classe. Outros casos podem alterar o resultado. Infelizmente 
$a.P_{1}$ obteve $30\%$ melhor desempenho, para essa configuração da tabela, do 
que $a.P_{2}$, o que é uma pena, pois eu estava confiante que $a.P_{2}$ iria 
vencer.

\clearpage

\paragraph{1.4.2b} What is the global CPI for each implementation?

$CPI_{b.P_{1}} = \frac{\sum_{i=0}^{3} (CPI_{i} \times Contagem\
Instr_{i})}{Total\ Instr} = \frac{(2 \times 10^{4} \times 10 + 1,5 \times 10^{4}
\times 20 + 2 \times 10^{4} \times 50 + 1 \times 10^{4} \times 20)}{10^{6}} =
\frac{170 \times 10^{4}}{10^{6}} = 1,7$ ciclos por instrução (em média e
considerando separação por classes).

$CPI_{b.P_{2}} = \frac{\sum_{i=0}^{3} (CPI_{i} \times Contagem\
Instr_{i})}{Total\ Instr} = \frac{(1 \times 10^{4} \times 10 + 2 \times 10^{4}
\times 20 + 1 \times 10^{4} \times 50 + 1 \times 10^{4} \times 20)}{10^{6}} =
\frac{120 \times 10^{4}}{10^{6}} = 1,2$ ciclos por instrução (em média e
considerando separação por classes).

A primeira implementação obteve melhor resultado, interessante.

\paragraph{1.5.1a} Considere duas implementações diferentes, P1 e P2, do mesmo 
conjunto de instruções (mesma arquitetura). Existem 5 classes de instruções (A, 
B, C, D, e E) nesse conjunto de instruções. A frequência de relógio e o CPI de 
cada classe é dado abaixo:

\begin{tabular}{|c|c|c|c|c|c|c|c|}
  \hline  &
          & Clock Rate
          & CPI Class A
          & CPI Class B
          & CPI Class C
          & CPI Class D
          & CPI Class E \\
  \hline    a.
          & P1
          & 2,0 GHz
          & 1
          & 2
          & 3
          & 4
          & 3 \\
  \hline  & P2
          & 4,0 GHz
          & 2
          & 2
          & 2
          & 4
          & 4 \\
  \hline
\end{tabular}

Assuma que o desempenho \underline{\textbf{pico}} é definido como a taxa mais 
rápida que o computador pode executar qualquer sequência de instruções 
(rajada). Quais são os picos de desempenhos para P1 e P2 individualmente medido 
em instruções por segundos?

$MIPS_{P_{1}} = \frac{Clock\ Rate}{CPI \times 10^{6}} = \frac{2\ \textbf{\tiny 
GHz}}{1 \times 10^{6}} = 2000$ milhões de instruções por segundo.

$MIPS_{P_{2}} = \frac{Clock\ Rate}{CPI \times 10^{6}} = \frac{2\ \textbf{\tiny 
GHz}}{2 \times 10^{6}} = 1000$ milhões de instruções por segundo.

Nessa questão precisamos notar que, o menos CPI é o maior IPC. Se você tem um 
CPI igual à 4, terá um IPC = 1/4 que é menor que se você tivesse um CPI igual à 
2 e então um IPC = 1/2. Ou seja, $1/4 < 1/2$. E IPC é uma média de instruções 
por ciclo. Note que na fórmula do MIPS você está trabalhando com IPC.

$$MIPS = Clock\ Rate \times {\color{red} \frac{1}{CPI \downarrow}} \times 
\frac{1}{10^{6}}$$

\clearpage
\paragraph{1.6.1a} Compiladores diferentes tem impactos em desempenho nos
processadores. Para diferentes compiladores em um mesmo processador, ache o CPI 
se o clock cycle time é de 1 nanosegundo.

\begin{tabular}{|c|c|c|c|c|}
\hline    & Compiler A &            & Compiler B &            \\
\hline a. & No. Instr. & Exec. Time & No. Instr. & Exec. Time \\
\hline    & 1,00E+09   & 1,8 s      & 1,20E+09   & 1,8s       \\
\hline
\end{tabular}

$Tempo_{copmilador_{A}} = 1,8s = \frac{CPI \times No.\ Instr.}{\frac{1}{1 \times
10^{-9}} \textbf{\tiny GHz}} \rightarrow \frac{1,8 \times 10^{9}}{1,0 \times
10^{9}} = 1,8$ ciclos por instrução.

$Tempo_{copmilador_{B}} = 1,8s = \frac{CPI \times No.\ Instr.}{\frac{1}{1 \times
10^{-9}} \textbf{\tiny GHz}} \rightarrow \frac{1,8 \times 10^{9}}{1,2 \times
10^{9}} = 1,5$ ciclos por instrução.

Para um mesmo processador, mesma frequência de relógio, temos que um compilador
gerou mais instruções que o outro. Esse que gerou mais instruções obteve menor
CPI, e portanto melhor desempenho. Tempo vezes frequência de relógio, dá em
ciclos do relógio naquele comprimento de tempo, isso dividido pelo número de
instruções dá o CPI ciclos totais naquela largura de tempo dividido pelo total
de instruções geradas pelo compilador. Quanto maior o número de instruções e
mantendo fixo $tempo \times freq$ menor irá ser o CPI médio. O que é bom, termos
poucos ciclos médios por instrução, isso aumenta o IPC que é o inverso do CPI,
aumentando IPC você tem mais instruções dentro de um ciclo de relógio. Portanto
o compilador que gerar mais instruções nesse ensaio, fará com que o processador
tenha mais desempenho. Podemos pensar que o compilador que gerou menos
instruções, foi mais eficiente. O compilador B obteve 16,6\% mais eficiência do
que o compilador A. É mito dizer que o compilador que gerar menos instruções
será melhor sempre, e hoje em dia com multinucleado multicores e pipeline você
pode ter um compilador gerando mais instruções que outro porém ele sabe muito
bem que as instruções estão organizadas para atingir bom desempenho utilizando
os vários cores disponíveis, ou thread, etc.

\paragraph {
  \cancel{1.7.2},
  \cancel{1.7.3},
  \cancel{1.7.4},
  \cancel{1.8.1}, 
  \cancel{1.8.2}
} cancelei essas questões, na quarta edição revisada, elas me parecem não ter 
relevância.

\paragraph{1.9.3a} Determine a razão da potência-estática e potência-dinâmica 
para cada tecnologia.

Frequência de troca é uma função da frequência de relógio. A carga capacitiva 
por transistor é uma função de ambos o número de transistores conectados a uma 
saída (chamada de fanout) e a tecnologia. As quais ambas determinam a 
capacitância de ambos wires e transistores.

\begin{verbatim}
   Technology Dynamic Power (W) Static Power (W) Voltage (V)
a. 180 nm     50                10               1.2
b.  70 nm     90                60               0.9
\end{verbatim}

Dissipação estática de potência\footnote{A potência consumida pelo dispositivo.
potência estática vai ser consumida sempre, e tem contribuição de vazamento nos
transistores.} $\int_{0}^{1}V_{DD}I_{leak}dt$ Você pode minimizar $I_{leak}$
fazendo diminuição da tensão, ou também colocando menos transistores $I_{leak}$.

Dissipação dinâmica de potência\footnote{Potência consumida quando o dispositivo
está em operação. Esta associado com a frequência de troca dos transistores.}
$\int_{0}^{1}VC^{2}_{DD}f_{c}dt$, podemos diminuir $I_{switch}$ fazendo
diminuição da tensão, menos trocas capacitivas, diminuição da frequência de
trocas.

Ambos somados dá na potência total $E = 
\int_{0}^{t}(V_{DD}I_{leak}+VC^{2}_{DD}f_{c})dt$

$Power = Capacitive\ load \times Voltage^{2} \times Frequency_{switched}$

$Raz = \frac{10}{50} = \frac{1}{5}$

\begin{tabular}{|l|l|l|l|l|l|l|l|}
\hline  & Processadores & & No. Instruções por Processador & & & CPI & \\
\hline  & & Aritméticas &
            LOAD/STORE  &
            Branch      &
            Aritméticas &
            LW/SW       &
            Branch      \\
\hline a.& 1& 2560& 1280& 256& 1&  4& 2\\
\hline   & 2& 1280&  640& 128& 1&  5& 2\\
\hline   & 4&  640&  320&  64& 1&  7& 2\\
\hline   & 8&  320&  160&  32& 1& 12& 2\\
\hline
\end{tabular}

\paragraph{1.10.2a} Dado os valores de CPI, ache o tempo total de execução para 
esses programas em números de processadores: 1,2,4, e 8. Assumir que cada 
processador tem frequência de relógio 2 GHz.

$Tempo = CPI \times Instr. \times Clock\ Rate$

$CPI_{p_{1}} = \sum_{1}^{3} (CPI_{i} \times Instr_{i}) =
1 \times 2560 + 4 \times 1280 + 2 \times 256 = 8192$

$Tempo_{p_{1}} = {8192} \times \frac{1}{2\times10^{9}} =
\frac{8192}{2\times10^{9}} = 4096\ nanosegundos$

$CPI_{p_{2}} = \sum_{1}^{3} (CPI_{i} \times Instr_{i}) =
1 \times 1280 + 5 \times 640 + 2 \times 128 = 4736$

$\frac{CPI}{Nro\ Processadores} = \frac{4736}{2} = 2368$

$Tempo_{p_{2}} = {2368} \times \frac{1}{2\times10^{9}} =
\frac{2368}{2\times10^{9}} = 1184\ nanosegundos$, mas com uma vazão duas vezes 
maior que o primeiro ensaio.

$CPI_{p_{3}} = \sum_{1}^{3} (CPI_{i} \times Instr_{i}) =
1 \times 640 + 7 \times 320 + 2 \times 64 = 3008$

$\frac{CPI}{Nro\ Processadores} = \frac{3008}{4} = 752$

$Tempo_{p_{3}} = {752} \times \frac{1}{2\times10^{9}} =
\frac{752}{2\times10^{9}} = 376\ nanosegundos$, mas com uma vazão duas vezes 
maior que o segundo ensaio.

$CPI_{p_{4}} = \sum_{1}^{3} (CPI_{i} \times Instr_{i}) =
1 \times 320 + 12 \times 320 + 2 \times 32 = 2304$

$\frac{CPI}{Nro\ Processadores} = \frac{2304}{8} = 288$

$Tempo_{p_{4}} = {288} \times \frac{1}{2\times10^{9}} =
\frac{288}{2\times10^{9}} = 36\ nanosegundos$, mas com uma vazão duas 
vezes maior que o terceiro ensaio e quatro vezes maior que o segundo ensaio.

Na melhor das hipóteses os tempos serão esses, mas nunca é garantido que os 
processadores conseguirão dividir o CPI médio. Existe uma dificuldade em 
relacionar instruções e sincronia de threads.

\paragraph{1.10.4a} Assuma frequência de relógio 3 GHz. Qual são os tempos de 
execuções?

\begin{verbatim}
   +---------------------+-----------------------+-------------+
   | Cores per Processor | Instructions per Core | Average CPI |
   +---------------------+-----------------------+-------------+
a. | 1 cores             | 1.00E+10              | 1.2         |
   | 2 cores             | 5.00E+09              | 1.4         |
   | 4 cores             | 2.50E+09              | 1.8         |
   | 8 cores             | 1.25E+09              | 2.6         |
   +---------------------+-----------------------+-------------+
\end{verbatim}

$CPI = 1.2 \times \frac{\textbf{1.00E+10}}{1\ core} \times \frac{1}{3 \times 
10^{9}} = 4,0$

$CPI = 1.4 \times \frac{\textbf{5.00E+09}}{2\ core} \times \frac{1}{3 \times 
10^{9}} = 1,16$

$CPI = 1.8 \times \frac{\textbf{2.50E+09}}{4\ core} \times \frac{1}{3 \times 
10^{9}} = 0,375$

$CPI = 2.6 \times \frac{\textbf{1.25E+09}}{8\ core} \times \frac{1}{3 \times 
10^{9}} = 0,13$

\paragraph{\cancel{1.10.5a}} cancelei essa questão, na quarta edição 
revisada. essa não me parece ter relevância.

\paragraph{1.10.6a} Usando um core único, ache o CPI requerido nesse core para 
obter um tempo de execução igual aos tempos obtidos na questão 1.10.4. Note que 
o número de instruções deve ser o número de instruções agregado ao tempo de 
execução através dos cores.



\end{document}
