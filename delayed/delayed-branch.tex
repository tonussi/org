\documentclass{article}
\usepackage[utf8]{inputenc}
\usepackage[greek, brazil]{babel}
\usepackage[left=1cm, right=1.5cm, top=5cm, bottom=5cm]{geometry}
\usepackage{cancel}
\usepackage{amsfonts}
\usepackage{amssymb}
\usepackage{makeidx}
\usepackage{graphicx}
\usepackage{hyperref}
\usepackage[usenames,dvipsnames]{xcolor}
\renewcommand{\thefootnote}{\alph{footnote}}
\setlength{\parskip}{\baselineskip}
\setlength{\parindent}{0pt}

\usepackage{tikz}
\usetikzlibrary{arrows,automata}

\hypersetup {
  colorlinks,
  citecolor = NavyBlue,
  filecolor = NavyBlue,
  linkcolor = NavyBlue,
  urlcolor = NavyBlue
}
\author{Lucas}
\title{Desvios atrasados e predição dinâmica}
\begin{document}
\maketitle

\begin{enumerate}
\item[pg 380] Assuma a seguinte forma de predição (forma básica): "desvios não
são tomados". Nesse caso predizemos que \textit{branches} (ble, beq, bne, begz,
etc) não são tomados, e quando nós estamos errados nessa hipótese? então
descarregamos o pipeline quando estivermos errados a respeito dessa forma
básica.\footnote{Para o pipeline de 5 estágios predições em tempo de compilação
são suficientes, porém pipelines mais profundos requerem um preditor mais
avançado.} Em um pipeline agressivo um preditor estático vai gastar muita
performance (degradar performance). O custo da penalidade por \textit{branches}
mau preditos, \textit{branches} preditos erroneamente é em instruções perdidas,
no pipeline isso degrada a vazão de instruções.

Uma abordagem é olhar para o endereço da instrução do desvio que já  foi 
processado pelo datapath e verificar se ele \textbf{\underline{foi tomado}}, e 
se foi mesmo, começa a buscar novas instruções do mesmo lugar da última vez. 
Essa técnica é chamada predição dinâmica de desvios.

Uma implementação dessa abordagem é criar uma fila de predições dos desvios que 
já foram processados pelo datapath, também é chamado de tabela histórico dos 
desvios.\footnote{É um jeito simples de ordenara a filha (buffer), a máquina 
não tem certeza se a predição está correta na tabela histórico de desvios.}.

Nessa predição 1-bit (0-não tomado, 1-tomado): mesmo se o desvio é sempre 
tomado, nós podemos predizer incorretamente 2 vezes, ao invés de apenas uma 
vez, quando um desvio não for tomado ("e foi predito que era para ser tomado").

Considere o exemplo de um loop que garantidamente é tomado 9 vezes. Porém o
preditor está com seu histórico vazio. Assim, ele começa predizendo 'não-tomado'
e verifica que foi tomado (pois no histórico estava como foi tomado), o preditor
segue 8x predizendo corretamente, porém na décima predição ele errada, pois
estava no histórico que antigamente aquele esse branch fora tomado, errando
assim duas vezes.

\begin{tabular}{|l|l|l|l|l|l|l|l|l|l|}
\hline t & t & t & t & t & t & t & t & t & n \\
\hline n & t & t & t & t & t & t & t & t & t \\
\hline
\end{tabular}

\begin{enumerate}
\item Primeiro o preditor considera não tomado, mas foi tomado.
\item O preditor dinâmico, então, registra no histórico que foi tomado
\item Aí então ele considera tomado as próximas iterações.
\item E então na última iteração, que era para ser não tomado, o preditor 
considera tomado por causa do histórico. Errou duas vezes, ao invés de uma vez.
\end{enumerate}

\item[pg 381] Idealmente, a acurácia do preditor iria combinar com a frequência 
de desvios, considerando eles regulares, isto é não variam. Para remediar essa 
fraqueza, projetistas de hardware usam comumente: 2-bit prediction scheme. Veja 
que agora a predição precisa errar duas vezes antes de mudar a tabela histórico 
dos desvios.

\begin{figure}[ht!]
\centering
\begin{tikzpicture}[->,>=stealth',shorten >=1pt,auto,node distance=3.8cm,
                    semithick]

  \tikzstyle{every state}=[draw=none,text=white]

  \node[state,fill=green] (A)                    {00};
  \node[state,fill=green] (B) [above right of=A] {01};
  \node[state,fill=red]   (D) [below right of=A] {11};
  \node[state,fill=red]   (C) [below right of=B] {10};


  \path (A) edge [loop left]  node {T } (A)
            edge [bend left]  node {NT} (B)
        (B) edge [bend left]  node {T } (A)
            edge              node {NT} (C)
        (C) edge [loop right] node {NT} (C)
            edge [bend left]  node {T } (D)
        (D) edge [bend left]  node {NT} (C)
            edge              node {T } (A);
\end{tikzpicture}
\caption{Esquemático da predição dinâmica usando 2 bits. Os estados em verde
"gravam" predição de desvio tomado na tabela histórico de desvios, e os estados
em vermelho ''atualizam'' predição de desvio não tomado, na mesma tabela
(exemplo: começando no estado '10'---histórico vazio, e então o desvio foi
tomado ele pula para o estado '11'---atualiza a tabela (NT), na próxima iteração
foi tomado novamente---atualiza a tabela (T), pula para o estado '10' e acerta
todas até o final do loop, predizendo corretamente $\frac{9}{10}$ do tempo.)}
\end{figure}

A implementação, vista a grosso modo, se da pelas alternativas (1): tabela
histórico de desvios é feita em cache dedicada, indexada pelos bits menos
significativos do endereço de desvio, e o acesso acontece no estágio de
\underline{Busca da Instrução} (IF); alternativa (2) a tabela histórico de
desvios é embutido na cache de instruções e exitem um campo para armazenas par
de bits extras para o controle. O esquema de dois bits é uma instancia
generalizada de um contador-preditor, ele incremente quando uma predição é bem
sucedida e decrementa quando uma predição é mau sucedida.\footnote{Implementando
um preditor de 2 bits, quando temos um branch que fortemente é tomado ou é
fortemente não tomado, estatisticamente os \textit{branches} são assim. Ou eles
são muito tomados, ou são muito não tomados, esse preditor de 2-bits se da
melhor em preencher a tabela histórico.}

\item[pg 382] Um branch com atraso vai sempre executar a próxima instrução, mas
a segunda instrução seguinte do branch vai ser afetada pelo branch. Compiladores
e montadores tentam colocar uma instrução que sempre executa após o branch no
campo de atraso do \textit{branch}\footnote{Campo de atraso do \textit{branch}:
é o campo diretamente depois da instrução \underline{\textit{branch} com
atraso}, a qual na arquitetura MIPS é preenchido por uma instrução que não afeta
o branch.}. As limitações em agendar branch atrasado vem de (1) a restrição
sobre a instrução que foi acomodada dentro do campo de atraso e (2) nossa
habilidade de predizer em tempo de compilação quando um branch é provavelmente
tomado ou não. \textit{branch} atrasados funcionam bem em pipelines de 5
estágios, mas quanto mais estágios maior fica o campo de atraso do branch, e
pipelines que emitem múltiplas instruções aumentam ainda mais o campo de atraso
do branch, e acaba que um único campo de atraso, nesses pipelines mais modernos,
é insuficiente. Branch atrasado perdeu a popularidade, comparado com abordagens
caras porém dinamicamente flexíveis\footnote{Paralelamente, o crescimento de
maior números de transistores por chip tem feito possível baratear o custo de
predição dinâmica.}.

\textbf{Elaboração:} Um preditor de desvios nos diz quando um branch é tomado e
quando não é. mas ainda requer cálculos do alvo para onde se deslocar. Em um
pipeline de 5 estágios, esse cálculo \textbf{leva 1 ciclo}, significando que
\textit{branches} tomados irão ter \textbf{1 ciclo} de penalidade associado. Os
3 tipos de \textit{delayed} \textit{branches} são uma maneira de eliminar essa
\underline{penalidade de 1 ciclo}, veja:

\textbf{a.} Preenchimento com uma instrução \underline{independente} e anterior 
ao branch, essa estratégia é a melhor de todas.

\begin{verbatim}
+---------------------+
| add $s1, $s2, $s3   |
| if $s2 = 0 then     |
|   [delay slot]      |
+---------------------+

+---------------------+
| if $s2 = 0 then     |
|   add $s1, $s2, $s3 |
+---------------------+
\end{verbatim}

\textbf{b.} Instrução vindo a partir do endereço alvo do branch, \textbf{b.} é 
usado com \textbf{a.} não é possível de se fazer. Existe uma dependência de 
dados para com \$s1---por isso eu não poderia mover o \verb|add| para dentro do 
\textit{delay} \textit{slot}. Essa estratégia é preferível quando existem 
branch de alta probabilidade de acontecer como em \textit{branch-loop}.

\begin{verbatim}
        +---------------------+
target: | sub $t4, $t5, $t6   |
        |     ...             |
        | add $s1, $s2, $s3   |
        | if $s1 = 0 then     |
        |   [delay slot]      |
        +---------------------+

        +---------------------+
target: |     ...             |
        | add $s1, $s2, $s3   |
        | if $s1 = 0 then     |
        |   sub $t4, $t5, $t6 |
        +---------------------+
\end{verbatim}

\textbf{c.} Vindo de dentro do branch, \textbf{c.} é usado com \textbf{a.} não 
é possível de se fazer. Veja que eu não posso mover \verb|add| para o campo de 
atraso, pois \$s1 é modificado/atualizado logo antes do branch, que por sua vez 
utiliza o \$s1 para verificar se é igual a zero, existe uma dependência de 
dados. Só é possível mover instruções de dentro do branch para o \textit{delay} 
\textit{slot} se obrigatoriamente elas não afetam de nenhum jeito a maneira 
como o branch irá desvios o fluxo de execução.

\begin{verbatim}
+---------------------+
| add $s1, $s2, $s3   |
| if $s1 = 0 then     |
|   [delay slot]      |
|   ...               |
|   sub $t4, $t5, $t6 |
+---------------------+

+---------------------+
| add $s1, $s2, $s3   |
| if $s2 = 0 then     |
|   sub $t4, $t5, $t6 |
+---------------------+
\end{verbatim}

\item[pg 383] Outra abordagem que iremos discutir é a utilização de cache para
salvar o destino do PC ou salvar a instrução destino, usando enfileiramento de
desvios alvo (branch target buffer)\footnote{Estrutura que salva na cache o PC
destino, ou instrução destino para um branch qualquer. É mais custoso que um
simples preditor enfileirador.}. O esquema de predição com 2-bits dinâmico usa
apenas informação sobre um branch em particular. Pesquisadores notaram que
usando informações sobre ambos o \textit{branch} local, e o comportamento global
dos \textit{branches} ajudam a aumentar a acurácia da predição, usando o mesmo
número de bits para no preditor. Esses preditores são chamados preditores
correlatos. Um preditor correlato\footnote{É um preditor de branches que combina
comportamento de desvios locais e globais (recentemente executados)} poderá ter
2-bit para cada branch. E mais bits extra para índice do comportamento global
dos \textit{branches}. Uma descoberta mais recente é o uso de torneio do
preditor, que usa múltiplos preditores, traçando, para cada branch, qual
preditor levanta o melhor resultado. Um torneio\footnote{Um conjunto de
preditores, cada um fazendo predições para cada branch e há uma seleção do
preditor que se sai melhor predizendo branches.} típico pode conter duas
predições para cada índice de branch: m é baseado na informações de
comportamento local e o outro na global. Um seletor poderia escolher qual
preditor usar, para qualquer predição dada. O seletor pode operar similarmente
para 1-bit ou 2-bit preditor, alternando entre esse dois preditores tem mostrado
ser mais acurado. Alguns microprocessadores essa elaboração.

\textbf{Elaboração:} Um caminho para reduzir o número de desvios condicionais é 
adicionar instruções condicionais de \verb|moção|. Ao invés de mudar o PC com 
um desvio condicional, a instrução condicionalmente muda o registrador destino 
do \verb|move|. Se a condição falhar, o move atua como um nop. Por exemplo, uma 
versão do MIPS tem duas novas instruções chamadas \verb|movn| (move if not 
zero) e \verb|movz| (move if zero). Então, \verb|movn $8, $11, $4| copia o 
conteúdo do registrador 11 para o registrador 8, ''sabendo que o valor do 
registrador 4 não é negativo''; do contrário, não acontece nada.

O ARM tem um campo condicional na maioria das suas instruções. Isso foi feito 
sabendo que os programas ARM poderiam ter menos desvios condicionais que 
programas MIPS.

\end{enumerate}

\paragraph{1.0} Mova a instrução para preencher o \textit{delay} \textit{slot}.

\end{document}