\documentclass{article}
\usepackage[utf8]{inputenc}
\usepackage[greek,brazil]{babel}
\usepackage[left=1cm, right=1.5cm, top=5cm, bottom=5cm]{geometry}
\usepackage{cancel}
\usepackage{makeidx}
\usepackage{graphicx}
\usepackage{hyperref}

\usepackage[usenames,dvipsnames]{xcolor}
\renewcommand{\thefootnote}{\alph{footnote}}
\setlength{\parskip}{\baselineskip}
\setlength{\parindent}{0pt}

\hypersetup {
  colorlinks,
  citecolor = NavyBlue,
  filecolor = NavyBlue,
  linkcolor = NavyBlue,
  urlcolor = NavyBlue
}

\author{Lucas Skywalker}

\title{Treino \textit{caches 2-way}}

\usepackage[flushmargin]{footmisc}

\addtolength{\footnotesep}{2mm} % change to 1mm

\begin{document}

As questões a seguir requerem que você traga os blocos para a memória cache: A 
Tabela l mostra de forma simbólica, para alguns endereços de memória
na faixa de 0 até 31C, o conteúdo da memória principal. As colunas em branco
são campos auxiliares para facilitar a correspondência entre endereços
hexadecimais e binários.

Considere uma cache do tipo 2-way, inicialmente vazia, com 128 palavras, sendo
que cada bloco contém 8 palavras. Preencha a Tabela 2 com o conteúdo final da
cache imediatamente após aplicada a sequência de referências acima, usando os
seguintes critérios e convenções: 1-Havendo 2 blocos livres num conjunto, o
bloco trazido da memória deve ser armazenado no bloco livre de menor número (o
preenchimento dos blocos na Tabela 2 deve ser da esquerda para a direita);
2-Havendo 1 bloco livre, nele deve ser armazenado o bloco trazido da memória;
3-Não havendo blocos livres, um dos blocos deve ser substituído de acordo com o
critério LRU (dentre os dois blocos o último preenchido ainda é mais atual que o
segundo, então o segundo será anulado e preenchido com novos dados, ou seja o
segundo foi o menos usado recentemente MUR = LRU); 4-O conteúdo de cada bloco
válido deve ser indicado explicitando todas as suas palavras.


\textbf{Observação:} Por ser uma cache 2-vias, e foi avisado que temos 128
palavras para comportar os dados (na cache inteira). Primeiro você tem que fazer
umas contas básicas antes de começar a preencher a cache. $\frac{128\
palavras}{2-vias \times 8\ conjuntos} = 8$ palavras por via por conjunto, cada
palavra tem 4 bytes, portanto $8 \times 4 = 32\ bytes$ e para indexar 32 bytes 
é preciso ter apenas 5 bits de offset, para indexar 8 conjuntos é preciso ter 3 
bits de índice e o cálculo da tag é igual $tag = 32 - 5 - 3 = 24$ bits.

\begin{tabular}{|c|c|c|}
\hline TAG & ÍNDICE & OFFSET \\
\hline 24 & 3 & 5 \\
\hline
\end{tabular}

Tamanho dos dados: $2 \times 2^{3} \times 2^{5} = 512\ bytes = 128\ palavras$

Tamanho da cache total: $2 \times 2^{3} \times (24 + 0 + (2^{5} * 8)) = 4480\ 
bits = 140\ palavras$, um pouco a mais do que o tamanho de dados da 
cache\footnote{\textbf{\underline{0}} bits de controle.}.

\begin{center}
\Large \textbf{Imprima os treinos e faça a lápix e borracha que é melhor e 
mais prático.}
\end{center}

\clearpage

Treino 1: Referências: 4 \textbf{\tiny hex}, 20 \textbf{\tiny hex}, 10C 
\textbf{\tiny hex}, 318 \textbf{\tiny hex}, 1C \textbf{\tiny hex}.

% A 1010
% B 1011
% C 1100
% D 1101
% E 1110
% F 1111

\begin{table}[ht!]
\centering
\resizebox{\linewidth}{!}{
\begin{tabular}{|c|c|c|c|c|c|}
\hline Endereço (0x) & End. [7:0] (0b) & Conteúdo &
       Endereço (0x) & End. [7:0] (0b) & Conteúdo \\
\hline 0000 0000 &          & X & 0000 0100 &              & P \\
\hline 0000 0004 & 00000100 & X & 0000 0104 &              & L \\
\hline 0000 0008 &          & X & 0000 0108 &              & U \\
\hline 0000 000C &          & Y & 0000 010C & 000100001100 & G \\
\hline 0000 0010 &          & Z & 0000 0110 &              & A \\
\hline 0000 0014 &          & Z & 0000 0114 &              & N \\
\hline 0000 0018 &          & X & 0000 0118 &              & D \\
\hline 0000 001C & 00011100 & C & 0000 011C &              & P \\
\hline 0000 0020 & 00100000 & O & 0000 0300 &              & L \\
\hline 0000 0024 &          & M & 0000 0304 &              & A \\
\hline 0000 0028 &          & P & 0000 0308 &              & Y \\
\hline 0000 002C &          & U & 0000 030C &              & X \\
\hline 0000 0030 &          & T & 0000 0310 &              & Y \\
\hline 0000 0034 &          & A & 0000 0314 &              & Y \\
\hline 0000 0038 &          & R & 0000 0318 & 001100011000 & Y \\
\hline 0000 003C &          & K & 0000 031C &              & Y \\
\hline
\end{tabular}
}
\caption{Conteúdo (parcial) da memória principal.}
\end{table}

\begin{table}[ht!]
\centering
\resizebox{\linewidth}{!}{
\begin{tabular}{|c|c|c|c|c|c|c|c|c||c|c|c|c|c|c|c|c|}
\hline Palavra & 000 & 001 & 010 & 011 & 100 & 101 & 110 & 111
               & 000 & 001 & 010 & 011 & 100 & 101 & 110 & 111 \\
\hline Conjunto 0 & & & & & & & & & &  &  &  &  &  &  &  \\
\hline Conjunto 1 & & & & & & & & & &  &  &  &  &  &  &  \\
\hline Conjunto 2 &  &  &  &  &  &  &  &  &  &  &  &  &  &  &  &  \\
\hline Conjunto 3 & & & & &  &  &  &  &  &  &  &  &  &  &  &  \\
\hline Conjunto 4 &  &  &  &  &  &  &  &  &  &  &  &  &  &  &  &  \\
\hline Conjunto 5 &  &  &  &  &  &  &  &  &  &  &  &  &  &  &  &  \\
\hline Conjunto 6 &  &  &  &  &  &  &  &  &  &  &  &  &  &  &  &  \\
\hline Conjunto 7 &  &  &  &  &  &  &  &  &  &  &  &  &  &  &  &  \\
\hline
\end{tabular}
}
\caption{Status da cache após a sequência de acessos. Dois conjuntos foram 
substituídos.}
\end{table}

\clearpage
Treino 2: Referências: 24 \textbf{\tiny hex}, 104 \textbf{\tiny hex}, 100 
\textbf{\tiny hex}, 11C \textbf{\tiny hex}, 300 \textbf{\tiny hex}.

\begin{table}[ht!]
\centering
\resizebox{\linewidth}{!}{
\begin{tabular}{|c|c|c|c|c|c|}
\hline Endereço (0x) & End. [7:0] (0b) & Conteúdo &
       Endereço (0x) & End. [7:0] (0b) & Conteúdo \\
\hline 0000 0000 &          & J & 0000 0100 & 000100000000 & V \\
\hline 0000 0004 &          & E & 0000 0104 & 000100000100 & A \\
\hline 0000 0008 &          & S & 0000 0108 &              & A \\
\hline 0000 000C &          & U & 0000 010C &              & A \\
\hline 0000 0010 &          & S & 0000 0110 &              & A \\
\hline 0000 0014 &          & C & 0000 0114 &              & M \\
\hline 0000 0018 &          & R & 0000 0118 &              & O \\
\hline 0000 001C &          & I & 0000 011C & 000100011100 & R \\
\hline 0000 0020 &          & S & 0000 0300 & 001100000000 & E \\
\hline 0000 0024 & 00100100 & T & 0000 0304 &              & P \\
\hline 0000 0028 &          & O & 0000 0308 &              & A \\
\hline 0000 002C &          & T & 0000 030C &              & Z \\
\hline 0000 0030 &          & E & 0000 0310 &              & X \\
\hline 0000 0034 &          & S & 0000 0314 &              & Y \\
\hline 0000 0038 &          & A & 0000 0318 &              & Z \\
\hline 0000 003C &          & L & 0000 031C &              & K \\
\hline
\end{tabular}
}
\caption{Conteúdo (parcial) da memória principal.}
\end{table}

\begin{table}[ht!]
\centering
\resizebox{\linewidth}{!}{
\begin{tabular}{|c|c|c|c|c|c|c|c|c||c|c|c|c|c|c|c|c|}
\hline Palavra & 000 & 001 & 010 & 011 & 100 & 101 & 110 & 111
               & 000 & 001 & 010 & 011 & 100 & 101 & 110 & 111 \\
\hline Conjunto 0 & & & & & & & & & &  &  &  &  &  &  &  \\
\hline Conjunto 1 & & & & & & & & & &  &  &  &  &  &  &  \\
\hline Conjunto 2 &  &  &  &  &  &  &  &  &  &  &  &  &  &  &  &  \\
\hline Conjunto 3 & & & & &  &  &  &  &  &  &  &  &  &  &  &  \\
\hline Conjunto 4 &  &  &  &  &  &  &  &  &  &  &  &  &  &  &  &  \\
\hline Conjunto 5 &  &  &  &  &  &  &  &  &  &  &  &  &  &  &  &  \\
\hline Conjunto 6 &  &  &  &  &  &  &  &  &  &  &  &  &  &  &  &  \\
\hline Conjunto 7 &  &  &  &  &  &  &  &  &  &  &  &  &  &  &  &  \\
\hline
\end{tabular}
}
\caption{Status da cache após a sequência de acessos.}
\end{table}

\clearpage
Treino 3: Referências: 310 \textbf{\tiny hex}, 14 \textbf{\tiny hex}, 18 
\textbf{\tiny hex}, 0 \textbf{\tiny hex}, 2C \textbf{\tiny hex}.

\begin{table}[ht!]
\centering
\resizebox{\linewidth}{!}{
\begin{tabular}{|c|c|c|c|c|c|}
\hline Endereço (0x) & End. [7:0] (0b) & Conteúdo &
       Endereço (0x) & End. [7:0] (0b) & Conteúdo \\
\hline 0000 0000 & 00000000 & A & 0000 0100 &              & J \\
\hline 0000 0004 &          & B & 0000 0104 &              & U \\
\hline 0000 0008 &          & C & 0000 0108 &              & S \\
\hline 0000 000C &          & D & 0000 010C &              & T \\
\hline 0000 0010 &          & P & 0000 0110 &              & I \\
\hline 0000 0014 & 00010100 & P & 0000 0114 &              & Ç \\
\hline 0000 0018 & 00011000 & R & 0000 0118 &              & A \\
\hline 0000 001C &          & U & 0000 011C &              & J \\
\hline 0000 0020 &          & D & 0000 0300 &              & E \\
\hline 0000 0024 &          & E & 0000 0304 &              & F \\
\hline 0000 0028 &          & N & 0000 0308 &              & G \\
\hline 0000 002C & 00101100 & C & 0000 030C &              & H \\
\hline 0000 0030 &          & I & 0000 0310 & 001100010000 & T \\
\hline 0000 0034 &          & A & 0000 0314 &              & U \\
\hline 0000 0038 &          & P & 0000 0318 &              & V \\
\hline 0000 003C &          & P & 0000 031C &              & J \\
\hline
\end{tabular}
}
\caption{Conteúdo (parcial) da memória principal.}
\end{table}

\begin{table}[ht!]
\centering
\resizebox{\linewidth}{!}{
\begin{tabular}{|c|c|c|c|c|c|c|c|c||c|c|c|c|c|c|c|c|}
\hline Palavra & 000 & 001 & 010 & 011 & 100 & 101 & 110 & 111
               & 000 & 001 & 010 & 011 & 100 & 101 & 110 & 111 \\
\hline Conjunto 0 & & & & & & & & & &  &  &  &  &  &  &  \\
\hline Conjunto 1 & & & & & & & & & &  &  &  &  &  &  &  \\
\hline Conjunto 2 &  &  &  &  &  &  &  &  &  &  &  &  &  &  &  &  \\
\hline Conjunto 3 & & & & &  &  &  &  &  &  &  &  &  &  &  &  \\
\hline Conjunto 4 &  &  &  &  &  &  &  &  &  &  &  &  &  &  &  &  \\
\hline Conjunto 5 &  &  &  &  &  &  &  &  &  &  &  &  &  &  &  &  \\
\hline Conjunto 6 &  &  &  &  &  &  &  &  &  &  &  &  &  &  &  &  \\
\hline Conjunto 7 &  &  &  &  &  &  &  &  &  &  &  &  &  &  &  &  \\
\hline
\end{tabular}
}
\caption{Status da cache após a sequência de acessos.}
\end{table}

\clearpage
Treino 3: Referências: 3C \textbf{\tiny hex}, 31C \textbf{\tiny hex}, 304 
\textbf{\tiny hex}, 10C \textbf{\tiny hex}, 4 \textbf{\tiny hex}.

\begin{table}[ht!]
\centering
\resizebox{\linewidth}{!}{
\begin{tabular}{|c|c|c|c|c|c|}
\hline Endereço (0x) & End. [7:0] (0b) & Conteúdo &
       Endereço (0x) & End. [7:0] (0b) & Conteúdo \\
\hline 0000 0000 &          & A & 0000 0100 &              & \textgreek{α} \\
\hline 0000 0004 & 00000100 & B & 0000 0104 &              & \textgreek{β} \\
\hline 0000 0008 &          & C & 0000 0108 &              & \textgreek{χ} \\
\hline 0000 000C &          & D & 0000 010C & 000100001100 & \textgreek{δ} \\
\hline 0000 0010 &          & K & 0000 0110 &              & \textgreek{π} \\
\hline 0000 0014 &          & L & 0000 0114 &              & \textgreek{θ} \\
\hline 0000 0018 &          & M & 0000 0118 &              & \textgreek{ρ} \\
\hline 0000 001C &          & N & 0000 011C &              & \textgreek{ω} \\
\hline 0000 0020 &          & W & 0000 0300 &              & E \\
\hline 0000 0024 &          & X & 0000 0304 & 001100000100 & F \\
\hline 0000 0028 &          & Y & 0000 0308 &              & G \\
\hline 0000 002C &          & Z & 0000 030C &              & H \\
\hline 0000 0030 &          & P & 0000 0310 &              & T \\
\hline 0000 0034 &          & Q & 0000 0314 &              & U \\
\hline 0000 0038 &          & R & 0000 0318 &              & V \\
\hline 0000 003C & 00111100 & S & 0000 031C & 001100010011 & J \\
\hline
\end{tabular}
}
\caption{Conteúdo (parcial) da memória principal.}
\end{table}

\begin{table}[ht!]
\centering
\resizebox{\linewidth}{!}{
\begin{tabular}{|c|c|c|c|c|c|c|c|c||c|c|c|c|c|c|c|c|}
\hline Palavra & 000 & 001 & 010 & 011 & 100 & 101 & 110 & 111
               & 000 & 001 & 010 & 011 & 100 & 101 & 110 & 111 \\
\hline Conjunto 0 & & & & & & & & & &  &  &  &  &  &  &  \\
\hline Conjunto 1 & & & & & & & & & &  &  &  &  &  &  &  \\
\hline Conjunto 2 &  &  &  &  &  &  &  &  &  &  &  &  &  &  &  &  \\
\hline Conjunto 3 & & & & &  &  &  &  &  &  &  &  &  &  &  &  \\
\hline Conjunto 4 &  &  &  &  &  &  &  &  &  &  &  &  &  &  &  &  \\
\hline Conjunto 5 &  &  &  &  &  &  &  &  &  &  &  &  &  &  &  &  \\
\hline Conjunto 6 &  &  &  &  &  &  &  &  &  &  &  &  &  &  &  &  \\
\hline Conjunto 7 &  &  &  &  &  &  &  &  &  &  &  &  &  &  &  &  \\
\hline
\end{tabular}
}
\caption{Status da cache após a sequência de acessos.}
\end{table}

\end{document}