\documentclass{article}
\usepackage[utf8]{inputenc}
\usepackage[greek, brazil]{babel}
\usepackage[left=1cm, right=1.5cm, top=5cm, bottom=5cm]{geometry}
\usepackage{amsmath}
\usepackage{amsfonts}
\usepackage{amssymb}
\usepackage{makeidx}
\usepackage{graphicx}
\usepackage{hyperref}
\usepackage[usenames,dvipsnames]{xcolor}
\renewcommand{\thefootnote}{\alph{footnote}}
\setlength{\parskip}{\baselineskip}
\setlength{\parindent}{0pt}
\hypersetup {
  colorlinks,
  citecolor = NavyBlue,
  filecolor = NavyBlue,
  linkcolor = NavyBlue,
  urlcolor = NavyBlue
}
\author{Luke Skywalker}
\title{Caches}
\begin{document}
\maketitle

Considere uma cache com mapeamento \underline{direto} e endereços de memória de
x bits organizados da forma abaixo, responda qual o espaço ocupado pela cache no
processador em bits. Considere que c bits de controle são usado por linha de
cache.

\begin{table}[ht!]
  \begin{tabular}{|c|c|c|}
    \hline TAG & ÍNDICE & OFFSET \\
    \hline t bits & i bits & f bits \\
    \hline
  \end{tabular}
\end{table}

$2^{i}\times(\underline{(x-i-f)}+c+(2^{f} \times 8))\ bits$\\

\small{x: igual ao tamanho de bits do endereço, nesse caso 32 bits.}\\
\small{c: igual  número de bits de controle por linha de cache.}\\

$2^{i}\times(\underline{(x-i-f)}+c+2^{f})\ bytes$\\

Considere um sistema com as seguintes configurações:

\begin{enumerate}
\item Memória virtual de $2^{36}$ bytes
\item Memória física de $2^{22}$ bytes
\item Páginas de $2^{11}$ bytes
\item 2 bits extra para o controle das páginas
\end{enumerate}

Informe o tamanho da tabela de páginas em bits.

\pagebreak
\begin{verbatim}
O campo de offset para memória virtual e para memória física é o mesmo.
virtual memory address  [ tag virtual   |  offset virtual ]
                           |                     |
                 (36-11=25)|                 (11)|
                           |                     |
                        transl.                  |
                           |                     |
                 (22-11=11)|                 (11)|
                           |                     |
physical memory address [ tag phys      |  offset phys    ]
\end{verbatim}

$2^{36-11}\times(2+(22-11))$

$2^{25}\times(2+(22-11))$

Considere uma cache com mapeamento 4-associativo e endereços de memória de 32
bits organizados da forma abaixo, responda quantas linhas tem a cache.

\begin{tabular}{|c|c|c|}
\hline TAG & ÍNDICE & OFFSET \\
\hline 31-26 & 25-10 & 9-0 \\
\hline 6 bits & 16 bits & 10 bits \\
\hline
\end{tabular}

$\text{n-way} \times 2^{\text{índice bits size}}$

$4 \times 2^{16} = 262144$ linhas

Considere um sistema com as seguintes configurações:

\begin{enumerate}
\item $2^{28}$ bytes endereçáveis de memória
\item Cache com $2^{5}$ blocos de $2^{7}$ bytes cada
\item Linhas de cache com 1 bit de validade
\end{enumerate}

Qual seria o tamanho efetivo da cache em bits caso ela fosse implementada com
um mapeamento 4-associativo?

$\text{n-way} \times \frac{2^{5}}{4} \times((28-7-3)+1+2^{7}\times8)$

$4 \times \frac{2^{5}}{4}\times(18+1+2^{7}\times8)$

\pagebreak
Considere um sistema com as seguintes configurações:

\begin{enumerate}
\item Memória virtual de $2^{33}$ bytes
\item Memória física de $2^{24}$ bytes
\item Páginas de $2^{11}$ bytes
\item 6 bits extra para o controle das páginas
\end{enumerate}

Informe o tamanho da tabela de páginas em bits.

$2^{33-11}\times(6+(24-11))$

$2^{22}\times(6+13))$

\end{document}
