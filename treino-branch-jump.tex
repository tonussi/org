\documentclass{article}
\usepackage[utf8]{inputenc}
\usepackage[brazil]{babel}
\usepackage[left=1cm, right=1.5cm, top=1.5cm, bottom=1.5cm]{geometry}
\usepackage{cancel}
\usepackage{makeidx}
\usepackage{graphicx}
\usepackage{hyperref}
\usepackage[usenames,dvipsnames]{xcolor}
\renewcommand{\thefootnote}{\alph{footnote}}
\setlength{\parskip}{\baselineskip}
\setlength{\parindent}{0pt}
\hypersetup {
  colorlinks,
  citecolor = NavyBlue,
  filecolor = NavyBlue,
  linkcolor = NavyBlue,
  urlcolor = NavyBlue
}
\author{Lucas}
\title{Treino de Branchs e Jumps}
\begin{document}
\maketitle

\paragraph{Questão 1:} Você está no endereço efetivo 0x600 e quer deslocar o 
fluxo do código para o endereço LABEL 2 = 0x0, você pode fazer isso usando 
branch? jump? Quais os valores do branch (bne \$s1, \$0, LABEL2), e do jump (j 
LABEL2) no formatos I, J respectivamente a seguir, para poder deslocar de 0x600 
para 0x0.

Formato I

\begin{tabular}{|c|c|c|c|}
  \hline
  opcode & rs & rt & imediato \\
  \begin{tabular}{|c|c|c|c|c|c|}
    \hline  &  &  &  &  & \\
    \hline
  \end{tabular}
  & 
  \begin{tabular}{|c|c|c|c|c|}
    \hline  &  &  &  & \\
    \hline
  \end{tabular}
  & 
  \begin{tabular}{|c|c|c|c|c|}
    \hline  &  &  &  & \\
    \hline
  \end{tabular}
  & 
  \begin{tabular}{|c|c|c|c|c|c|c|c|c|c|c|c|c|c|c|c|}
    \hline  
    &  &  &  &  &  &  &  &  &  &  &  &  &  &  & \\
    \hline
  \end{tabular}
  \\[12px] \hline
\end{tabular}

Formato J

\begin{tabular}{|c|c|}
  \hline
  opcode & imediato \\
  \begin{tabular}{|c|c|c|c|c|c|}
    \hline  &  &  &  &  & \\
    \hline
  \end{tabular}
  & 
  \begin{tabular}{|c|c|c|c|c|c|c|c|c|c|c|c|c|c|c|c|c|c|c|c|c|c|c|c|c|c|}
    \hline  &  &  &  &  &  &  &  &  &  &  &  &  &  &  &  &  &  &  &  &  &  &  & 
    &  & \\
    \hline
  \end{tabular}
  \\[12px] \hline
\end{tabular}

\pagebreak
\paragraph{Questão 2:} Você está no endereço efetivo 0x400400 e quer deslocar o 
fluxo do código para o endereço LABEL 2 = 0x403400, você pode fazer isso usando 
branch? jump? Quais os valores do branch (bne \$t1, \$0, LABEL2), e do jump (j 
LABEL2) no formatos I, J respectivamente a seguir, para poder deslocar de 
0x400400 para 0x403400.

Formato I

\begin{tabular}{|c|c|c|c|}
  \hline
  opcode & rs & rt & imediato \\
  \begin{tabular}{|c|c|c|c|c|c|}
    \hline  &  &  &  &  & \\
    \hline
  \end{tabular}
  & 
  \begin{tabular}{|c|c|c|c|c|}
    \hline  &  &  &  & \\
    \hline
  \end{tabular}
  & 
  \begin{tabular}{|c|c|c|c|c|}
    \hline  &  &  &  & \\
    \hline
  \end{tabular}
  & 
  \begin{tabular}{|c|c|c|c|c|c|c|c|c|c|c|c|c|c|c|c|}
    \hline  
    &  &  &  &  &  &  &  &  &  &  &  &  &  &  & \\
    \hline
  \end{tabular}
  \\[12px] \hline
\end{tabular}

Formato J

\begin{tabular}{|c|c|}
  \hline
  opcode & imediato \\
  \begin{tabular}{|c|c|c|c|c|c|}
    \hline  &  &  &  &  & \\
    \hline
  \end{tabular}
  & 
  \begin{tabular}{|c|c|c|c|c|c|c|c|c|c|c|c|c|c|c|c|c|c|c|c|c|c|c|c|c|c|}
    \hline  &  &  &  &  &  &  &  &  &  &  &  &  &  &  &  &  &  &  &  &  &  &  & 
    &  & \\
    \hline
  \end{tabular}
  \\[12px] \hline
\end{tabular}

\pagebreak
\paragraph{Questão 3:} Você está no endereço efetivo 0x800 e quer deslocar o 
fluxo do código para o endereço LABEL 2 = 0x900, você pode fazer isso usando 
branch? jump? Quais os valores do branch (bne \$s3, \$s1, LABEL2), e do jump (j 
LABEL2) no formatos I, J respectivamente a seguir, para poder deslocar de 
0x800 para 0x900.

Formato I

\begin{tabular}{|c|c|c|c|}
  \hline
  opcode & rs & rt & imediato \\
  \begin{tabular}{|c|c|c|c|c|c|}
    \hline  &  &  &  &  & \\
    \hline
  \end{tabular}
  & 
  \begin{tabular}{|c|c|c|c|c|}
    \hline  &  &  &  & \\
    \hline
  \end{tabular}
  & 
  \begin{tabular}{|c|c|c|c|c|}
    \hline  &  &  &  & \\
    \hline
  \end{tabular}
  & 
  \begin{tabular}{|c|c|c|c|c|c|c|c|c|c|c|c|c|c|c|c|}
    \hline  
    &  &  &  &  &  &  &  &  &  &  &  &  &  &  & \\
    \hline
  \end{tabular}
  \\[12px] \hline
\end{tabular}

Formato J

\begin{tabular}{|c|c|}
  \hline
  opcode & imediato \\
  \begin{tabular}{|c|c|c|c|c|c|}
    \hline  &  &  &  &  & \\
    \hline
  \end{tabular}
  & 
  \begin{tabular}{|c|c|c|c|c|c|c|c|c|c|c|c|c|c|c|c|c|c|c|c|c|c|c|c|c|c|}
    \hline  &  &  &  &  &  &  &  &  &  &  &  &  &  &  &  &  &  &  &  &  &  &  & 
    &  & \\
    \hline
  \end{tabular}
  \\[12px] \hline
\end{tabular}

\end{document}