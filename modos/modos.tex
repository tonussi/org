\documentclass{article}
\usepackage[utf8]{inputenc}
\usepackage[greek, brazil]{babel}
\usepackage[left=1cm, right=1.5cm, top=5cm, bottom=5cm]{geometry}
\usepackage{amsmath}
\usepackage{amsfonts}
\usepackage{amssymb}
\usepackage{makeidx}
\usepackage{graphicx}
\usepackage{hyperref}
\usepackage[usenames,dvipsnames]{xcolor}
\renewcommand{\thefootnote}{\alph{footnote}}
\setlength{\parskip}{\baselineskip}
\setlength{\parindent}{0pt}
\hypersetup {
  colorlinks,
  citecolor = NavyBlue,
  filecolor = NavyBlue,
  linkcolor = NavyBlue,
  urlcolor = NavyBlue
}
\author{Lucas}
\title{Modos}
\begin{document}
\maketitle

% 122-136 pg
\begin{enumerate}

\item Sinalização contra não-sinalização se aplica a instruções 'loads' assim
como para instruções que mexem com aritmética (Patterson pg 124).

\item Caracteres são normalmente combinados em \underline{strings}, essas
mesmas tem um número variado de caracteres. Existem três possibilidades para
representar uma string: (1) a primeira posição da string é reservada para
conter o comprimento da string, (2) outro variável parametrizada contém o
comprimento (como em uma estrutura de lista, pilha, fila, etc), ou (3) a última
posição da string é indicada por um caractere usado para marcar o fim da
string. C usa a terceira opção, terminando a string com um byte do qual o valor 
é \'0' (nomeado null em ASCII). Por exemplo a string "Cal" é representada em C
por 4 bytes => \verb|67, 97, 108, 0|. Em laboratório o java usa a primeira
opção.

\item Os modos de endereçamento da figura página 133 retratam bem, da forma que:

\item O modo de endereçamento dos registradores utiliza os 5 bits para
identificar um dos 32 registradores no caso do mips 32. Mas por exemplo se eu
tivesse um banco de 43 registradores, quantos bits eu precisaria nos campos RS,
RT, RD, respectivamente ? ($log_2 43$)

\item O modo de endereçamento "base+deslocamento" você \underline{soma} o
valor do campo de imediato << 2 e com sinal estendido, ao valor contido no
registrador base. Você pode capturar da com lb, lbu, lh, lhu, lw, ou também
pode salvar na memória usando sw, sh, sb. A forma como você vai somar um
deslocamento em relação à capturar ou salvar na memória usando esse modo de
endereçamento, vai da lógica de programação. E cuidado com essa ideia, pois
deslocar para bytes é diferente de deslocar para palavras inteiras, é óbvio mas
muita gente esquece e eu também .\_.

\item O PC-relativo é o modo dos branchs eles movem o fluxo de execução de
código baseando-se no valor completo do $PC_atual$. E isso dá uma flexibilidade
boa para que seja possível a implementação de if then else (que são usado
extensivamente em alto nível). O salvo do branch é menor que o do jump devido ao
campo de imediato ser menor, 16 bits em relação ao campo de um tipo-j que é de
26 bits. Lembrar que é $PC_{atual}$ + 4 + (SignExt :: BranchAddress :: 00). Pois
o branch mexe no fluxo de execução de instruções do MIPS32 que é alinhado de 4
em 4 bytes, outras arquiteturas como x86 tem um mecanismo de branching mais
complexo, pois o tamanho das instruções varia e isso tem que ser cuidado pelo
sistema. O x86 não perde desempenho por causa disso.

\item Pseudo-Direct Adressing pega o valor direto do imediato como endereço, por
exemplo jump. Porém saiba que o jump depende do último byte do valor do
$PC_{atual}$ que em nem sempre é 0000, por vezes será 0001. Pois o último valor
do espaço alocado no layout de memória para .text (instruções, seus programas,
etc) é $10000000_{hex}$ (basta ver no mips greencard).
\end{enumerate}

\paragraph{Exercise 2.24}

Assume that the register \$t1 contains the address 0x10000000 and the register
\$t2 contains the address 0x10000010. Note the MIPS architecture utilizes big-
endian addressing.

\begin{verbatim}
a. lbu  $t0, 0($t1)
   sw   $t0, 0($t2)

b. lb   $t0, 0($t1)
   sh   $t0, 0($t2)
\end{verbatim}

\paragraph{2.24.1 [5] 2.9} Assume that the data (in hexadecimal) at address
0x10000000 is:

\begin{verbatim}
[10000000][12][34][56][78]
\end{verbatim}

What value is stored at the address pointed to by register \$t2 ? Assume that
the memory location pointed to \$t2 is initialized to 0xFFFFFFFF.

\begin{itemize}
\item Esse "inicializado para 0xffffffff" talvez queira dizer que tem 4GiB de
memória, não tenho certeza. Mas basta você entender que little endian começa
endereçar pelo byte menos significativo e vai em direção ao MSB. E o big endian
começa a endereçar pelo byte mais significativo e vai em direção ao LSB. Agora
a resposta se dá da seguinte forma. Lembre que o lb ele estende sinal, por isso
existe lbu. E também saiba que após o bit oitavo do byte os '1' estendidos não
tem valor significativo, mas representativo, representam um valor negativo.

\item O valor que esta na memória tem o 0x12 como byte mais significativo.
Então é ele o candidato a ser trazido primeiro da memória. E não o byte 0x78.

\item 0x12 = 00010010, então não vai acontecer de estender o sinal negativo.

\item Após 0x12 ter cima carregado no registrador \$t1, a saber, 0x00000012.
Essa mesma palavra 0x00000012 vai ser inteiramente colocada na memória na
posição 0x10000010, deverá aparecer do jeito que esta 0x 00 00 00 12.

\item Sim, por incrível que pareça, quando você coloca byte a byte, a sua
percepção de que é byte endian ou little endian melhora. Mas quando você coloca
uma palavra inteira de uma vez, a sua percepção diminui, pois você não está
vendo os "passos intermediários" de que ele esta colocando byte a byte, e
endereçando automaticamente para você da forma big endian addressing.

\item Se a máquina fosse little endian? Aí sim a palavra sendo colocada
inteiramente apareceria 0x 78 00 00 00. Aqui 0x78, pois no passo anterior ao
'sw' ele carregaria 0x78, pois é little endian.
\end{itemize}



\begin{verbatim}
a. lbu  $t0, 0($t1) #
   sw   $t0, 0($t2) #

b. lb   $t0, 0($t1)
   sh   $t0, 0($t2)
\end{verbatim}

\paragraph{Exercise 2.25}

In this exercise, you will explore 32-bit constants in MIPS. For the following
problems, you will be using the binary data in the table below.

\begin{verbatim}
a. 0010 0000 0000 0001 0100 1001 0010 0100 two
b. 0000 1111 1011 1110 0100 0000 0000 0000 two
\end{verbatim}

\paragraph{2.25.1  [10] 2.10} Write the MIPS assembly code that creates the
32-bit constants listed above and stores that value to register \$t1 .

\begin{verbatim}
a. lui $at, 0x2001
   ori $t1, $at, 0x4924

b. lui $at, 0x0fbe
   ori $t1, $at, 0x4000
\end{verbatim}

\paragraph{2.25.2 [5] 2.6, 2.10} If the current value of the PC is 0x00000000,
can you use a single jump instruction to get to the PC address as shown in the
table above?

\begin{verbatim}
a. 0010 0000 0000 0001 0100 1001 0010 0100 two = 0x20014924

max(pc) -> pc+4[31-28] == 0000 :: 1111 1111 1111 1111 1111 1111 11 :: 00

           pc atual +4 == 0000    0000 0000 0000 0000 0000 0000 01    00

min(pc) -> pc+4[31-28] == 0000 :: 0000 0000 0000 0000 0000 0000 00 :: 00

ffffffc < 20014924, portanto não alcança.

b. 0000 1111 1011 1110 0100 0000 0000 0000 two = 0xFBE4000

max(pc) -> pc+4[31-28] == 0000 :: 1111 1111 1111 1111 1111 1111 11 :: 00

           pc atual +4 == 0000    0000 0000 0000 0000 0000 0000 01    00

min(pc) -> pc+4[31-28] == 0000 :: 0000 0000 0000 0000 0000 0000 00 :: 00

ffffffc > fbe4000, portanto alcança.
\end{verbatim}

\paragraph{2.25.3 [5] 2.6, 2.10} If the current value of the PC is 0x00000600,
can you use a single branch instruction to get to the PC address as shown in
the table above?

\begin{verbatim}
a. 0010 0000 0000 0001 0100 1001 0010 0100 two = 0x20014924

max(pc) -> pc+4[31-28] == 0000 :: 1111 1111 1111 1111 1111 1111 11 :: 00

           pc atual +4 == 0000    0000 0000 0000 0000 0100 0000 01    00

min(pc) -> pc+4[31-28] == 0000 :: 0000 0000 0000 0000 0000 0000 00 :: 00

ffffffc < 20014924, portanto não alcança.

b. 0000 1111 1011 1110 0100 0000 0000 0000 two = 0x0fbe4000

max(pc) -> pc+4[31-28] == 0000 :: 1111 1111 1111 1111 1111 1111 11 :: 00

           pc atual +4 == 0000    0000 0000 0000 0000 0100 0000 01    00

min(pc) -> pc+4[31-28] == 0000 :: 0000 0000 0000 0000 0000 0000 00 :: 00

ffffffc > fbe4000, portanto alcança.
\end{verbatim}

\paragraph{2.25.4  [5] 2.6, 2.10} If the current value of the PC is 0x1FFFf000,
can you use a single branch instruction to get to the PC address as shown in
the table above?

\begin{verbatim}
a. 0010 0000 0000 0001 0100 1001 0010 0100 two = 0x20014924

max(pc) -> pc+4[31-28] == 0001 :: 1111 1111 1111 1111 1111 1111 11 :: 00

           pc atual +4 == 0001    ffff ffff ffff ffff 0000 0000 01    00

min(pc) -> pc+4[31-28] == 0001 :: 0000 0000 0000 0000 0000 0000 00 :: 00

ffffffc < 20014924, portanto não alcança, por poucos bytes, precisaria de mais
um jump para completar.

b. 0000 1111 1011 1110 0100 0000 0000 0000 two = 0x0fbe4000

max(pc) -> pc+4[31-28] == 0000 :: 1111 1111 1111 1111 1111 1111 11 :: 00

           pc atual +4 == 0000    0000 0000 0000 0000 0100 0000 01    00

min(pc) -> pc+4[31-28] == 0000 :: 0000 0000 0000 0000 0000 0000 00 :: 00

ffffffc > fbe4000, portanto alcança.
\end{verbatim}

\paragraph{2.25.5  [10] 2.10} If the immediate field of an MIPS instruction was
only 8 bits wide, write the MIPS code that creates the 32-bit constants listed
above and stores that value to register \$t1.

\begin{verbatim}
a. 0010 0000 0000 0001 0100 1001 0010 0100 two = 0x20014924

lui $t1, 0x20
ori $t1, $t1, 0x01  # 0x20010000

lui $t2, 0x49
ori $t2, $t2, 0x24  # 0x49240000

srl $t2, $t2, 16    # 0x00004924

add $t1, $t1, $t2   # 0x20014924

b. 0000 1111 1011 1110 0100 0000 0000 0000 two = 0x0fbe4000

lui $t1, 0x0f
ori $t1, $t1, 0xbe  # 0x0fbe0000

lui $t2, 0x40
ori $t2, $t2, 0x00  # 0x40000000

srl $t2, $t2, 16    # 0x00004000

add $t1, $t1, $t2   # 0x0fbe4000
\end{verbatim}


\pagebreak

\begin{verbatim}
a. 0x00400000       beq $s0, $0, FAR
   ...
   0x00403100 FAR:  addi $s0, $s0, 1

------------------------------------

b. 0x00000100       j 3FFFC
   ...
   0x04000010 AWAY: addi $s0, $s0, 1
\end{verbatim}

\paragraph{2.27.5 [10] 2.10, questão de prova} By reducing the size of the
immediate fields of the I-type and J-type instructions, we can save on the
number of bits needed to represent these types of instructions. If the immediate
field of I-type instructions were 8 bits and the immediate field of J-type
instructions were 18 bits, rewrite the MIPS code above to reflect this change.
\textbf{Avoid using the lui instruction}.

\begin{verbatim}
I-type [opcode 6 bits][rs 5 bits][rt 5 bits][immed 8 bits]
       [oooooo]       [rrrrr]    [ttttt]    [iiiiiiii]

J-type [opcode 6 bits][immed 18 bits]
       [oooooo]       [iiiiiiiiiiiiiiiiii]
\end{verbatim}

Na página 132 do livro texto (Patterson \& Hennessy, Computer Organization and
Design) temos a solução para branchs que vão muito adiante, lembre que o branch
contém a informação de deslocamento em número de palavras adiante ou
retrocedentes\footnote{Interface Software/Hardware: A maioria dos branchs vai se
deslocar por perto, mas ocasionalmente os branchs irão pular para longe, tão
longe que 16 bits não conseguem mais representar o quanto de palavras deverão
ser puladas. O assembler vem deve vir com uma solução para isso, ele deverá
inserir um branch com a lógica invertida e logo em seguida um jump incondicional
para o local desejado (pois o jump pode ir mais longe e estatisticamente vai
conseguir chegar lá).}. Isso é o suficiente para resolver os exercícios 2.27.5.

\begin{itemize}
\item Suponha que L1 é um endereço muito distante.
  \begin{verbatim}
    bne $t0, $zero, Distante
  \end{verbatim}
\item Então o par de instruções que viabiliza o deslocamento distante é:
  \begin{verbatim}
           beq $t0, $zero, Perto
           j Distante
    Perto:
  \end{verbatim}
\end{itemize}

Para essa questão se você seguir como o compilador elabora a solução:
\begin{itemize}
\item Inverte a lógica do branch
\item Adiciona um jump
\end{itemize}

E também considerando que a questão não quer que usemos \verb|lui| para carregar
endereços grandes para que o jr possa saltar ainda mais longe que o
\textbf{jump}, e também a questão quer que consideremos 8 bits de imediato para
formato-I e 18 bits para formato-J, isso complica a vida.

\begin{verbatim}
a. 0x00400000          beq $s0, $0, FAR

   ...
   0x00403100 FAR:     addi $s0, $s0, 1

   Solução:

   0x00400000        bne $s0, $0, 0x01 # (0x01 << 2) + 0x400004 = 0x400008
   0x00400004        j 0xC40           # 000000001000 :: 0000110001000000 :: 00
   0x00400008 Close:                                            |
                                                                |
   ...                                                          |
   0x00403100 FAR:   addi $s0, $s0, 1              <------------+
\end{verbatim}

A solução que o ARM da para problemas assim é fazer shifts dos valores, 
preenchendo um registrador até possibilitar que eu salte para grandes 
distâncias.

\begin{verbatim}
b. 0x00000100        j CLOSE

   0x00000104 CLOSE: add $at, $0, 0x40  ; t0 recebe 0x00000040
                     sll $at, $at, 0x14 ; 0x00000040 vira 0x4000000
                     ori $at, $at, 0x10 ; t0 recebe 0x4000010

                     jr $at             ; agora é possível pular para
                                        ; longas distâncias com pouco
                                        ; campo de immediato.

   0x04000010  AWAY: addi $s0, $s0, 1
\end{verbatim}

\end{document}
