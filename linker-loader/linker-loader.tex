\documentclass{article}
\usepackage[utf8]{inputenc}
\usepackage[brazil]{babel}
\usepackage[left=1cm, right=1.5cm, top=5cm, bottom=5cm]{geometry}
\usepackage{cancel}
\usepackage{makeidx}
\usepackage{graphicx}
\usepackage{hyperref}
\usepackage[usenames,dvipsnames]{xcolor}
\renewcommand{\thefootnote}{\alph{footnote}}
\setlength{\parskip}{\baselineskip}
\setlength{\parindent}{0pt}

\hypersetup {
  colorlinks,
  citecolor = NavyBlue,
  filecolor = NavyBlue,
  linkcolor = NavyBlue,
  urlcolor = NavyBlue
}

\author{Lucas}

\title{Linker {\small(mesclador de arquivos)}, Loader {\small(carregador de 
arquivos)}}

\begin{document}

\maketitle

\begin{enumerate}

\item[142-147] Passos do Ligador: \begin{enumerate}
\item Coloca códigos e módulos simbolicamente na memória.
\item Determina endereços de dados e labels de instruções.
\item Procura as bibliotecas dos programas para achar as rotinas contidas nelas 
e que são usadas pelo programa em sí, combina ambas referências internas e 
externas. Resolve referências entre os arquivos (B3 Patterson).
\end{enumerate}

Quando todas as referências estão resolvidas, o ligador então vai determinar as 
localizações na memória que cada módulo deverá ocupar.

\item[B18-B19] Passos do Carregador: \begin{enumerate}
\item Lê o cabeçalho do executável, para determinar o tamanho do segmento de 
dados e texto.
\item Cria um novo espaço de endereços para o programa.
\item Copia instruções e dados para um novo espaço de endereços.
\item Copia para a pilha os argumentos passados pera o programa.
\item Inicializa registradores da máquina, em geral a maioria dos registradores
são limpados.
\item Pula para uma rotina start-up que copia os argumentos do programa da pilha
para registradores e chama a rotina principal (\verb|main|). Se ela terminar o
programa termina com o \verb|exit system call|.
\end{enumerate}

\end{enumerate}

\pagebreak
\paragraph{2.31.1a} As tabelas abaixo tem as informações para o ligador, mesclar
os objetos na memória, resolvendo os endereços internos e externos.

\textbf{2.31.1 [5] 2.12} Ligue os objetos $\rightarrow$ Procedure A com .text
size de 0x140 e .data size de 0x40 $\rightarrow$ Procedure B com .text size de
0x300 e .data size de 0x50.\footnote{Como você pode ver nunca haverá a
possibilidade de um jal, j, concatenar alguma coisa além de \textsc{PC+4[31-28]
== 0000}. Senão você estaria pulando para uma área de dados, no layout que
estamos trabalhando aqui. É por isso que dá erro de acesso indevido a memória.}

\resizebox{\linewidth}{!}{
\begin{tabular}{|l|l|l|l|l|l|l|l|}
\hline Text Segment & Address & Instruction & & Text Segment & Address & 
Instruction & \\
\hline  & 400000 & lbu \$a0, 8000(\$gp) & & & 400140 & sw \$a1, 8040(\$gp) & \\
\hline  & 400004 & jal 100050 &  &  & 400144 & jal 100000 & \\
\hline Data Segment & 10000000 & (X) &  & Data Segment & 10000040 & (Y) & \\
\hline  & ... & ... & & & ... & ... & \\
\hline Relocation Info & Address & Instruction Type & Dependency & Relocation
Info & Address & Instruction Type & Dependency \\
\hline  & 0 & lbu (I type) & X &  & 0 & sw (I type) & Y \\
\hline  & 4 & jal (J type) & B &  & 4 & jal (J type) & A \\
\hline Symbol Table & Address & Symbol &  & Symbol Table & Address & Symbol &  
\\ 
\hline  & --- & X &  &  & --- & Y &  \\ 
\hline  & --- & B &  &  & --- & A &  \\ 
\hline 
\end{tabular}
}

\paragraph{2.31.1b} Ligue os objetos $\rightarrow$ Procedure A com .text
size de 0x140 e .data size de 0x40 $\rightarrow$ Procedure B com .text size de
0x300 e .data size de 0x50.

\resizebox{\linewidth}{!}{
\begin{tabular}{|l|l|l|l|l|l|l|l|}
\hline Text Segment & Address & Instruction &  & Text Segment & Address &
Instruction &  \\
\hline  & 400000 & lui \$at, 0x1000 &  &  & 400140 & sw \$a0, 8050(\$gp) &  \\
\hline  & 400004 & ori \$a0, \$at, 0x0000 &  &  & 400144 & jmp 0x60 &  \\
\hline  & ... & ... &  &  & ... & ... &  \\
\hline  & 0x84 & jr \$ra &  &  & 0x180 & jal 0x100000 &  \\
\hline  & ... & ... &  &  & ... & ... &  \\
\hline Data Segment & 10000000 & (X) &  & Data Segment & 10000050 & (Y) &  \\
\hline  & ... & ... &  &  & ... & ... &  \\
\hline Relocation Info & Address & Instruction Type & Dependency & Relocation
Info & Address & Instruction Type & Dependency \\
\hline  & 0 & lui (I type) & X &  & 0 & sw (I type) & Y \\
\hline  & 4 & ori (I type) & X &  & 4 & jump (J type) & F00 \\
\hline  &  &  &  &  & 0x180 & jal (J type) & A \\
\hline Symbol Table & Address & Symbol &  & Symbol Table & Address & Symbol &
\\
\hline  & --- & X &  &  & --- & Y &  \\
\hline  &  &  &  &  & 0x180 & F00 &  \\
\hline  &  &  &  &  & --- & A &  \\
\hline
\end{tabular}
}

\end{document}
