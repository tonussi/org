\documentclass{article}
\usepackage[utf8]{inputenc}
\usepackage[brazil]{babel}
\usepackage[left=1cm, right=1.5cm, top=5cm, bottom=5cm]{geometry}
\usepackage{cancel}
\usepackage{makeidx}
\usepackage{graphicx}
\usepackage{hyperref}
\usepackage[usenames,dvipsnames]{xcolor}
\renewcommand{\thefootnote}{\alph{footnote}}
\setlength{\parskip}{\baselineskip}
\setlength{\parindent}{0pt}
\hypersetup {
  colorlinks,
  citecolor = NavyBlue,
  filecolor = NavyBlue,
  linkcolor = NavyBlue,
  urlcolor = NavyBlue
}
\author{Lucas}
\title{Aritmética e Sincronia de Processadores}
\begin{document}
\maketitle

\section{Aritmética e sincronia de processadores, alguns comentários}

\begin{enumerate}

\item[87-93] Essa seção fala sobre complemento de dois, números com sinal e sem
sinal. Importante para load words, branchs e operações aritméticas com sinal.
Ver por exemplo que o subi para o MIPS não é nativo e se ele trabalhar com um
subi com uma constante mais que 16 bits, maior do que o campo imediato poderá
representar ele vai, inserir instruções para carregar a constante grande em um
registrador e vai depois realizar um sub em cima dos registradores alvos, e
colocar por fim o valor no registrador destino. Na página 94 ele explica que o
complemento de um número $x$ é $2^{n} - x - 1$ genericamente falando. Que
32{0's} é zero, 1::31{0's} é o maior valor negativo em 32 bits que você poderá
ter nessa faixa de representação, assim 0::31{1's} é o maior valor positivo que
você poderá ter. Isso é útil para entender a faixa de deslocamento de instruções
tipo I, branchs, loads, e stores.
\item[109-110] Compiladores MIPS usam \verb|slt, slti, beq, bne| e o valor
fixado de 0 (sempre disponível em \$zero) para criar \underline{todas} as
condições relativas: iguais, não-iguais, menor-ou-igual, maior-que,
maior-ou-igual-que. Por causa do conselho de Neumann's sobre simplicidade do
equipamento, a arquitetura MIPS não inclui '\underline{branch on less then}' por
ser muito complicada, ou iria alongar demais em ciclos de relógio ou iria
aumentar ciclos de relógio por instruções CPI. Duas instruções rápidas é mais
útil. \verb|slt $t0, $s0, $s1| é uma comparação sinalizada. e
\verb|sltu $t1, $s0, $s1| é uma comparação \textbf{\underline{não}} sinalizada.
Tratar números sinalizados como se eles não fossem, nos dá uma técnica baixo 
custo de checar $0<=x<y$ e isso é muito útil por exemplo em Java que testa tudo 
sempre automaticamente quando se fala de Java Arrays.
\begin{verbatim}
Bounds Check Shortcut

                                  # Pula para IndexOutOfBounds se $s1 >= $t2
                                  # ou se $s1 é negativo ($1 < 0).
sltu $t0, $s1, $t2                # $t0 = 0 if $s1 >= length or $s1 < 0
beq  $t0, $zero, IndexOutOfBounds # if bad, goto ERROR
\end{verbatim}
\item[224-227] Essas páginas exploram a noção de subtração de valores usando 
adição de valores em complemento de dois.

  \begin{verbatim}
      Subtraindo 6 decimal de 7 dec pode ser feito diretamente como:
    
    - 0000 0000 0000 0000 0000 0000 0000 0111 = 7 dec
      0000 0000 0000 0000 0000 0000 0000 0110 = 6 dec
    = 0000 0000 0000 0000 0000 0000 0000 0001 = 1 dec
    
      ou via adição usando complemento de 2 representando o -6
    
    + 0000 0000 0000 0000 0000 0000 0000 0111 =  7 dec
      1111 1111 1111 1111 1111 1111 1111 1010 = -6 dec
    = 0000 0000 0000 0000 0000 0000 0000 0001 =  1 dec
  \end{verbatim}

Ou seja, $x-y = x+(-y)$ para a máquina.
Na página 226 tem uma tabela muito interessante:

  \begin{table}[ht!]
    \centering
    \begin{tabular}{|c|c|c|c|}
      \hline Operação & Operando A & Operando B & Resultado com Overflow \\
      \hline $A+B$ & $\ge 0$ & $\ge 0$ & $< 0$ \\
      \hline $A+B$ & $< 0$ & $< 0$ & $\ge 0$ \\
      \hline $A-B$ & $\ge 0$ & $< 0$ & $< 0$ \\
      \hline $A-B$ & $< 0$ & $\ge 0$ & $\ge 0$ \\
      \hline
    \end{tabular}
    \caption{Overflow conditions for addition and subtraction.}
  \end{table}

Na página 227 o autor explora uma ideia de Aritmética para Multimídia. Desde 
que cada microprocessador desktop por definição tem seu próprio display 
gráfico, e o custo dos transistores aumentou então foi inevitável que suporte 
deveria ser adicionado para operações gráficas. Muitos sistemas gráficos 
originalmente usam 8 bits para representar cada uma das 3 cores primárias mais 
8 bits para a localização de um pixel. A adições de auto-falantes e caixa de 
som, para teleconferência e video games sugere suporte sonoro. Gravações de 
Áudio requerem 8 bits de precisão, mas 16 é o suficiente. Cada microprocessador 
tem suporte especial para que bytes e halfwords ocupem menos espaço em memória. 
Mas dado a não frequência de operações aritméticas nesses tamanhos de dados em 
programas tipicamente 32 bits. Existe pouco suporte em transferência de dados.

\item[243] Tabela de instruções MIPS. Da tabela podemos lembrar que:

  \begin{verbatim}
    NOT pode ser implementado assim $s1 = ~ ($s2 | $zero)
  \end{verbatim}

Branchs são relativos ao PC+4 (ver tabela).
Load Linked de fato trás da memória algum valor e armazena em \verb|$alvo|, mas 
ele ativa um flag especial de operação atômica = 1, em seguida se um próximo 
store conditional avaliar que esse flag permanece = 1 ele armazena o que tem em 
seu registrador \verb|$fonte| na posição de memória em que você ligou 
atomicamente, se ele conseguir armazenar lá, quer dizer que ele avaliou que 
aconteceu uma operação atômica corretamente, então ele sinaliza o seu 
registrador com o valor 1 nele. O store conditional usa o mesmo registrador 
fonte como destino depois de colocar o valor do fonte na posição da memória 
desejada.

\item[247] IEEE 754 tem um símbolo para o resultado de operações invalida, 
tal como $0/0$ ou subtraindo infinito de infinito. Esse simbolo é NaN, que é 
Not a Number.

\item[137-139] Aqui nessas páginas ele trata o problema de semáforos em MIPS. 
Como eu expliquei acima, no item 243 essas instruções ll e sc são usadas para 
garantir troca de dados de maneira consistente, isto é, sem que ninguém se 
intrometa no meio, esse ninguém que eu falo são outros processadores fazendo 
data racing. Vamos analisar o código:

\begin{verbatim}
ll $t1, 0($s0)              ; vincula um flag de operação atômica

sc $t0, 0 ($s0)             ; verifica se a operação é atômica
                            ; se sim, então armazena o valor de $t0
                            ; na memória e retorna dentro do registrador
                            ; que é ao mesmo tempo fonte/destino
                            ; esse mesmo registrador é usado para testar caso 
                            ; precise travar até concluir a operação.

lock: la $s0, sem           ; pega o endereço base da label sem
try:  ll $t1, 0($s0)        ; liga atômicamente com o endereço 0($s0)
      bne $t1, $0, try      ; se para esse endereço ainda não estiver disponível
                            ; operações atômicas, então tenta novamente o try:
                            ; até algum outro processador liberar esse 0($s0)

      addi $t1, $0, 1       ; esse addi representa operações complexas
                            ; que são opcionais, depende do código de cada um

      sc $t1, 0($s0)        ; armazena o valor de $t1 (que é um valor qualquer)
                            ; depende da lógica de cada pedaço de programa, etc.
                            ; armazena na mesma posição de memória a qual você
                            ; deu load linked, senão você pode estar fazendo
                            ; operações estranhas para os outros processadores.

      beq $t1, $0, try      ; se ainda sim, quando você tentou dar 'sc' na
                            ; na posição desejada, e ela estava ocupada por
                            ; outro processador, você devera tentar tudo dinovo.

      jr $ra                ; caso contrário, você conseguiu terminar sua op
                            ; atômica. parabéns você não está mexendo com
                            ; atomos. retorna ao procedimento chamador.
\end{verbatim}

Para destravar o semáforo basta você setar 0 na posição de memória dedicada ao 
label '\underline{sem}'.

\begin{verbatim}
unlock: la $t1, sem         ; pega a posição de memória correta de 'sem'
        sw $zero, 0($t1)    ; armazena 32{0}'s lá, zera tudo, sinal verde.
        jr $ra              ; retorna ao procedimento chamador.
\end{verbatim}

\end{enumerate}

\paragraph{não fazer \cancel{2.9.1} $\rightarrow$ fazer 2.10.1 da edição 4 
revisada.}

\begin{verbatim}
a. 0000 0010 0001 0000 1000 0000 0010 0000 two

opcode rs    rt    rd    shamt funct
000000 10000 10000 10000 00000 100000

Então é tipo R pois opcode é 0x0. Agora veja no campo funct que vale 0x20. 
Então a instruções no cartão verde é 0/20 hex.

É um add $rd, $rs, $rt (fiz dessa forma para você ver que é trocado a ordem dos 
registradores em relação ao formato, mas fica alegre isso é o de menos.). A  
instruções do tipo add podem causar overflow.

add $16, $16, $16 ou ainda de forma simbólica add $s0, $s0, $s0 lembrando você 
que $s0 é preservado pela função chamada, $s[0-9] são preservador através da 
chamada eu quero dizer.

b. 0000 0001 0100 1011 0100 1000 0010 0010 two

opcode 12    13    9     shamt funct
000000 01010 01011 01001 00000 100010

Opcode é 0x0 novamente então é do tipo R, agora saibamos qual é o campo funct. 
E ele vale 0x22, basta buscar a instrução do 0/22 hex no mips green card. E é 
um subtract.

sub $9, $12, $13 ou simbolicamente falando podemos colocar sub $t1, $t4, $t5.
\end{verbatim}

\paragraph{não fazer \cancel{2.9.2} $\rightarrow$ fazer 2.10.2 da edição 4 rev.}

Ambas são tipo R. (Patterson \& Hennesy - Computer Organization and Design, 
Edição 4 Revisada)

\paragraph{não fazer \cancel{2.9.4} $\rightarrow$ fazer 2.10.4 da edição 4 rev.}

In the following problems, the data table contains MIPS instructions. You will 
be asked to translate the entries into the bits of the opcode and determine the 
MIPS instruction format.

\pagebreak
\begin{verbatim}
a.   addi $t0, $t0, 0
            \ /
             .                                     ; ele troca os campos de
            / \                                    ; lugar no formato I
           /   \
opcode   rs      rt      immed                     ; uma coisa experta de se
[001000] [01000] [01000] [0000000000000000]        ; fazer é contar os bits
31    26  25  21 20   16 15               0        ; para cada campo e depois
                                                   ; contar para ver se tem 32
                                                   ; bits no formato.

b.   sw $t1, 32($t2)
          \   /
           \ /                                     ; ele troca os campos de
            .                                      ; lugar no formato I
           / \
          |   \
          |    \
opcode   rs      rt      immed
[101011] [01010] [01001] [0000000000100000]
31    26  25  21 20   16 15               0

\end{verbatim}

\paragraph{2.28.2}

\begin{verbatim}
try: mov  r3, r4        ; move o conteúdo de r4 para r3
     ll   r2, 0(r2)     ; salva em r2 o conteúdo de 0(r2) e inic. op. atômica
     addi r2, r2, 1     ; incrementa r2 de 1
     sc   r3, 0(r1)     ; tenta finalizar op. atômica porém em outro endereço
     beqz r3, try       ; se não conseguir realizar op. atômica tenta novamente
                        ; isto é, se r3 retornar com 0 ele falhou na operação
                        ; atômica e deverá tentar novamente, se r3 retornar com 
                        ; o valor 1 então deu certo e o branch não desloca para
                        ; try: "tentar novamente".
     mov  r4, r2        ; move o conteúdo de r2 para r4
\end{verbatim}

O store conditional pode não estár apontando para a mesma posição de memória 
que o load linked. Isto é, pode ser que 0(r2) seja diferente de 0(r1). Mas o 
código irá executar normalmente. Digamos que os endereços são diferentes, então 
nesse caso, \underline{se não tiver} outra thread fazendo load linked na 
posição 0(r1) igual a posição 0(r1) do store conditional aqui, então esse código
vai ficar eternamente tentando pois r3 vai ser sempre 0. Mas se tem outra 
thread que faz load linked a tempo do código acima verificar operação atômica 
então o store conditional aqui vai conseguir retornar 1 em seu registrador e 
vai conseguir terminar a operação atômica, e colocar o valor de r3 em 0(r1). 
Outro caso é se 0(r2) é igual a 0(r1), nesse caso então temos que o código 
executará no mínimo 6 instruções (todo o corpo), e no máximo $5n + 1$ 
instruções.

\paragraph{2.28.3}

\begin{verbatim}
try:  mov  r3, r4        ; move o conteúdo de r4 para r3
      ll   r2, 0(r3)     ; salva em r2 o conteúdo de 0(r2) e inic. op. atômica

      bne  r2, $0, try   ; se não liberar o semáforo fica tentando

      addi r2, r2, 1     ; incrementa r2 de 1
      sc   r2, 0(r3)     ; tenta finalizar op. atômica porém em outro endereço
      beqz r2, try       ; se não conseguir realizar op. atômica tenta novamente
                         ; isto é, se r3 retornar com 0 ele falhou na operação
                         ; atômica e deverá tentar novamente, se r3 retornar 
                         ; com o valor 1 então deu certo e o branch não desloca 
                         ; para try: "tentar novamente".

      mov  r4, r2        ; move o conteúdo de r2 para r4
\end{verbatim}

\pagebreak
\paragraph{2.28.4a [5] 2.11 Fill out the table with the value of the registers 
for each given cycle.}

Each entry in the following table has code and also shows the contents of
various registers. The notation ''(\$s1)'' shows the contents of a memory
location pointed to by register \$s1 . The assembly code in each table is
executed in the cycle shown on parallel processors with a shared memory space.


\begin{table}[ht!]
\centering
\begin{tabular}{|l|l|l|l|l|l|l|l|}
\hline Processador 1 (P1) &
       Processador 2 (P2) &
       Ciclo              &
       P1 - \$t1          &
       P1 - \$t0          &
       (\$s1)             &
       P2 - \$t1          &
       P2 - \$t0          \\

\hline                  &                  & 0 & 1  & 2 & 99 & 30 & 40 \\ 
\hline                  & ll \$t1, 0(\$s1) & 1 & 1  & 2 & 99 & 99 & 40 \\ 
\hline ll \$t1, 0(\$s1) &                  & 2 & 99 & 2 & 99 & 99 & 40 \\ 
\hline                  & sc \$t0, 0(\$s1) & 3 & 99 & 1 & 2  & 99 & 40 \\ 
\hline sc \$t0, 0(\$s1) &                  & 4 & 99 & 1 & 2  & 99 & 0  \\ 
\hline 
\end{tabular} 
\end{table}

\paragraph{2.28.4b} Esse é um pouco mais complicado, mas da para fazer também, 
manda bala.

\begin{table}[ht!]
\centering
\begin{tabular}{|l|l|l|l|l|l|l|l|}

\hline Processador 1 (P1) &
       Processador 2 (P2) &
       Ciclo              &
       P1 - \$t1          &
       P1 - \$t0          &
       (\$s1)             &
       P2 - \$t1          &
       P2 - \$t0          \\

\hline                   &                    & 0 & 1   & 2 & 99  & 30  & 40 \\
\hline  ll \$t1, 0(\$s1) &                    & 1 & 99  & 2 & 99  & 30  & 40 \\
\hline                   & ll \$t1, 0(\$s1)   & 2 & 99  & 2 & 99  & 99  & 40 \\
\hline                   & addi \$t1, \$t1, 1 & 3 & 99  & 2 & 99  & 100 & 40 \\
\hline                   & sc \$t1, 0(\$s1)   & 4 & 99  & 2 & 100 & 1   & 40 \\
\hline sc \$t0, 0(\$s1)  &                    & 5 & 99  & 0 & 100 & 1   & 40 \\
\hline
\end{tabular} 
\end{table}

\end{document}
