\documentclass{article}
\usepackage[utf8]{inputenc}
\usepackage[greek, brazil]{babel}
\usepackage[left=1cm, right=1.5cm, top=5cm, bottom=5cm]{geometry}
\usepackage{cancel}
\usepackage{amsfonts}
\usepackage{amssymb}
\usepackage{makeidx}
\usepackage{graphicx}
\usepackage{hyperref}
\usepackage[usenames,dvipsnames]{xcolor}
\renewcommand{\thefootnote}{\alph{footnote}}
\setlength{\parskip}{\baselineskip}
\setlength{\parindent}{0pt}
\usepackage{tikz}
\usetikzlibrary{arrows,automata}

\hypersetup {
  colorlinks,
  citecolor = NavyBlue,
  filecolor = NavyBlue,
  linkcolor = NavyBlue,
  urlcolor = NavyBlue
}

\author{Lucas}
\title{Pipelining}

\begin{document}

\maketitle

\begin{enumerate}

\item[pg 391] Pipeline explora paralelismo em potencial entre
instruções\footnote{\textit{\textbf{ILP}}.: significa paralelismo entre
instruções.}. Técnica (1) aumentar a profundidade do pipeline para transpor mais
instruções para pegar a velocidade máxima é preciso rebalancear os outros
estágios para acomodar as instruções preenchendo todo o datapath (sempre que
possível) isto é, ter 5 instruções ocupando o datapath (nem sempre isso é
possível), e uma técnica (2) é replicar componentes digitais do computador
(datapath) para processar múltiplas instruções\footnote{\textbf{Emissão
múltipla}.: é um esquema onde múltiplas instruções são lançadas juntas em um
ciclo de clock. } ao mesmo tempo de relógio, por exemplo: um microprocessador de
4 GHz 4-way emissor-múltiplo pode executar uma taxa pico de 16 bi instruções por
segundo e ter o melhor CPI de 0.25 ciclos por instrução, ou o contrário um IPC
igual 4 instruções por ciclo, porém emissão múltipla não é sempre viável para o
microprocessador, e existe a problemática dos hazards de dados e controle que
degradam o CPI.

\begin{table}[ht!]
  \centering
  \begin{tabular}{|l|c|c|c|c|c|c|c|}
    \hline Instruções & Ciclo 1 & Ciclo 2 & Ciclo 3 &
                        Ciclo 4 & Ciclo 5 & Ciclo 6 &
                        Ciclo $\ldots$\\
    \hline I0 & IF & ID & EX & MEM & WB & &\\
    \hline I1 & IF & ID & EX & MEM & WB & &\\
    \hline I2 & & IF & ID & EX & MEM & WB &\\
    \hline I3 & & IF & ID & EX & MEM & WB &\\
    \hline I4 & & & $\ldots$ & $\ldots$ & $\ldots$ & $\ldots$ & $\ldots$\\
    \hline I5 & & & $\ldots$ & $\ldots$ & $\ldots$ & $\ldots$ & $\ldots$\\
    \hline
  \end{tabular}
  \caption{Exemplo de emissão múltipla}
\end{table}

\clearpage
\item[pg 392] Existem dois esquemas para implementar emissão múltipla: (1) 
estática\footnote{\textbf{Emissão múltipla estática}.: É uma abordagem onde 
muitas decisões são feitas pelo compilador, antes da execução das instruções}, 
e (2) dinâmica\footnote{\textbf{Emissão múltipla dinâmica}.: É uma abordagem 
onde muitas decisões são feitas, sobre as instruções, em tempo de execução.}.  
Preocupações principais:

\begin{enumerate}
\item Empacotar instruções em lacunas de emissão: significa que o processador 
tem que decidir quantas instruções ele vai poder empacotar para serem emitidas 
multiplamente, e também, quais instruções ele poderá combinar em cada ciclo 
para emitir elas em um pacote só.
\begin{table}[ht!]
\centering
\begin{tabular}{|l|l|l|l|l|}
\hline Possibilidade 1 & instrução 1 & instrução 2 & instrução 3 & instrução 4\\
\hline Possibilidade 2 & instrução 1 & instrução 2 & instrução 3 &  \\
\hline Possibilidade 3 & instrução 1 & instrução 2 &  &  \\
\hline Possibilidade 4 & instrução 1 &  &  &  \\
\hline
\end{tabular}
\caption{Exemplo de um emissor 4-way, tentando decidir se ele pode emitir 4 
instruções junta, senão 3, senão 2, senão 1 (pelo menos uma instrução ele 
emite) e no melhor caso emite 4 instruções de uma só vez.}
\end{table}

\item Lidando com hazards de controle e dados: em emissão estática as 
consequências geradas por esses tipos de hazards (dependências) é cuidada 
estaticamente pelo compilador. Em contraste emissão dinâmica também cuida das 
consequências geradas por essas classes de hazards (dados e controle) em tempo
de compilação, mas Emi. din. tenta aliviar o impacto de hazards na emissão
múltipla; em tempo de execução, adicionando técnicas de hardware (+custo).
\end{enumerate}

\item[pg 393] Uma abordagem clássica para aumentar \textit{ILP} é a
especulação\footnote{\textbf{Speculation}.: é uma abordagem onde o compilador ou
processador tenta adivinhar o que será de uma instrução para remover ela como
uma dependência, quando executando outras instruções.} que é uma abordagem que
permite o computador ''adivinhar'' sobre as propriedades de uma instrução, para
habilitar a possibilidade de outras instruções serem executadas, ou seja, o
processador começa buscar instruções "especulando" que o resultado delas vai ser
$x$ e continua puxando instruções, se o resultado não for $x$, mas $y$, então
acontece um descarregamento das instruções, na verdade ele simplesmente não
grava o resultado no banco de registradores, ou na memória assim as instruções
que foram anuladas por causa que deu $y$.

\begin{enumerate}
\item[Exemplo 1] Poder-se-ai especular que um branch (à diante) vai dar certo,
então o processador pode ir executando as instruções no bloco básico
pertencente ao escopo do branch e quando chegar no branch as instruções já
foram processadas.

\item[Exemplo 2] Poder-se-ai especular que um store precedente a um load, e
também, ambos \underline{não} referenciarão o mesmo local da memória, o que
permitiria ao processador executar o load anteriormente ao store.
\end{enumerate}

A problemática da especulação é que ela pode errar, e isso joga fora ciclos de
processamento. Precisa de mecanismo de hardware para, em tempo de execução,
especular sobre os resultados das instruções futuras e também um mecanismo para
poder voltar atrás e não executar as instruções, como eu disse é um componente
digital que verifica se o acerto deu correto ao não, isto é, se o resultado
atual é igual ao resultado ''chutado'' se for igual, prossegue, se for diferente
o datapath tem que acionar o enfileirador de instruções para eliminar
instruções\footnote{Essas instruções que o especulador buscou ''acreditando''
que elas iriam ser executadas se o ''chute'' tivesse dado certo}. Especulação
pode ser feita tanto pelo compilador, quanto pelo hardware. O compilador pode
usar especulação para reordenar instruções, movendo uma instrução através do
branch ou um load através do store.

Especulação trás outro problema: Especular certos tipos de instruções podem 
introduzir exceções! Imagine uma instrução LOAD é movida de maneira 
especulativa, mas o ENDEREÇO que ela usou não é LEGAL se a especulação estiver 
incorreta! Uma exceção que não deveria acontecer, mas acontecerá.

\item[pg 394] EMISSÃO ESTÁTICA, basicamente essa técnica ''mistura'' instruções
para serem emitidas em um mesmo ciclo, isso aumenta performance, mas requer que
o datapath suporte esses caminhos múltiplos para acomodar as instruções, o mix
de instruções (pacote de instrução gerado pelo emissor múltiplo estático)
precisa de componente para administrar hazards de controle e dados, esse mix de
instruções em um mesmo ciclo é na verdade uma grande instrução com várias
operações, a saber, independentes \footnote{\textbf{Very Long Instruction
Word}.: \textit{VLIW}, que é muito parecida, porém processadores VLIW emitem uma
grande instrução com várias operações que foram decididas como sendo
independentes e tipicamente essa instrução vem com vários códigos
operacionais.}. Se no mix de instruções, acontecer de apenas uma instrução poder
ser emitida em um ciclo, é recomendado que as outras lacunas do pacote emitido
sejam preenchidas com \verb|nop|.

Em emissão estática, varia como é feita o tratamento de hazards de dados e
controle. Em alguns modelos, o compilador leva total responsabilidade em remover
todos os hazards, escalonando o código e inserindo no-ops para que o código
execute sem a necessidade de detecção de hazards, ou suspensões
hardware-geradas. Em outras palavras o hardware detecta hazards e gera as
suspensões entre duas emissões múltiplas, enquanto requerer que o compilador se
livre de todas as dependências entre pares de instruções. Mesmo assim, um hazard
geralmente força um pacote inteiro contendo a instrução dependente a suspender
seu processamento no datapath (''entrar em \textit{stall}''). Onde o software
precisa tratar todos os hazards ou apenas tentar reduzir uma fração de hazards
entre pacotes de emissão, a aparência de ter uma ''única instrução larga'' com
múltiplas operações é reforçada. Nós vamos assumir a segunda abordagem para esse
exemplo.

\clearpage
\item[pg 395] Para emitir uma operação com ALU e uma transferência de dado em
paralelo (''processador emissor estático duas vias''), o que precisamos fazer é
adicionar portas extras no banco de registradores (além de que precisamos do
detector de hazard e componente para a logica de suspensões). Isso aumenta
performance 2x mais. Porém requer que duas vezes mais instruções sejam
transposta em execução. Por exemplo, loads têm 'latência de uso' igual a 1 ciclo
de relógio, o que antecipa uma instrução em usar o resultado sem ser suspendida.
Em um emissor duplo, o resultado de uma instrução load não pode ser usado no
próximo ciclo. Isso significa que as próximas duas instruções não podem usar o
resultado do load sem suspender.

\item[pg 396] No código escalonado abaixo, foi feito \underline{reordenamento}
(pelo compilador) para escapar do máximo possível de suspensões no pipeline
(''\textit{stalls}''), possível. Estamos assumindo que branchs são preditos,
assim hazards de controle são tratados pelo hardware.

\begin{table}[ht!]
\centering
\begin{tabular}{cllc}
\textbf{LABEL}&\textbf{ALU/BRANCH} & \textbf{LOAD/STORE}&\textbf{CICLO}\\
LOOP: & addi \$s1, \$s1, -16 & lw \$t0, 0(\$s1) & 1\\
& & lw \$t1, 12(\$s1) & 2 \\
& add \$t0, \$t0, \$s2 & lw \$t2, 8(\$s1) & 3 \\
& add \$t1, \$t1, \$s2 & lw \$t3, 4(\$s1) & 4 \\
& add \$t2, \$t2, \$s2 & sw \$t0, 16(\$s1) & 5 \\
& add \$t3, \$t3, \$s2 & sw \$t1, 12(\$s1) & 6 \\
& & sw \$t2, 8(\$s1) & 7 \\
& bne \$s1, \$zero, LOOP & sw \$t3, 4(\$s1) & 8 \\
\end{tabular}
\caption{Exemplo de emissão múltipla, considerando latência de uso do LOAD, 
esse código foi feito desenrolamento de laço 4 vezes. Durante o desenrolar, o 
compilador introduz registradores adicionais, criando espaço para que não haja 
dependência entre instruções, permitindo acomodar elas nos ciclos, e criando 
uma anti-dependência entre elas. \textbf{Anti-dependência}, ou dependência de 
nome, é o ordenamento forçado pelo reuso de nomes, tipicamente um registrador, 
mais do que a dependência de dados mesmo. Tratar desse problema é reordenando, 
renomeando registradores, porém as vezes o compilador não consegue ir muito 
longe para cuidar disso.}
\end{table}

Ou seja, a latência de uso de uma LOAD gera um 1 ciclo extra, e é apenas quando 
existe a dependência de dados gerado no LOAD e consumido em uma próxima 
instrução, nesse caso o hardware deve acomodar as instruções para que a 
instrução que consumir o dado gerado no LOAD fique a 1 ciclo de distância do 
LOAD, para se desviar do hazard de dados (em inglês o termo é ''use latency'').

\begin{table}[ht!]
  \centering
  \begin{tabular}{cllc}
    \textbf{LABEL} & \textbf{ALU/BRANCH} & \textbf{LOAD/STORE} & 
    \textbf{CICLO} \\
    LOOP: &  & lw \$t0, 0(\$s1) & 1 \\
    & addi \$s1, \$s1, -4 &  & 2 \\
    & addu \$t0, \$t0, \$s2 &  & 3 \\
    & bne \$s1, \$zero, LOOP & sw \$t0, 4(\$s1) & 4 \\
  \end{tabular}
  \caption{O mesmo código (sem desenrolamento quádruplo) sendo processado por um
  emissor múltiplo duas vias. Nesse ensaio temos que o ''use latency'' do
  primeiro \textbf{load} está em ação, é necessário um ciclo de distância para
  evitar o hazard de dados.}
\end{table}

\item[pg 397] Desenrolamento de laços é uma técnica importante, onde múltiplas
cópias do \underline{laço repetitivo} são feitas. Essa é claramente uma técnica
para aumentar a taxa de instruções por ciclo, em um emissor múltiplo essa
técnica é bem aproveitada. Busca aumentar a vazão de instruções pelo datapath.
Essa técnica exige implementação extra---renomeador de
registradores\footnote{busca renomear quantos necessário, visto que no
desenrolar irão aparecer mais instruções executando a mesma tarefa de maneira
adiantada, renomear registradores remove anti-dependências.}.

\item[pg 398] Processadores com emissores dinâmicos também são conhecidos como 
super escalares (\textit{superscalar}). No mais simples, as instruções são 
emitidas em ordem\footnote{No emissor estático o compilador pode porventura 
reordenar as instruções, para eliminar hazards.} E o processador decide quando 
zero, uma, ou mais instruções poderão ser emitidas em um pacote único em um 
mesmo ciclo de relógio.

Existe uma grande diferença entre um simples super escalar e um VLIW: o código, 
quando escalonado (\textit{scheduled}) ou não, é garantido pelo hardware que 
irá executar corretamente. Além do mais, código compilado vai sempre rodar 
corretamente independente da taxa de emissão, ou estrutura do pipeline do 
processador. Em alguns VLIW, isso não tem sido o caso, por vezes quando código 
compilado para uma máquina emissora estática era rodado em outra máquina tinha 
perda de performance e tinha de ser recompilado.

\item[pg 399] Uma técnica boa é escalonamento em pipeline dinâmico, onde 
escolhe quais instruções irão executar em um dado ciclo de relógio, enquanto 
tentando desviar de hazards e suspensões (\textit{stalls}).

\begin{verbatim}
lw   $t0, 20($s2)
addu $t1, $t0, $t2
sub  $s4, $s4, $t3
slti $t5, $s4, 20
\end{verbatim}

Mesmo que o SUB já esteja apto a ser processado, ele terá que esperar pelo seus 
companheiros LW e ADDU, o que ira levar muitos ciclos de relógio se a memória 
for lenta \footnote{Faltas na caches torna o acesso a memória custoso em termos 
de ciclos de relógio (tempo)}. Escalonamento em pipeline dinâmico permite tais 
hazards serem ''anulados'' completamente ou parcialmente.

\textbf{Escalonamento com pipeline dinâmico}: é um modelo que escolhe quais 
serão as próximas instruções a serem executadas, possivelmente reordenando 
elas para evitar \textit{stalls}. Em tais processadores que usam esse modelo, o 
pipeline é dividido em 3 unidades majoritárias: um \underline{buscador} de 
pacotes de instrução (\textit{mix}, \textit{issue packet}), múltiplas unidades 
funcionais\footnote{Cada unidade funcional tem buffers, chamado de estações 
reservatório, o que segura os ''operandos'' e ''operadores'', em 2008 
eram 12 unidades para se ter noção}, e uma unidade de \textit{commit}, que 
contém um buffer (\textit{enfileirador reordenador}), é usado para dar suporte 
aos operandos (registradores) da mesma maneira que a lógica de forwarding em 
escalonamento com pipeline estático. Quando um resultado é cometido para o 
banco de registradores, o resultado pode ser buscado diretamente de lá, assim 
como em um pipeline normal.

\item[pg 400] A implementação que cuida dos reservatórios e das unidades 
funcionais deve dentre outras coisas, enfileirar instruções que chegam para as 
unidades funcionais, as instruções terão prioridade de execução intrínseca 
muitas vezes, mas nada impede do modelo de escalonamento dinâmico reordenar 
instruções dentro do processo, trabalhando em cima das unidades funcionais, e 
quando tiver que executar algumas instruções com prioridade ele executa as 
prioridades, esperando que algumas unidades funcionais liberem seus resultados 
para que outras unidades funcionais que dependem desses resultados, possam 
processar as suas instruções, no final a unidade de entrega, vai reordenar as 
instruções para que siga a lógica que o programador fez, o produto final é o 
mesmo. Mas o modelo dinâmico possivelmente trocou a ordem de instruções para 
obter maiores taxas de emissões múltiplas (\textit{out-of-order execution}).

Escalonamento dinâmico também permite uma opção chamada de \textit{in-order}, 
ondem a ordem que o compilador gerou, vai ser a mesma ordem que entra e saí do 
pipeline com escalonador dinâmico, hoje em dia é o que mais se usa. 
Escalonadores dinâmicos em-ordem usam também especulação para BRANCHES e 
endereços de LOAD/STORES, visto que a execução é em ordem, após a unidade de 
entrega se pode fazer certas especulações sobre o resultados dessas classes de 
instruções.

\item[pg 401] Mas por que usar escalonamento dinâmico? \textbf{(1)} Stalls são
um vilão para os processadores, por causa de faltas na caches, stalls podem ser
imprevistos, e o pipeline dinâmico pode utilizar os ciclos de stall para
substituí-los inserindo outras instruções para serem processadas, não degradando
\textit{instruction} \textit{level} \textit{parallelism}, \textbf{(2)}
incorporar especulação dinâmica funciona melhor com um hardware que escalona
dinamicamente as instruções, pois o compilador não consegue especular BRANCHES,
LOADS e STORES em tempo de compilação, \textbf{(3)} escalonamento dinâmico
comporta melhor compilações de máquinas diferentes, e código legado também
recebe um benefício rodando em máquinas que usam escalonamento
dinâmico\footnote{Pipelining e emissão múltipla aumentam a vazão de instruções e
ILP. Todavia, hazards de dados e controle colocam uma cota superior para a 
sustentabilidade da performance, por causa que processadores precisam esperar 
que a dependência seja resolvida, modelagens centradas em software exigem que o 
compilador seja eficaz em remover ao máximo as dependências de dados e controle 
para que o emissor múltiplo com pipeline possa ser eficaz em obter grandes 
rajadas de instruções. E modelagens centradas em hardware encontram saídas em 
extensões do pipeline e dispositivos de emissão. Especulação, realizada pelo 
compilador ou pelo hardware, pode aumentar o montante de ILP que pode ser 
explorado, no entanto atente que especulação incorreta reduz performance.}

\item[pg 402] Power Efficiency and Advanced Pipelining The downside to the
increasing exploitation of instruction-level parallelism via dynamic multiple
issue and speculation is power efficiency.

\begin{table}[ht!]
\centering
\resizebox{\textwidth}{!}{
\begin{tabular}{llllllll}
\hline
Microprocessor & Year & Clock Rate & Pipeline Stages & Issue Width & 
Out-of-Order/ & Speculation Cores/Chip & Power \\
\hline
Intel 486 & 1989 & 25 MHz & 5 & 1 & No & 1 & 5 W \\
Intel Pentium & 1993 & 66 MHz & 5 & 2 & No & 1 & 10 W \\
Intel Pentium Pro & 1997 & 200 MHz & 10 & 3 & Yes & 1 & 29 W \\
Intel Pentium 4 Willamette & 2001 & 2000 MHz & 22 & 3 & Yes & 1 & 75 W \\
Intel Pentium 4 Prescott & 2004 & 3600 MHz & 31 & 3 & Yes & 1 & 103 W \\
Intel Core & 2006 & 2930 MHz & 14 & 4 & Yes & 2 & 75 W \\
UltraSPARC IV+ & 2005 & 2100 MHz & 14 & 4 & No & 1 & 90 W \\
Sun UltraSPARC T1 (Niagara) & 2005 & 1200 MHz & 6 & 1 & No & 8 & 70 W \\
\end{tabular}}
\caption{Tabela de processadores Intel e Microprocessadores Sun em termos de
complexidade de pipeline, número de cores, e custo energético. Os estágios do
pipeline Pentium 4 não incluem unidade de \textit{commit}. Se incluirmos, o
então pipeline do Pentium 4 será mais profundo.}
\end{table}

Acessos à memória se beneficiam de caches que não bloqueiam. Execução
fora-de-ordem precisa de modelos de cache que permitem instruções serem
executadas durante um miss.

\item[pg 404] Opteron X4 usa um esquema para resolver anti-dependências e 
especulação incorreta que usa reordenamento do buffer junto com renomeação de 
registradores. Renomeia registradores arquiteturais\footnote{O conjunto de 
instruções de registradores visíveis do processador; por exemplo, em MIPS, eles 
são os 32 registradores para inteiros e os 16 para ponto flutuantes.}


\end{enumerate}

\paragraph{4.29} Nesse execício, nós consideramos a execução do loop em 
escalonamento estático super-escalar. Para simplificar o exercício, assuma que 
qualquer combinação de tipos de instruções é possível no mesmo ciclo de 
relógio, e.g., em um 3-\textit{issue} super-escalar, as 3 instruções pode ser 3 
operações ALU, 3 branches, 3 load/store, ou combinações dessas mesmas. Notar 
que isso apenas remove hazards estruturais, mas permanece a possibilidade de 
hazards de controle e dados e esses dois últimos precisam ser tratados 
corretamente.

\paragraph{4.29.4b} Desenrole o laço do código abaixo uma vez apenas e escalone
ele para um emissor duplo estático super-escalar. Assumir que o laço sempre
executa um número \underline{par} de iterações. Você pode usar registradores R10
até R20 quando mudar o código para eliminar possíveis dependências.

\textbf{antes do desenrolar}.

\textbf{a.}

\begin{verbatim}
Loop: ADDI R1,R1,4
      LW R2,0(R1)
      LW R3,16(R1)
      ADD R2,R2,R1
      ADD R2,R2,R3
      BEQ R2,zero,Loop
\end{verbatim}

\begin{tabular}{|c|l|l|}
\hline Loop: & ADDI R1,R1,4 & \textbf{nop} \\ 
\hline & \textbf{nop} & LW R2,0(R1) \\
\hline & \textbf{nop} & LW R3,16(R1) \\
\hline & ADD R2,R2,R1 & \textbf{nop} \\ 
\hline & ADD R2,R2,R3 & \textbf{nop} \\
\hline & BEQ R2,zero,Loop & \textbf{nop} \\
\hline
\end{tabular} 

\textbf{pós desenrolar}.

\begin{verbatim}
Loop: ADDI R1, R1, 4

      LW R2, -4(R1)
      LW R3, 12(R1)

      LW R10, 0(R1)
      LW R11, 16(R1)

      ADD R2, R2, R1

      ADD R2, R2, R3
      ADD R2, R10, R11

      BEQ R2, zero, Loop
\end{verbatim}

\begin{tabular}{|c|l|l|}
\hline Loop: & ADDI R1,R1,4 & \textbf{nop} \\
\hline & \textbf{nop} & LW R2, -4(R1) \\
\hline & \textbf{nop} & LW R3, 12(R1) \\
\hline & ADD R2, R2, R1 & LW R10, 0(R1) \\
\hline & ADD R2, R2, R3 & LW R11, 16(R1) \\
\hline & ADD R2, R10, R11 & \textbf{nop} \\
\hline & BEQ R2, zero, Loop & \textbf{nop} \\
\hline
\end{tabular} 

\clearpage
\textbf{antes do desenrolar}.
\textbf{b.}

\begin{verbatim}
Loop: LW R1,0(R1)         F D E M W
                          X X X X X
      AND R1,R1,R2          X F D E M W
                            X X X X X X
      LW R2,0(R2)               F D E M W
                                X X X X X
      BEQ R1,zero,Loop            F D E M W
                                  X X X X X
\end{verbatim}

Podemos mover LW R@,0(R2) para segunda linha.

\begin{tabular}{|c|l|l|}
\hline Loop: & LW R1,0(R1) & \textbf{nop} \\
\hline & AND R1,R1,R2 & \textbf{nop} \\
\hline & LW R2,0(R2) & \textbf{nop} \\
\hline & BEQ R1,zero,Loop & \textbf{nop} \\
\hline
\end{tabular}

\clearpage
\textbf{pós desenrolar}.

\begin{verbatim}
Loop: LW R1,0(R1)        F D E M W
                         X X X X X
      LW R3,4(R1)          F D E M W
                           X X X X X
      AND R1,R1,R2           F D E M W
                             X X X X X
      AND R3,R3,R2             F D E M W
                               X X X X X
      AND R1,R1,R3               F D E M W
      LW R2,0(R2)                F D E M W
      LW R2,4(R2)                  F D E M W
      BEQ R1,zero,Loop             F D E M W
\end{verbatim}

\begin{quotation}
''A static two-issue datapath. The additions needed for double issue are
highlighted: another 32 bits from instruction memory, two more read ports and
one more write port on the register file, and another ALU. Assume the bottom ALU
handles address calculations for data transfers and the top ALU handles
everything else.''

\flushright -----page 395, Patterson \& Hennessy, Computer Org. and Design
\end{quotation}

No entanto, subir uma instrução AND R3,R3,R2 para combinar em um issue-packet 
causa dependência de dados com LW R3,4(R1) que exige \textit{use latency} de 1 
ciclo.

\begin{tabular}{|c|l|l|l|}
\hline LABEL & EMISSÃO 1 & EMISSÃO 2 & Ciclo \\
\hline Loop: & \textbf{nop} &  LW R1,0(R1) & 1 \\
\hline & \textbf{nop} & LW R3,4(R1) & 2 \\
\hline & AND R1,R1,R2 & \textbf{nop} & 3 \\
\hline & AND R3,R3,R2 & \textbf{nop} & 4 \\
\hline & AND R1,R1,R3 & LW R2,0(R2) & 5 \\
\hline & \textbf{nop} & LW R2,4(R2) & 6 \\
\hline & \textbf{nop} & BEQ R1,zero,Loop & 7 \\
\hline
\end{tabular}

Número de instruções $7*2=14$ instruções contando os nops junto  e $14*4=56$ 
bytes ao total.


\paragraph{4.29.5}

\paragraph{4.29.6}

\end{document}