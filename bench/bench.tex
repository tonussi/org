\documentclass{article}
\usepackage[utf8]{inputenc}
\usepackage[brazil]{babel}
\usepackage[left=1cm, right=1.5cm, top=5cm, bottom=5cm]{geometry}
\usepackage{cancel}
\usepackage{makeidx}
\usepackage{graphicx}
\usepackage{hyperref}
\usepackage[usenames,dvipsnames]{xcolor}
\renewcommand{\thefootnote}{\alph{footnote}}
\setlength{\parskip}{\baselineskip}
\setlength{\parindent}{0pt}

\hypersetup {
  colorlinks,
  citecolor = NavyBlue,
  filecolor = NavyBlue,
  linkcolor = NavyBlue,
  urlcolor = NavyBlue
}

\author{Lucas Skywalker}

\title{Bench\textit{marks}}

\begin{document}

\maketitle

\begin{enumerate}

\item[pg. 35] $$CPU_{time} = Instruction\ count \times CPI \times Clock\ cycle\ 
time$$

$$CPU_{time} = \frac{Instruction\ count \times CPI}{Clock\ cycle\ time}$$

\item[pg. 48] Wordload é um conjunto de programas rodando em um computador que 
pode 
ser tanto uma coleção de aplicações rodadas pelos usuário ou construída de 
programas reais para aproximar o \verb|mix|. Um wordload típico especifica 
ambos os programas e as frequências relativas.\footnote{SPEC significa System 
Performance Evaluation Cooperative, Sistema Corporativo de Avaliação do 
Desempenho.} A elaboração é que quando avaliar dois computadores usando 
SPECratios, usar média geométrica para dar uma resposta normalizada sempre.

\item[pg. 49] Tempo de execução é o produto de três fatores na tabela; Contagem
de Instruções (em bilhões), Relógios (Clocks) por Instrução (CPI), e Tempo de
Ciclo de Relógio (em nanosegundos). SPECratio é simplesmente o time de
referência, o qual é suportado pelo SPEC, dividido pelo tempo de execução
medido. ($SPECratio = \frac{Reference\ Time\ (Seconds)}{Execution\ Time\
(Seconds)}$). Exemplo: $SPECratio(perl) = \frac{9770}{637} = 15,3$

\resizebox{\linewidth}{!}{
\begin{tabular}{|r|r|r|r|r|r|r|r|}
\hline Descrição & Nome & Contagem de Instruções em Bilhões & CPI & Tempo de 
Ciclo de Relógio $(Segundos \times 10^{-9}$) & Tempo de Execução (Segundos) & 
Tempo Referencial (Segundos) & SPECratio \\
\hline Processador interpretador de strings & perl & 2118 & 0,75 & 0,4 & 637 & 
9770 & 15,3 \\
\hline
\end{tabular}
}


\item[pg. 51] Lei de Amdahl, diz que aumento de desempenho possível dado por 
uma melhoria é limitado pelo montante que o dispositivo melhorado usa (suporta).
É uma medida quantitativa da lei do retorno diminuitivo.

$$Execution\ time\ after\ improvement = \frac{Execution\ time\ affected\ by\ 
improvement}{Amount\ of\ improvement} + Execution\ time\ unaffected$$

\pagebreak
\begin{center}
\textbf{Pitfall}
\end{center}

$$Execution\ time\ after\ improvement =  \frac{80\ segundos}{n} + (100 - 80\ 
segundos)$$

$$20\ segundos =  \frac{80\ segundos}{n} + (20)$$

$$0 =  \frac{80\ segundos}{n}$$

\item[pg. 52] Lei de Amdahl é por vezes, usada para argumentar para limites 
práticos para o número possível de processadores em paralelo. Ou seja, chega 
uma hora que existirá uma barreira de processadores trabalhando em paralelo.

\textbf{Falácia}: Computadores de baixo custo utilizam pouca energia.

Eficiência de energia importa bastante quando se fala de computadores custo
baixo, por causa que workload em servidores variam muito. Utilização de CPU para
servidores da Goodle, por exemplo, está entre 10\% e 50\% na maioria do tempo e
100\% em menos que 1\% do tempo.

\begin{table}[ht!]
\resizebox{\linewidth}{!}{
\begin{tabular}{|c|c|c|c|c|c|c|c|c|c|c|c|}

\hline Server Manufacturer                &
       Micro Processor                    &
       $\frac{Total\ Cores}{Sockets}$     &
       Clock Rate                         &
       Peak Performance (ssj\_ops)        &
       100\% Load Power                   &
       50\% Load Power                    &
       $\frac{50\%\ Load}{100\%\ Power}$  &
       10\% Load Power                    &
       $\frac{10\%\ Load}{100\%\ Power}$  &
       Active Idle Power                  &
       $\frac{Active Idle}{100\%\ Power}$ \\

\hline HP & Xeon E5440 & $\frac{8}{2}$ & 3.0 Ghz & 308022 & 269 W & 227 W & 
84\% & 174 W & 65\% & 160 W & 59\% \\
\hline 
\end{tabular}
}
\caption{Até mesmo servidores que são utilizados 10\%, queimam próximo de dois
terços do seu pico de energia.}
\end{table}

Desde que o workload de servidores varie mas use uma larga fração do seu pico de
energia, Luiz Barroso e Urs Hölzle [2007] argumentaram que nós deveríamos
redesenhar hardware para atingir "energia proporcional de computação". Se
futuros servidores usaram 10\% do seu pico de energia à 10\% de worload, nós
poderíamos reduzir o custo energético de central de dados, numa era pró-verde e
de redução da emissão de $CO_{2}$ isso pode vir a calhar.

\item[pg. 53] Milhões de instruções por segundo (MIPS) é uma medida de
velocidade da execução de programas baseado no número de milhões de instruções.

$$MIPS = \frac{Contagem\ de\ instr}{Tempo\ de\ exec\ \times  10^{6}}$$

$$MIPS = \frac{Contagem\ de\ instr}{\frac{Contagem\ de\ instr \times
CPI}{\frac{1}{Tempo\ de\ exec}} \times 10^{6}}$$

$$MIPS = \frac{Contagem\ de\ instr}{\frac{Contagem\ de\ instr \times
CPI}{Clock\ Rate} \times 10^{6}}$$

\item[pg. 54] \textbf{Tempo de execução} é a única opção impecável de medição 
da performance.

$$\frac{Tempo\ (Segundos)}{Programa} = \frac{Contagem\ Instr.}{Programa} \times 
\frac{Ciclos\ de\ Relog.}{Contagem\ Instr.} \times \frac{Tempo\ 
(Segundos)}{Ciclos\ de\ Relog.}$$

Individualmente esses fatores não definem performance. Mas o produto deles 
igual ao tempo de execução, vira então, uma medida confiável de performance.
\end{enumerate}


\paragraph{1.12.1} Achar o CPI se temos um ciclo de relógio é 0,333 
nanosegundos.

\begin{verbatim}
+---------------------------------------------------------------+
|    Name  | Intr. Count 10^9 | Execution Time | Reference Time |
+---------------------------------------------------------------+
| a. bzip2 | 2389             | 750 s          | 9650  s        |
+---------------------------------------------------------------+
| b. go    | 1658             | 700 s          | 10490 s        |
+---------------------------------------------------------------+
\end{verbatim}

{\flushleft

\textbf{bzip2}

$CPUfreq = \frac{1}{0,333 \times 10^{-9}}$

$CPUfreq = 5,005005005 \times 10^{9}\ \textbf{Hz} = 5\ \textbf{Ghz}$

$CPI = 5 \times 10^{9} \times 
\frac{Execution Time}{Instr.\ Count \times 10^{9}} = 5 \times 10^{9} \times 
0,313938887 \times 10^{-9}$

$CPI = 5 \times 10^{9} \times 
\frac{750}{2389 \times 10^{9}} = 5 \times 10^{9} \times 
0,313938887 \times 10^{-9} = 1,569694433 \rightarrow 1,57\ Ciclos\ por\ instr.$

}

{\flushleft

\textbf{go}

$CPUfreq = \frac{1}{0,333 \times 10^{-9}}$

$CPUfreq = 5,005005005 \times 10^{9}\ \textbf{Hz} = 5\ \textbf{Ghz}$

$CPI = 5 \times 10^{9} \times 
\frac{Execution Time}{Instr.\ Count \times 10^{9}} = 5 \times 10^{9} \times 
0,313938887 \times 10^{-9} = 1,569694433 \rightarrow 1.6\ Ciclos\ por\ instr.$

$CPI = 5 \times 10^{9} \times 
\frac{700}{1658 \times 10^{9}} = 5 \times 10^{9} \times 
0,313938887 \times 10^{-9} = 2,110977081 \rightarrow 2.1\ Ciclos\ por\ instr. $

}

\paragraph{1.12.2}

{\flushleft

\textbf{bzip2}

$CPUfreq = \frac{1}{0,333 \times 10^{-9}}$

$CPUfreq = 5,005005005 \times 10^{9}\ \textbf{Hz} = 5\ \textbf{Ghz}$

$CPI = 5 \times 10^{9} \times 
\frac{Execution Time}{Instr.\ Count \times 10^{9}} = 5 \times 10^{9} \times 
0,313938887 \times 10^{-9}$

$CPI = 5 \times 10^{9} \times 
\frac{750}{2389 \times 10^{9}} = 5 \times 10^{9} \times 
0,313938887 \times 10^{-9} = 1,569694433\ Ciclos\ por\ instr.$

}

{\flushleft

\textbf{go}

$CPUfreq = \frac{1}{0,333 \times 10^{-9}}$

$CPUfreq = 5,005005005 \times 10^{9}\ \textbf{Hz} = 5\ \textbf{Ghz}$

$CPI = 5 \times 10^{9} \times 
\frac{Execution Time}{Instr.\ Count \times 10^{9}} = 5 \times 10^{9} \times 
0,313938887 \times 10^{-9} = 1,569694433 \rightarrow 1.6\ Ciclos\ por\ instr.$

$CPI = 5 \times 10^{9} \times 
\frac{700}{1658 \times 10^{9}} = 5 \times 10^{9} \times 
0,313938887 \times 10^{-9} = 2,110977081 \rightarrow 2.1\ Ciclos\ por\ instr. $

}

\paragraph{1.12.3}

\textbf{bzip2}

$SPECratio = \frac{RefTime}{ExecTime} = \frac{9650}{750} = 12,86$

\textbf{go}

$SPECratio = \frac{RefTime}{ExecTime} = \frac{10490}{700} = 14,98$

\textbf{média geométrica}

$GEOmean = \sqrt[2]{12,866666667 \times 14,985714286} = \sqrt[2]{192,81} 
= 13,88$

\pagebreak
\paragraph{1.14.1}
A seção 1.8 cita como pitfall a utilização de um subconjunto da equação da 
performance como uma medida de desempenho. Para ilustrar isso, considere as 
configurações abaixo para execução de um programa em diferentes processadores.

\begin{tabular}{|l|l|l|l|l|}
\hline  & Processador & Clock Rate & CPI & Nro. Instr. \\
\hline a. & P1 & 4 Ghz & 0.9 & 5.00E+06 \\
\hline  & P2 & 3 Ghz & 0.75 & 1.00E+06 \\
\hline b. & P1 & 3 Ghz & 1.1 & 3.00E+06 \\
\hline  & P2 & 2.5 Ghz & 1.0 & 0.50E+06 \\
\hline
\end{tabular}

Uma falácia comum, é considerar o computador com a maior frequência como tendo 
garantidamente a melhor performance. Veja se isso é verdade para os 
processadores da tabela acima.

$Tempo_{a.P1} = Nro.\ Instr \times \frac{CPI}{Clock Rate} = 5 \times 10^{6} 
\times \frac{0,9}{4\ GHz} = 1,125 \times 10^{-3}\ segundos$

$Tempo_{a.P2} = Nro.\ Instr \times \frac{CPI}{Clock Rate} = 1 \times 10^{6} 
\times \frac{0,75}{3\ GHz} = 0,25 \times 10^{-3}\ segundos$

$Tempo_{b.P1} = Nro.\ Instr \times \frac{CPI}{Clock Rate} = 3 \times 10^{6} 
\times \frac{1,1}{3\ GHz} = 1,1 \times 10^{-3}\ segundos$

$Tempo_{b.P2} = Nro.\ Instr \times \frac{CPI}{Clock Rate} = 0,5 \times 10^{6} 
\times \frac{1,0}{2,5\ GHz} = 0,2 \times 10^{-3}\ segundos$

Vemos que $Tempo_{a.P1} > Tempo_{a.P2}$. Tempo maior quer dizer que performance 
de $Tempo_{a.P1}$ $<$ performance de $Tempo_{a.P2}$.

Vemos também que $Tempo_{b.P1} > Tempo_{b.P2}$. Tempo maior quer dizer que 
performance de $Tempo_{b.P1}$ $<$ performance de $Tempo_{b.P2}$.

\paragraph{1.14.2} Outra falácia é considerar que o processador executando o
maior número de instruções vai precisar de maior Tempo da CPU. Considerando que
o processador P1 está executando a sequencia de $10^{6}$ instruções e que o CPI
dos processadores P1 e P2 não mudam, \underline{\bfseries determine o número 
de instruções que o P2 pode executar} no mesmo tempo que P1 precisa para 
executar $10^{6}$ instruções.

$Tempo_{a.P1} = Nro.\ Instr \times \frac{CPI}{Clock Rate} = 10^{6}
\times \frac{0,9}{4\ GHz} = 0,225 \times 10^{-3}\ segundos$

$Tempo_{a.P2} = Nro.\ Instr \times \frac{CPI}{Clock Rate} =  Nro.\ Instr
\times \frac{0,75}{3\ GHz} = 0,225 \times 10^{-3}\ segundos$

$Nro.\ Instr = \frac{3\ GHz \times 0,225 \times 10^{-3}\ segundos}{0,75} = 
0,9 \times 10^{6}\ \textbf{instruções}$

\pagebreak
$Tempo_{b.P1} = Nro.\ Instr \times \frac{CPI}{Clock Rate} = 10^{6}
\times \frac{1,1}{3\ GHz} = 0,36 \times 10^{-3}\ segundos$

$Tempo_{b.P2} = Nro.\ Instr \times \frac{CPI}{Clock Rate} =  Nro.\ Instr
\times \frac{1,1}{2,5\ GHz} = 0,36 \times 10^{-3}\ segundos$

$Nro.\ Instr = \frac{3\ GHz \times 0,225 \times 10^{-3}\ segundos}{0,75} = 
0,8 \times 10^{6}\ \textbf{instruções}$

\textbf{1.14.2a}. Número de instruções para \underline{a.P2} aumentou muito 
pouco para a nova configuração.

\textbf{1.14.2b}. Número de instruções para \underline{b.P2} aumentou muito 
pouco para a nova configuração.

\textbf{Comentário:} Quer dizer que é uma falácia considerar genericamente que o
processador com maior número de instruções levará (por causa disso) maior tempo
de CPU.

\paragraph{1.14.3} Calcule os MIPS para cada processador.

$MIPS_{a.P1} = \frac{Clock Rate}{CPI \times 10^{6}} = \frac{4\ \textbf{GHz}}{0,9
\times 10^{6}} = 4444$

$MIPS_{a.P2} = \frac{Clock Rate}{CPI \times 10^{6}} = \frac{3\
\textbf{GHz}}{0,75 \times 10^{6}} = 4000$

$MIPS_{b.P1} = \frac{Clock Rate}{CPI \times 10^{6}} = \frac{3\ \textbf{GHz}}{1,1
\times 10^{6}} = 2727$

$MIPS_{b.P2} = \frac{Clock Rate}{CPI \times 10^{6}} = \frac{2,5\ 
\textbf{GHz}}{1,0 \times 10^{6}} = 2500$

Vemos que $MIPS_{a.P1} > MIPS_{a.P2}$. Porém $MIPS_{a.P1}$ leva mais tempo para 
executar 5 milhões de instruções do que $MIPS_{a.P2}$ que leva menos tempo para 
executar 1 milhão de instruções.

Vemos também que $MIPS_{b.P1} > MIPS_{b.P2}$.


\end{document}