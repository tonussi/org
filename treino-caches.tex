\documentclass[11pt,twocolumn]{article}
\usepackage[utf8]{inputenc}
\usepackage[brazil]{babel}
\usepackage{makeidx}
\renewcommand{\thefootnote}{\alph{footnote}}
\setlength{\parskip}{\baselineskip}
\setlength{\parindent}{0pt}
\author{Lucas}
\title{Treino Caches}
\begin{document}
\maketitle

Cache tipo-diretamente-mapeada. 1 bit de validade, endereços físicos de 32 bits:

\begin{tabular}{|l|l|l|}
\hline tag & índice & offset \\ 
\hline 16 & 7 & 9 \\ 
\hline 
\end{tabular}

%| tag  | índice | offset |
%|---|---|---|
%| 16 | 7  | 9 |

$2^{7} \times (16 + 1 + 2^{9} \times 8)\ bits$

$128 \times (16+1+4096)\ bits$

$\frac{264320}{8}=33040\ bytes$

--------------------------------------------------

Cache tipo-2-way. 1 bit de validade, endereços físicos de 32 bits:

\begin{tabular}{|l|l|l|}
\hline tag & índice & offset \\ 
\hline 13 & 9 & 10 \\ 
\hline 
\end{tabular}

%| tag  | índice | offset |
%|---|---|---|
%| 13 | 9  | 10 |

$2\times(2^{9}\times(13+1+2^{10}\times8))\ bits$

$2\times(512\times(13+1+8192))\ bits$

$\frac{8402944}{8}=1050368\ bytes$

--------------------------------------------------

Cache tipo-3-way. 1 bit de validade, endereços físicos de 34 bits:

Mas nesse exemplo, temos que a cache tem 24576 KB de dados.

\begin{tabular}{|l|l|l|}
\hline tag & índice & offset \\ 
\hline ? & ? & 12 \\ 
\hline 
\end{tabular}

%| tag  | índice | offset |
%|---|---|---|
%| ? | ?  | 12 |

Para revelar as interrogações:

$24576\ \textbf{\tiny KB} \times 1024 \times 8 = 201326592\ bits$ (de 
dados)\footnote{$24576\ \textbf{\tiny KB} \times 1024 = 25165824\ bytes$ serve 
para transformar de KB para bytes e vezes 8 para mostrar a quantidade em 
bits.}

$\frac{201326592}{3\ ways} = 67108864\ bits$ (por via)

$\frac{8388608}{2^{12}} = 16384 = 2^{14}\ entradas$ (conjuntos)

\begin{tabular}{|l|l|l|}
\hline tag & índice & offset \\ 
\hline $34-14-12=8$ & 14 & 12 \\ 
\hline 
\end{tabular}

%| tag  | índice | offset |
%|---|---|---|
%| $34-12-11=11$ | 11  | 12 |

$3\times(2^{14}\times(8+1+(2^{12}\times8)))\ bits$

$3\times(16384\times(8+1+32768)) = 1611055104\ bits$

$\frac{201400320}{8} = 201381888\ bytes$

$\frac{201381888}{1024} = 196662\ KB$ (total da cache)

--------------------------------------------------

\clearpage

Cache tipo-3-way. 1 bit de validade, endereços físicos de 32 bits:

Mas nesse exemplo, temos que a cache tem 24 MB de dados.

\begin{tabular}{|l|l|l|}
\hline tag & índice & offset \\ 
\hline ? & ? & 12 \\ 
\hline 
\end{tabular}

Para revelar as interrogações:

$24 \times 1024 \times 8 = 196608\ bits$ (de dados)

Aqui tomar cuidado com as unidades, não
divida bytes por bits ou o contrário.
Sempre bytes por bytes e bits por bits.

$\frac{196608}{3} = 65536\ bits$ (por via)

$\frac{65536}{2^{12} \times 8} = 2$ MB (conjuntos)

$2$ MB são $2^{11}\ bytes$ portanto já revelamos o índice que é 11.

\begin{tabular}{|l|l|l|}
\hline tag & índice & offset \\ 
\hline $32-12-11=9$ & 11 & 12 \\ 
\hline 
\end{tabular}

Espaço ocupado pela cache em sua totalidade: 
$3 \times 2^{11} \times (9 + 1 + 2^{12} \times 8) = 201388032$ (bytes)
$201388032\ bits = 24583,5$ (MB)

\end{document}