\documentclass{article}
\usepackage[utf8]{inputenc}
\usepackage[greek, brazil]{babel}
\usepackage[left=1cm, right=1.5cm, top=5cm, bottom=5cm]{geometry}
\usepackage{amsmath}
\usepackage{amsfonts}
\usepackage{amssymb}
\usepackage{makeidx}
\usepackage{graphicx}
\usepackage{hyperref}
\usepackage[usenames,dvipsnames]{xcolor}
\renewcommand{\thefootnote}{\alph{footnote}}
\setlength{\parskip}{\baselineskip}
\setlength{\parindent}{0pt}
\hypersetup {
  colorlinks,
  citecolor = NavyBlue,
  filecolor = NavyBlue,
  linkcolor = NavyBlue,
  urlcolor = NavyBlue
}
\author{Luke Skywalker}
\title{P1, P2, P3}
\begin{document}
\maketitle

\begin{enumerate}
\item Definir o alcance de um branch e jump:

\begin{enumerate}
\item a partir do endereço 0, o jump é 0xFFF FFFC
\item a partir do endereço 0, o branch é:
\end{enumerate}

\begin{verbatim}
= pc + 4 + 0111 1111 1111 1111 palavras
= 0 + 4 + 0111 1111 1111 1111 00 bytes
= 4 + 0001 1111 1111 1111 1100 bytes
= 4 + 0x1FFFC = 0x20000
\end{verbatim}

\item Ele dava uma memória virtual de 32 bits com endereços físicos de 28 bits,
dizia que o tamanho da página era de 1 MB = $2^{20}$. Pedia para calcular
quantos bits ocupava o endereço físico disso na tabela de páginas; $= 28 - 20 =
8$.

\item Ele dava um cache com 1024 KB, com blocos de 64 bytes. E era uma cache do
tipo 16-way. Uma memória com endereços de 40 bits. Calcular o tamanho do índice.

\textbf{Resposta}: Ela tem $(64\ bytes) / (4\ bytes ) = 16$ palavras por blocos.
Ela é uma cache do tipo 16-way, portanto possui 16 blocos por conjunto. Ela
possui (1024 KB) / (64 bytes) = ($2^{20}$ bytes) / ($2^{6}$ bytes) = $2^{14}$
blocos Então ($2^{14}$ blocos) / (16 blocos por conjunto) = $2^{10}$ conjuntos
Assim precisamos de 10 bits para indexar os 1024 conjuntos.

\item Pedia para calcular o tamanho da \underline{TAG} anterior 'd)'. $= 40  -
10(conjuntos) - 4(wordOFFset) - 2(byteOFFset) = 24$ bits de \underline{TAG}.

\item Transformar para binário a instrução subi \$t0, a0, 1. Lembrando que ela é
pseudo e o certo seria um \textsc{addi} com um número complementado; isto é,
transformar em:

\begin{verbatim}
addi, $t0, $a0, ffff
RT  = RS + EXT
$t0 = $a0 + -1
OPCODE  | RS     | RT     | IMMEDIATE
OPCODE  | $a0    | $t0    | -1
0010 00 | 00 100 | 0 1000 | 1111 1111 1111 1111
\end{verbatim}

\item Relacionada a escalonamento estático, ela dá um código, e você tem que
escalonar ele por que o processador não tem controle de hazards, e não pode
desfazer as instruções feitas utilizando o branch prediction estático programado
sempre para desvio não tomado, e precisa que se adicione nop (no operation)
quando há necessidade, e o hardware, possui todos os atalhos, (full forwarding):

\begin{verbatim}
addi $s0, $zero, 0
loop: lw $t0, 0($s0)
add $v0, $v0, $t0
addi $s0, $s0, -4
bne $s0, $zero, loop
jr $ra
\end{verbatim}

Que fica igual a:

\begin{verbatim}
addi $s0, $zero, 0
loop: lw $t0, 0($s0)
addi $s0, $s0, -4
add $v0, $v0, $t0
bne $s0, $zero, loop
nop
jr $ra
\end{verbatim}

\item Ele pedia para fazer loop unrolling com 2 vezes, para emissão múltipla de
2 instruções por ciclo com qualquer combinação possível, e a pegadinha dele é
que no anterior o branch prediction era estático, contava com desvio não tomado,
e não podia desfazer as instruções que ele executava, agora ele pode desfazer
instruções indevidamente previstas por um previsor dinâmico que agora, está
programado para sempre tomar os desvios). O hardware também possui todos dos
atalhos físicos, (full forwarding). Ah, outras pegadinhas: é só para desenrolar
o laço, e descartar as instruções de fora: addi \$s0, \$zero, 0 e jr \$ra.


\begin{tabular}{|c|c|}
\hline loop: lw \$t0, 0(\$s0)  & nop \\
\hline lw \$t1, 0(\$s0) & addi \$s0, \$s0, -8 \\
\hline add \$v0, \$v0, \$t0 & nop \\
\hline add \$v0, \$v0, \$t1 & bne \$s0, \$zero, loop \\
\hline espaço vazio na tabela & espaço vazio na tabela \\
\hline
\end{tabular}

\begin{enumerate}
\item Qual o tamanho do LAÇO no item a) em bytes quando executado em uma máquila
com emissão múltipla estática (nops também contam tamanho de código): Resposta:
$4 * 8 = 32$ bytes.

\item Qual o tamanho do LAÇO no item a) em bytes quando executado em uma máquila
com emissão múltipla dinâmica: Resposta: 4 * 8 = 32 bytes.
\end{enumerate}

\item Um sistema possui uma cache primária .de dados (L1-D), uma cache primária
de instruções (L1 -I), uma cache unificada secundária (L2) e uma memória
principal (MP). A penalidade de falta da cache secundária foi calculada na
Questão 1a (expressa em ciclos do FSB). A penalidade de falta da cache primária
é 5 (expressa em ciclos de CPU). A frequência de relógio da CPU é 5 vezes a
frequência de relógio do FSB. A taxa de fracassos combinada das caches primárias
(\underline{mr} = número total de acessos a L2 / número total de acessos à
memória) é de 2\%. A taxa de faltas global (\underline{gmr} = número total de
acessos à MP / número total de acessos à memória) é de 0,5\%. Sabe-se que um
programa executa 1 milhão de instruções, das quais 25\% são loads ou stores.

\begin{enumerate}
\item Quantos ciclos de CPU são gastos no acesso a L2? Mostre seus cálculos.
Resposta: 125000 ciclos. Cálculos: número total de acessos $= (0,25 \times I +
I) = 1250000$ para o acesso a L2 temos uma penalidade de Ciclos $= I \times mr
\times\ penalidade\ = 1250000 \times 0,02 \times 5 = 125000$ ciclos

\item Quantos ciclos de CPU são gastos no acesso a MP? Mostre seus cálculos.
ciclos de CPU para a memória principal = 36*5 = 180 ciclos para 1 acesso. Ciclos
totais $= 1250000 \times 0,005 \times 180 = 1125000$

\item Quantos ciclos de CPU são gastos na execução total do programa? Mostre
seus cálculos. Número total de acessos $= (0,25 \times I + I) = 1250000$
\end{enumerate}
\end{enumerate}


Questão: Suponha os seguintes cenários: Cenário 1: uma instrução j L reside no
endereço 0; Cenário 2 um instrução jr \$s1 reside no endereço 0. Para o MIPS,
quais os endereços alvos máximos atingíveis pelos desvios em cada um dos
cenários. Resposta: Cenário 1: = 0x 0FFF FFFC; Cenário 2: 0xFFFF FFFC.

Questão: Um dispositivo de saída é mapeado em memória. Sabe-se que ele se
comunica para com a CPU através da técnica de consulta periódica (“polling”). O
esboço de código abaixo mostra parte da implementação da consulta do
dispositivo. Conhece-se o conteúdo de quatro dos registradores: \$s0 = 0x 0000
0004; \$s1 = 0x 0000 0008; \$s2 = 0x 0000 ABC1; \$s4 = 0x 0000 FFAC. Para o qual
endereço de memória está mapeado o registrador de status do dispositivo de
saída? Não é preciso justificar.

Endereço = 0x 0000 0004, Pois no programa abaixo ele está setando justamente o
bit ''interrupt enable'', está fazendo-o no registrador que tem o endereço 0x4.


\begin{verbatim}
P: ...
   lw $s2, 0($s0)
   andi s3, s2, 0x00000001
   beq $zero, $s3, L
   ...
   sw $s4, 0($s1)
   ...
   ...
   ...
   j P
\end{verbatim}


Questão: Sejam D7 a D0 dispositívos de E/S capazes de ativar (‘1’) e desativar
(‘0’) as requisições de iterrupção IRQ7 e IRQ0, respectivamente, à entrada de um
controlador de interrupções. O controlador possui uma codificação que garante a
prioridade de $D_{i}$ sobre $D_{i+1}$ e contém 2 registradores mapeados em
memória:

\begin{itemize}
\item ireq-req (mapeando em 0x1000 8000): cada um de seus bits armazena (do mais
significativo para o menos significativo) o estado de cada uma das entradas IRQ7
a IRQ0 (respectivamente).

\item mask-reg (mapeando em 0x1000 8001): armazena uma
máscara de interrupção. Escreva um programa com no máximo 4 intruções nativas do
MIPS que torne IRQ7 a requisição de alta prioridade. Restrição: todas as
constantes devem ser escritas em hexadecimal (0x).
\end{itemize}


Resposta:
\begin{verbatim}
lui $t0, 0x1000
ori $t0, $t0, 0x8001
addi $t1, $zero, 0x80
sb $t1, 0($t0)
\end{verbatim}

Questão: Afirmação: “Uma vez configurado na CPU para transferir um grande bloco
de dados, um controlador DMA nunca emite um sinal de requisição de interrupção
para a CPU; daí sua vantagem em relação do mecanismo de E/S acionado por
interrupção”. A afirmação é Verdadeira ou Falsa? Justificação: Resposta: F (V ou
F). Justificativa: O controlador de DMA interrompe a CPU ao final do tratamento
para indicar que o bloco foi transferido.

Questão: Uma cache tem capacidade nominal de 512 Kilobytes e associatividade do
tipo 8-way (8 BLOCOS POR CONJUNTO), bloco de 128 bytes e utiliza as políticas
LRU e “write-back”. Além do bit de validade, essa cache utiliza um bit para
implementar o critério LRU e um outro bit (“dirty bit”) para implementar o
mecanismo de “write-back”. Sabe-se que os endereços são representados em 32
bits.


Questão: Quantos bits são utilizados para indexar a cache? Mostrar seus
cálculos. Resposta: 9 bits. Cálculos: \#blocos $= 512\ KB / (128\
\frac{B}{bloco}) = 4K = 2^{12}$ \# conjuntos = $2^{12} / 2^{3} = 2^{9}$


Questão: Quantos bits sao utilizados para TAG? Mostrar seus cáculos. Resposta:
16 bits. Cálculos: $TAG\ + INDEX\ + BLOCK OFFSET\ = 32 \rightarrow TAG\ = 32 - 9
- 7 = 16 (bloco\ tem\ 2^{7}\ bytes \rightarrow 7$ bits para block offset)


Questão: Qual o número total de bits armazenados na cache (expresso em Kbits; $K
= 2^{10})$? Justifique, mostrando seus principais cálculos.

Resposta: 4172 Kbits. Cálculos: \# blocos $= 2^{12}$
Comentário: Note que o esquema da cache é o seguinte:

\begin{verbatim}
conj 0:  [VB][LRU][DB][tag 16 bits][data 128B] ... [VB][LRU][DB][tag 16
bits][data 128B]
.
.
.
conj 2^9-1:  [VB][LRU][DB][tag 16 bits][data 128B] ... [VB][LRU][DB][tag 16
bits][data 128B]
\end{verbatim}

Cálculos: Cada bloco contém:
\begin{itemize}
\item CONTROL: 3 bits;
\item TAG: 16bits;
\item DATA: 128 x 8 = 1024 bits;
\end{itemize}
Total de bits $= 2^{12} \times (3+16+1024) = 2^{18} x 2^{2} x 1043\ bits\ =
2^{10} \times 4172\ bits\ = 4172$ Kbits [SIMILAR AO EXERCÍCIO 5.x.y, MAS PARA
CACHE 16-WAY!]

Questão: Um processador possui um subsistema de memória que consiste de uma uma
cache de dados e uma memória principal. São executados I instruções de um
programa das quais 15\% são do tipo “load” e 5\% do tipo “store”. Suponha que a
CPU tem um pipeline ideal (sem “hazards”) que inicie uma intrução a cada ciclo.
Assumindo que, para ambas as caches, a taxa de acertos é de 90\% e a penalidade
por causa de falta na cache é de 10 ciclos, calcule:

a) O número de ciclos de parada (“stall”) gerados pelas faltas no acesso a
instruções. Resposta: I ciclos. Cálculos: I x 10\% x 10 [ciclos] = I ciclos.
Comentário: Load e Stores são Dados, então não entra nem 15\%, nem 5\% aqui.
Porém, entram 90\% ou 10\% (complemento). Se você quiser calcular o importânte
que é stall gerados por instruções, você usa 10\%, pois 90\% de acertos não
participa da penalidade por stalls.

b) O número de ciclos de parada (“stall”) gerados pelas faltas no acesso a
dados. Resposta: 0,2I ciclos. Cálculos: (20\% x I) x 10 x 10\% = 0,2I ciclos.
Comentário: Aqui ele somou todas as porcentagens com relação a dados (LS =
15\%+5\%). Ou seja, De todas as instruções temos 20\% que são do tipo LS. Agora
que temos essa porcentagem em mãos, quanto será a penalidade total por elas?
Note, que ele diz “penalidade por falta na cache é de 10 ciclos”, isto é, falta
de dados e falta de instruções. Poderia ser 10 ciclos de penalidade para
instruções e “12” ciclos de penalidade para dados.

c) O número médio de ciclos por intrução total, incluindo o efeito de todas as
falhas, CPI médio: Resposta: 2,2 [ciclos/instr.] Cálculos: CPI médio = CPI ideal
x CPI stall = ( I + ( I + 0,2 I ) ) / I = I + 1,2I = 2,2. Comentário: Soma das
médias. (1 x I [ciclos]) / I instr. seria o CPI ideal. Então basta somar com o
CPI stall.

Questão: Um Tratador de exceções para o MIPS invoca procedimentos de até 3
parâmetros, que obedecem à convernção de chamada, utiliza pseudo-instruções
implementadas pelo montador e usa os registradores \$k0 e \$k1. Exceto pelos
registradores reservados para o “kernel” (\$k0 e \$k1), pelos registradores
envolvidos na invocação e retorno de procedimentos e no registrador de suporte
a pseudo-instruções (\$at), o tratador não modifica quaisquer outros
registradores. Sabe-se que o tratador preserva o conteúdo de todos os
registradores que possam por ele ser modificados, exceto aqueles reservados
para o “kernel”, armazenando-os numa área estática de dados que se inicia no
endereço 0x80010000.

Questão: Escreva uma sequência de código que armazena, a partir do endereço
0x80010000, todos os valores que precisam ser preservados pelo tratador.
Restrição: Use a intrução add \$k0, \$zero, \$at para preservar o valor de \$at
no registrador \$k0 até o final do tratador.

Questão: É possível modificar o código do item anterior para preservar o
conteúdo de \$at na área estática de dados do “\underline{kernel}”, ao invés de
preservá-lo no registrador \$k0, liberando esse registrador para uso no próprio
tratador. Mostre como modificar o código da questão anterior para liberar o
registrador \$k0. Dica: Em ambos os itens desta questão, você pode usar
instruções nativas ou pseudo-instruções.

(a)
\begin{verbatim}
add $k0, $zero, $at
sw $a0, 0x8001 0000
sw $a1, 0x8001 0004
sw $a1, 0x8001 0004
sw $a2, 0x8001 0008
sw $ra, 0x8001 000C
\end{verbatim}

(b)
\begin{verbatim}
add $k0, $zero, $at
sw $a0, 0x8001 0000
sw $a1, 0x8001 0004
sw $a2, 0x8001 0008
sw $ra, 0x8001 000C
sw $k0, 0x8001 0010
\end{verbatim}

Questão: O processador Opteron usa endereços virtuais de 48 bits e endereços
físicos de 40 bits e admite um tamanho de página de 4MB.

* Quantos bits são necessários para armazenar o número de um página física
(real) na tabela de páginas?

Resposta: 18 bits. Cálculos: Página = $2^{22}$ bytes, page offset representado
em 22 bits \#páginas físicas: endereço físico - page offset $= 40 - 22 = 18$
bits

Questão: Qual o tamanho da tabela de páginas expresso em megabytes ($2^{28}$
bytes = 1 MB), considerando que cada linha da tabela ocupe exatamente 4 bytes?
Resposta: 256 MB. Cálculos: \#linhas / \#páginas $= 2^{48} / 2^{22}$ Tamanho $=
2^{26} x 2^{2} = 2^{28}$ bytes $= 256$ MB

1. Fazer AND do campo de pending interrupt e a máscara interrupção (mfc0) ?
2. Setar? máscara de interrupções pendentes.
3. Selecione a interrupção de maior prioridade. Select the higher priority of
these interrupts.
4. Salvar a máscara de interrupção.
5. Desabilitar interrupções menos prioritarias.
6. Salvar o contexto.
7. Chamar a rotina de tratamento apropriada.
8. Restaurar o campo de máscara.
9. Restaurar o contexto.
10. Retornar do tratador.

Para que o tratador de interrupções seja capaz de ser interrompido por
interrupções de maior prioridade, o bit IE deve ser ativado (em nível 1) entre
quais os passos acima enumerados? Não é preciso justificar. Resposta: Entre os
passos 6 e 7.

Elaboration: The two least significant bits of the pending interrupt and
interrupt mask fields are for software interrupts, which are lower priority.
These are typically used by higher-priority interrupts to leave work for
lower-priority interrupts to do once the immediate reason for the interrupt is
handled. Once the higher-priority interrupt is finished, the lower-priority
tasks will be noticed and handled.

Here are the steps that must occur in handling an interrupt:
1. Logically AND the pending interrupt field and the interrupt mask field to
see which enabled interrupts could be the culprit. Copies are made of these two
registers using the mfc0 instruction.
2. Select the higher priority of these interrupts. The software convention is
that the leftmost is the highest priority.
3. Save the interrupt mask field of the Status register.
4. Change the interrupt mask field to disable all interrupts of equal or lower
priority.
5. Save the processor state needed to handle the interrupt.
6. To allow higher-priority interrupts, set the interrupt enable bit of the
Cause register to 1.
7. Call the appropriate interrupt routine.
8. Before restoring state, set the interrupt enable bit of the Cause register
to 0. This allows you to restore the interrupt mask field.


Questão: Desenrolar o corpo do laço abaixo para expor duas iterações …
resultante para minimizar o número de ciclos necessários à correta … Suponha
que cada pipeline tenha 5 estágios e utilize todos os atalhos físico … desvio
condicional seja resolvido no estagio ID e que, no estágio WB … primeiro semi
ciclo enquanto, no estágio IDm a leitura dos registradores … que a unidade de
controle não possua um detector de hazards, sendo ... todos os nops necessários
para preservar a semântica do código … instruções pode ser emitida em um mesmo
ciclo de relógio, mas pela … emitida a cada ciclo, um nop deve ocupar a parte
de um slot em que … útil.


\begin{verbatim}
        add $s1, $zero, $zero
loop:   lw $t0, 100($s0)
        add $s1, $t0, $s1
        addi $s0, $s0, 4
        beq $s0, $s2, loop
\end{verbatim}


\begin{verbatim}
        add $s1, $zero, $zero
loop:   lw $t0, 100($s0)
lw $t1, 104($s0)
        add $s1, $t0, $s1
add $s1, $t1, $s1 # podemos optimizar o código nessa questão?
        addi $s0, $s0, 8
        beq $s0, $s2, loop
\end{verbatim}


\begin{verbatim}
        add $s1, $zero, $zero
loop:   lw $t0, 100($s0)
lw $t1, 104($s0)
        add $s1, $t0, $s1
        addi $s0, $s0, 8
        beq $s0, $s2, loop
\end{verbatim}


Cenário único: Pipeline emite 2 instruções por ciclo; o desvio condição:
\begin{verbatim}
lw $t0, 100($s0)
lw $t1, 104($s0)
addi $s0, $s0, 8
o dado em $t0 demora a ficar pronto?
nop
add $s1, $t0, $s1
Alu Alu Forwarding $s1 -> $s1
nop
add $s1, $t1, $s1
nop
beq $s0, $s2, loop
nop
\end{verbatim}

(Critério de correção: Cada erro desconta 0,5 ponto até que se anule a questão)

Questão: Cada um dos códigos abaixo .. diferente, cujos atalhos sao
desconhecidos, mas sabe-se que todos os pipelines … basica de 5 estágios (IF,
ID, EX, MEM, WB) ilustrada no Anexo V. Sabe-se  … escrita de um registrador
ocorre no primeiro semi ciclo do estágio WB e dois … segundo semi ciclo do
estágio ID. Sabe-se também que o número de nops inseridos … necessário para que
nenhuma pausa seja gerada pelo hardware durante a … datapath, marque com um ‘E’
o(s) atalho(s) que garantidamente existe(m), e com ‘?’ se nada puder ser
garantido sobre… Deduza a existência ou inexistência só a partir do código e
das informações … diagrama de ocupação como auxílio à resolução, mas ele não
será corrigido; só o ….

Datapath 1: Resposta (     ) ALU $\rightarrow$ ALU; (      ) ALU $\rightarrow$
MEM; (       ) MEM $\rightarrow$ ALU?

Questão: Suponha que uma instrução add causou “overflow” e chamou um tratador
de exceções, do qual o código abaixo faz parte. Lembre que o registrador \$14 é
o EPC. Suponha que o \$k1 contenha um valor (que corresponde à saturação em um
dos limites da faixa de representação) a ser armazenado no registrador destino
da instrução “add” que causou “overflow”. Complete o código abaixo, de forma
que o valor saturado armazenado em \$k1 seja escrita no registrador-destino da
instrução que causou a exceção. (Esta é uma oportunidade de demonstrar que você
realmente participou do desenvolvimento do código durante o Lab 07).

(Critério de avaliação: Esta questão não será pontuada parcialmente)

\begin{verbatim}
           mfc0 $k0, $14
           lw   $k0, 0($k0)
           andi $t0, $k0, 0xF800
           lw   $k0, corretiva
           li   $t1, 0x FFFF 07FF
           and  $k0, $k0, $t1
           or   $k0, $k0, $t0
           sw   $k0, corretiva
corretiva: add  $t0, $zero, $k1
\end{verbatim}

Questão: Um Tratador de exceções para o MIPS invoca procedimentos de até 3
parâmetros, que obedecem à convenção de chamada, utiliza pseudo-instruções
implementadas pelo Montador e usar os registradores $k0 e $k1. Exceto os
registradores reservados para o “kernel” ($k0 e $k1), pelos registradores
envolvidos na invocação e retorno de procedimentos e no registrador de suporte a
pseudo-instruções (\$at), o tratador não modifica quaisquer outros
registradores. Sabe-se que o tratador preserva o conteúdo de todos os
registradores que possam por ele ser modificados, exceto aqueles reservados para
o “kernel”, armazenando-os numa área estática de dados que se inicia no endereço
0x80010000.

Questão: Escreva uma sequência de código que armazena, a partir do endereço
0x80010000, todos os valores que precisam ser preservados pelo tratador.
Restrição: Use a instrução add \$k0, \$zero, \$at para preservar o valor de \$at
no registrador \$k0 até o final do tratador. É possível modificar o código do
item anterior para preservar o conteúdo de \$at na área estática de dados do
“kernel”, ao invéz de preservá-lo no registrador \$k0, liberando esses
registrador para uso no próprio tratador. Mostre como modificar o código da
questão anterior para liberar o registrador \$k0. Dica: em ambos os itens desta
questao, você pode usar instruções nativas ou pseudo-instruções.

Questão: Dado o código abaixo, escrito em linguagem C, onde \textsc{A[i][j]}
representa o elemento à linha i e à coluna j da matriz A. Suponha que as
variáveis I, J e x sejam todas alocadas em registrador e não interferem na cache
(D-cache). Cada elemento de A é um número inteiro representado em 32 bits e a
matriz A é de $4\times4$. Suponha que a D-cache tenha 16 bytes por bloco, que o
elemento \textsc{A[0][0]} foi armazenado no endereço 0 da MP e que a D-cache
esteja vazia (todos os blocos inválidos) antes de se iniciar a execução desse
código. Nestas condições, qual a sequência resultante de acertos (A) e falhas
(F) a D-cache depois da execução desse código? Lembrete em C, os elementos de
uma mesma linha da matriz são armazenados em … (contíguos?) somente depois de
armazenada uma linha completa, a próxima linha é armazenada em endereços
contíguos e crescentes.

\begin{verbatim}
for (int i = 0; i < 4; i = i + 1)
  for (int j = 0; j < 4; j = j + 1)
    x = x + a[j][i]
\end{verbatim}

Questão: Um sistema usa endereços virtuais de 32 bits e paginas de 4KB com os
seguintes atributos V (valid bit), D (dirty bit) e R (reference bit). Quando o
conteúdo do TLB é o abaixo mostrado, em … física reside a variável cujo
endereço virtual e 1010 0000 0000 0000 0000 1111 1111 1111. Dica: 1010 0000
0000 0000 0000 1111 1111 = 0x A000 0FFF.

\begin{tabular}{|c|c|c|c|c|}
\hline v & D & R & TAG & Número \\
\hline 0 & 0 & 0 & 0xA0000 & 0x0001 \\
\hline 0 & 0 & 0 & 0xC0000 & 0x7777 \\
\hline 1 & 0 & 0 & 0xA0000 & 0xFFF0 \\
\hline
\end{tabular}

Questão: Um subsistema de memoria consiste de uma cache primaria (L1), uma cache
secundária (L2) e uma memória principal (MP). Seja TA o tempo de acesso. Sabe-se
que TA(L1) = 1ns, TA(L2) = 10ns e TA(MP) = ? e que a taxa de faltas em L1 é de
2\% e que a taxa global de faltas medida na cache equivalente … Qual o \%tempo
medio (TM) para acessar um item em memória?

Questão: Ao ler a seção 5.3 do livro texto,  você aprendeu a noção de tempo de
acesso à memória (average memory access time) e como calculá-lo para uma
hierarquia de 2 níveis; cache L1 e memória (Exercicio 5.7.4), você aprendeu a
generalizar o AMAT para uma hierarquia de 3 níveis (L1) agora que você é capaz
de generalizar esse cáculo para uma hierarquia de 4 níveis (L1, L2, Memória)
AMAT para o sistema descrito na tabela abaixo. Não é preciso justificar.

\begin{tabular}{|c|c|c|c|}
\hline L1 & L2 & L3 & MP \\
\hline 1ns & 5ns & 20ns & 100ns \\
\hline 10\% & 20\% & 30\% & ~ \\
\hline
\end{tabular}

AMAT =    ns

Questão: Suponha que P1 e P2 realizem acessos de leitura e escrita em um
arranjo de inteiros X ocupa exatamente um bloco em cache. Cada processador
realiza acessos de leitura e escrita em diferentes … bloco de sua cache privada
que contém X. Suponha que o estado inicial desse bloco seja X[0] = X[1] = 1 em
cada cache privada. Sejam os acessos ao arranjo X especificados no código
abaixo:

\begin{tabular}{|c|c|}
\hline P1 & P2 \\
\hline X[0]++; X[1]=8; & X[0]=6; X[1]++; \\
\hline
\end{tabular}

(Refazer a questão, considerando uma falha no mecanismo de Write Throught onde
ele não consegue invalidar blocos através desse mecanismo.)

Apresente um par de valores (X[0], X[1]), observável na cache privada do
processador P2 imediatamente após a execução do código acima, que atestaria a
implementação incorreta do protocolo de coerência. Considere que X[0]++ precede
X[0] = 6; X[1] = 8 precede X[1]++.

Resposta: (    ,    ). Não é preciso justificar.

Nesse mesmo cenário, se o protocolo estiver correto, qual o valor do bit de
validade do bloco do processador P1? Resposta: V = 0 (zero, invalidado). Não é
preciso justificar.

Comentário: Uma vez que B escreveu X[1] = 8++, sendo esse 8 um valor anterior
cujo Processador A salvou anteriormente. E agora B está reescrevendo 8 + 1 = 9.
Por esse motivo, por B estar salvando um novo valor em X[1] o valor na cache de
A será invalidado (\underline{Valid} Bit = 0).

Questão: A Tabela l mostra de forma simbólica, para alguns endereços de memória
na faixa de 0x0 a 0x31C, o conteúdo da memória principal. As colunas em branco
são campos auxiliares para facilitar a correspondência entre endereços
hexadecimais e binários (seu preenchimento não será pontuado).

\begin{table}[ht!]
\centering
\begin{tabular}{|c|c|c|c|c|c|}
\hline Endereço (0x) & End. [7:0] (0b) & Conteúdo & Endereço (0x) & End. [7:0]
(0b) & Conteúdo \\
\hline 0000 0000 &  & A & 0000 0100 &  & \textgreek{α} \\
\hline 0000 0004 &  & B & 0000 0104 &  & \textgreek{β} \\
\hline 0000 0008 &  & C & 0000 0108 &  & \textgreek{χ} \\
\hline 0000 000C &  & D & 0000 010C &  & \textgreek{δ} \\
\hline 0000 0010 &  & K & 0000 0110 &  & \textgreek{π} \\
\hline 0000 0014 &  & L & 0000 0114 &  & \textgreek{θ} \\
\hline 0000 0018 &  & M & 0000 0118 &  & \textgreek{ρ} \\
\hline 0000 001C &  & N & 0000 011C &  & \textgreek{ω} \\
\hline 0000 0020 &  & W & 0000 0300 &  & E \\
\hline 0000 0024 &  & X & 0000 0304 &  & F \\
\hline 0000 0028 &  & Y & 0000 0308 &  & G \\
\hline 0000 002C &  & Z & 0000 030C &  & H \\
\hline 0000 0030 &  & P & 0000 0310 &  & T \\
\hline 0000 0034 &  & Q & 0000 0314 &  & U \\
\hline 0000 0038 &  & R & 0000 0318 &  & V \\
\hline 0000 003C &  & S & 0000 031C &  & J \\
\hline
\end{tabular}
\caption{Conteúdo (parcial) da memória principal.}
\end{table}

\begin{table}[ht!]
\centering
\begin{tabular}{|c|c|c|c|c|c|c|c|c|c|c|c|c|c|c|c|c|}
\hline Palavra & 000 & 001 & 010 & 011 & 100 & 101 & 110 & 111 & 000 & 001 &
010 & 011 & 100 & 101 & 110 & 111 \\
\hline Conjunto 0 &  &  &  &  &  &  &  &  &  &  &  &  &  &  &  &  \\
\hline Conjunto 1 &  &  &  &  &  &  &  &  &  &  &  &  &  &  &  &  \\
\hline Conjunto 2 &  &  &  &  &  &  &  &  &  &  &  &  &  &  &  &  \\
\hline Conjunto 3 &  &  &  &  &  &  &  &  &  &  &  &  &  &  &  &  \\
\hline Conjunto 4 &  &  &  &  &  &  &  &  &  &  &  &  &  &  &  &  \\
\hline Conjunto 5 &  &  &  &  &  &  &  &  &  &  &  &  &  &  &  &  \\
\hline Conjunto 6 &  &  &  &  &  &  &  &  &  &  &  &  &  &  &  &  \\
\hline Conjunto 7 &  &  &  &  &  &  &  &  &  &  &  &  &  &  &  &  \\
\hline
\end{tabular}
\caption{Status da cache após a sequência de acessos.}
\end{table}

Referências à memória: 1ªref - 0xE4, 2ªref - 0x60, 3ªref - 0x1Fc, 4ªref - 0x3E8,
5ªref – 0xFC. Considere uma cache do tipo 2-way, inicialmente vazia, com 128
palavras, sendo que cada bloco contém 8 palavras. Preencha a Tabela 2 com o
conteúdo final da cache imediatamente após aplicada a sequência de referências
acima, usando os seguintes critérios e convenções: 1-Havendo 2 blocos livres num
conjunto, o bloco trazido da memória deve ser armazenado no bloco livre de menor
número (o preenchimento dos blocos na Tabela 2 deve ser da esquerda para a
direita); 2-Havendo 1 bloco livre, nele deve ser armazenado o bloco trazido da
memória; 3-Não havendo blocos livres, um dos blocos deve ser substituído de
acordo com o critério LRU (dentre os dois blocos o último preenchido ainda é
mais atual que o segundo, então o segundo será anulado e preenchido com novos
dados, ou seja o segundo foi o menos usado recentemente MUR = LRU); 4-O conteúdo
de cada bloco válido deve ser indicado explicitando todas as suas palavras.

Comentário: Traduza para o formato \verb|[tag][index][offset]| apenas os
endereços que estão marcados nas referência à memória.

Questão: Qual o IPC de um GNU/ARM.

32G / 2G = 16
CPI = 1/2 = 1/IPC

Questão: IRQ\_PRIORITY. Nessa questão você tem que dar prioridade para
$D_{i+1}$.

Questão: In computer science, load-link and store-conditional (LL/SC) are a
pair of instructions used in multithreading to achieve synchronization.
Load-link returns the current value of a memory location, while a subsequent
store-conditional to the same memory location will store a new value only if no
updates have occurred to that location since the load-link. Together, this
implements a lock-free atomic read-modify-write operation.

Questão: Conte em hexadecimal (Você tem que aprender que é importante saber
contar em hexadecimal, superimportante para organização de computadores. E para
a sua vida. Obviamente você tem que vir preparado, pois claro.. tem que saber
contar hexadecimal super importante.)

Dado o programa abaixo, sabendo que cada instrução gerada ocupa 32 bits de
memória, isto é 4 bytes. E que estão alinhadas byte a byte. Conte quanto espaço
o programa \verb|scan.c| vai ocupar na memória após a ligação dinâmica.

\begin{verbatim}
#include <add.h>
#include <stdio.h>
int main() {
  int x=10, y=20.
  printf("\n\%d+ \%d= \%d", x, y, add(x,y));
  return 0;
}
\end{verbatim}

\begin{verbatim}
.data
.text
.glob main
main: jal scan
jal printf
\end{verbatim}

\begin{verbatim}
int add(int quant1, int quant2) {
  return (quant1 + quant2);
}
\end{verbatim}

\begin{verbatim}
.text
.glob scan
scan: jr #ra
\end{verbatim}


\end{document}
