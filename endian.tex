\documentclass{article}
\usepackage[utf8]{inputenc}
\usepackage[greek, brazil]{babel}
\usepackage[left=1cm, right=1.5cm, top=1.5cm, bottom=4.5cm]{geometry}
\usepackage{amsmath}
\usepackage{amsfonts}
\usepackage{amssymb}
\usepackage{makeidx}
\usepackage{graphicx}
\usepackage{hyperref}
\usepackage[usenames,dvipsnames]{xcolor}
\renewcommand{\thefootnote}{\alph{footnote}}
\setlength{\parskip}{\baselineskip}
\setlength{\parindent}{0pt}
\hypersetup {
  colorlinks,
  citecolor = NavyBlue,
  filecolor = NavyBlue,
  linkcolor = NavyBlue,
  urlcolor = NavyBlue
}
\author{Lucas}
\title{Endian}
\begin{document}
\maketitle

Imagine você lendo (da esquerda para direita $\rightarrow$) cada caractere dessa
\verb|string "ABCDEFGH"| como um byte\footnote{em java um byte pode armazenar 
dois caracteres ascii}.

\begin{table}[ht!]
  \centering
  \begin{tabular}{|c|c|c|c|c|c|c|c|}
  \hline A & B & C & D & E & F & G & H \\ 
  \hline 7 & 6 & 5 & 4 & 3 & 2 & 1 & 0 \\ 
  \hline 
  \end{tabular}
  \caption{Big Endian, começa endereçar pelo MSB. Primeiro exemplo.}
\end{table}

\begin{table}[ht!]
  \centering
  \begin{tabular}{|c|c|c|c|c|c|c|c|}
  \hline H & G & F & E & D & C & B & A \\ 
  \hline 7 & 6 & 5 & 4 & 3 & 2 & 1 & 0 \\ 
  \hline 
  \end{tabular}
  \caption{Little Endian, começa a endereçar pelo LSB. Primeiro exemplo.}
\end{table}

Agora você quer fazer um load byte de uma posição desses 8 bytes (que compõem 
ao todo 64 bits), suponha que o endereço base para essa palavra dupla esta no 
registrador \$s0.

Na big endian, exemplo 1.

\begin{verbatim}
lb $t0, 6($s0) # $t0 conterá o valor 'B'
\end{verbatim}

E na little endian, exemplo 1.

\begin{verbatim}
lb $t0, 6($s0) # $t0 conterá o valor 'G'
\end{verbatim}

Imagine você lendo (da direita para esquerda ($\leftarrow$)) cada caractere 
dessa \verb|string "ABCDEFGH"| como um byte.

\begin{table}[ht!]
  \centering
  \begin{tabular}{|c|c|c|c|c|c|c|c|}
  \hline H & G & F & E & D & C & B & A \\ 
  \hline 7 & 6 & 5 & 4 & 3 & 2 & 1 & 0 \\ 
  \hline 
  \end{tabular}
  \caption{Big Endian, começa endereçar pelo MSB. Segundo exemplo.}
\end{table}

\pagebreak
\begin{table}[ht!]
  \centering
  \begin{tabular}{|c|c|c|c|c|c|c|c|}
  \hline A & B & C & D & E & F & G & H \\ 
  \hline 7 & 6 & 5 & 4 & 3 & 2 & 1 & 0 \\ 
  \hline 
  \end{tabular}
  \caption{Little Endian, começa a endereçar pelo LSB. Segundo exemplo.}
\end{table}

Agora você quer fazer um load byte de uma posição \textbf{5} desses 8 bytes 
(que compõem ao todo 64 bits), suponha que o endereço base para essa palavra 
dupla esta no registrador \$s0.

Na big endian, exemplo 2.

\begin{verbatim}
lb $t0, 3($s0) # $t0 conterá o valor 'D'
\end{verbatim}

E na little endian, exemplo 2.

\begin{verbatim}
lb $t0, 3($s0) # $t0 conterá o valor 'E'
\end{verbatim}

\textbf{qualquer dúvida pergunta para o seu professor.}

\section{Na sua máquina local, como verificar qual endian é.}

Se você não tem a mínima ideia se sua máquina local em que você está trabalhando
é little endian ou big endian. Aqui esta uma maneira rápida de verificar isso:

No seu terminal local (windows/linux/unix):
  \begin{verbatim}
    python -c "import sys;sys.exit('0' if sys.byteorder=='big' else '1')"
  \end{verbatim}
Essa linha, retorna o caractere 0 se ordenamento de bytes na máquina em que você
rodou esse comando é big-endian. E retorna o caractere '1' caso contrário 
(little endian).

  \begin{verbatim}
    # por exemplo, na minha máquina:
    [lucastonussi:/home/lucastonussi] 1 $ python -c "import sys;sys.exit('0' if 
    sys.byteorder=='little' else '1')"
    0
    [lucastonussi:/home/lucastonussi] 1 $ python -c "import sys;sys.exit('0' if 
    sys.byteorder=='big' else '1')"
    1
  \end{verbatim}

No caso minha máquina é (\verb|uname -a|) Linux portia 3.13.0-61--generic 
\#100-Ubuntu SMP Wed Jul 29 11:21:34 UTC 2015 x86\_64 x86\_64 x86\_64 GNU/Linux.

Existem outras maneiras, ainda mais elegantes, veja (lscpu | grep -i byte):

  \begin{verbatim}
    [lucastonussi:/home/lucastonussi] $ lscpu | grep -i byte
    Byte Order:            Little Endian
  \end{verbatim}

  \begin{verbatim}
    [lucastonussi:/home/lucastonussi] $ dpkg-architecture | grep -i end
    DEB_BUILD_ARCH_ENDIAN=little
    DEB_HOST_ARCH_ENDIAN=little
  \end{verbatim}

\end{document}
